\begin{englishabstract}
    This paper contributes to the literatures on term structure models by relating the slowly evolving persistent component to the demographic structure of population. Following \cite{geanakoplos2004demography} and \citet{favero2012demographics}, a demographic variable, namely, MY, as the ratio of middle aged (40-49) to young population (20-29), is employed to model the common persistent component of the long-term behavior of the term structure of interest rates. I then continue to consider a dynamic \citeauthor{nelson1987parsimonious} model of term structure of interest rates where the yield on bond with $n$-period to maturity  is decomposed into two components: (i) a long-term expected value subject to permanent shocks related with demographic changes at generational frequency, and (ii) a local mean reverting component occurring at business-cycle length. The model proves to fit the historical yield curve well for sample covering from 1970:01 to 2000:12. A VAR-based forecast exercise is performed to compete other term structure modes. The result is promising in sense that my model out performs others in terms of forecast RMSE. Including demographic variables in yield curve modeling provides the practitioner as well as the academic new perspective on the effect of changing demographic structure on yield curve.\\

\englishkeywords{Term Structure Model; Nelson-Siegel; Demographics}
\end{englishabstract}
