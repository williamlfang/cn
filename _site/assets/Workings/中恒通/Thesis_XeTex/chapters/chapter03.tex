% !Mode:: "TeX:UTF-8"

\chapter{公司重大事项分析}{}
\label{chap03}

\section{经营范围变更}{}
中恒容公司在成立初期经由武平县工商行政管理局报备,其一般经营项目为:工程机械、矿山机械、汽车配件等机械产品的生产、销售(筹建,未取得前置审批的批准文件,不得从事该项目的生产经营)。公司于2008年10月16日申请经营范围变更,获得在相关领域进行销售的经营许可。

\section{股东变更}{}
公司与2010年3月9日进行了股东变更的登记。在股东变更前,其股东各自持股份额与所占比例为
%%------------------------------------------------------------------------
  \begin{center}
  \begin{threeparttable}\vspace{-1.0cm}
  %%
 %\caption{股本构成}
 \renewcommand{\arraystretch}{1.1} \arrayrulewidth=0.8pt \tabcolsep=8pt
 	 \begin{tabular}{cccrr}
 \multicolumn{4}{c}{\footnotesize \bfseries 变更前股东结构}& {\small 单位:元}\\
	\hline\hline
\rowcolor{mycyan}	股东名称 	& 股东性质 & 出资方式 &  出资金额      & 持股比例  \\
	\hline \renewcommand{\arraystretch}{1}
	卢汉旺   & 国内自然人 & 货币    &  32,130,000 &  51\% \\
	谢德中   & 国内自然人 & 货币    &  12,600,000 &  20\% \\
	简可贞   & 国内自然人 & 货币    &  12,600,000 &  20\% \\ 
	张芳辉   & 国内自然人 & 货币    &   5,670,00 &  9\% \\
	\bottomrule
	\end{tabular}
\end{threeparttable}
\end{center}
%%------------------------------------------------------------------------
股东变更后,由卢汉旺和张芳辉共同持股,分别占有91\%、9\%。其他原有股东不再持股,转为公司管理层参与公司日常经营与重大事项决策。

%%------------------------------------------------------------------------
  \begin{center}
  \begin{threeparttable}\vspace{-1.0cm}
  %%
 %\caption{股本构成}
 \renewcommand{\arraystretch}{1.1} \arrayrulewidth=0.8pt \tabcolsep=8pt
 	 \begin{tabular}{cccrr}
 \multicolumn{4}{c}{\footnotesize \bfseries 变更后股东结构}& {\small 单位:元}\\
	\hline\hline
\rowcolor{mycyan}	股东名称 	& 股东性质 & 出资方式 &  出资金额      & 持股比例  \\
	\hline \renewcommand{\arraystretch}{1}
	卢汉旺   & 国内自然人 & 货币    &  57,330,000 &  91\% \\
	张芳辉   & 国内自然人 & 货币    &   5,670,000 &  9\% \\ 
	\bottomrule
	\end{tabular}
\end{threeparttable}
\end{center}
%%------------------------------------------------------------------------

\section{投资建立中恒通(吉林)机械制造有限公司}{}
由中恒容(集团)投资建立中恒通(吉林)机械制造有限公司,其长期投资总额为62 354 733.21元。项目位于吉林省磐石市明城经济开发区五金园区,是集生产汽车桥总成和汽车部件铸造加工于一体的生产基地。项目分两期建设,总投资达十亿元,一期项目总投资五亿元,占地面积83239平方米,总建筑面积57000平方米,用于建设年产6万吨汽车零部件铸锻铸造生产基地和年产10万套汽车桥总成4条装配机加生产线,投产后,年产值达十亿元以上;二期项目计划投资五亿元,建设年产十万吨汽车零部件铸锻铸造生产线,投产后,年产值达十五亿元以上。两期项目全部建成投产后年产值可实现25亿元以上,年上缴税收实现2亿元以上。