% !Mode:: "TeX:UTF-8"

\chapter{收益率曲线预测}{Forecasting Yield Curve}
\label{chap05}

\ds 作为一个重要的影响因素与\ts 的长期均衡水平的波动存在着密切的关系。这类由\ds 的变动所引致的对\tsm 的影响只有在一个世代期限范畴内才能显现效果。\dsf 本身所具有的的稳定的可预测性特征有利于增强对未来\ts 的水平变动的预测能力。建立一个由人口因素驱动的\tsm 不仅能够有助于增强\tsm 的理论解释能力,为理解影响长期均衡状态利率水平提供良好的理论支持,而在由于\dsf 的可预测性,将进一步提高对未来\tsm 变动的预测能力,从而为政策制定和市场投资提供有效的信息价值。

\section{预测模型}{Forecasting}

首先需要对样本数据划分为两个子样本,一个从~1970:01~到~1989:12~,该子样本为模型参数估计;另一个从~1990:01~到~ 2000:12,作为样本为预测比较。

\subsection{人口因素驱动的\dns}
依据本文提出的由人口因素驱动的\dns ,对~$h$~期后的国债收益率做预测:
 \begin{align}
   \hat{y}_{t+h|t}(\tau) & = \hat{\omega}(\tau) MY_{t+h|t}(\tau)
        + \hat{\beta}_{2,t+h|t} \big[\frac{1-e^{-\lambda \tau}} {\lambda \tau} \big]  + \hat{\beta}_{3,t+h|t}\big[\frac{1-e^{-\lambda \tau}} {\lambda \tau} - e^{-\lambda \tau} \big],
 \end{align}
 式中,$[
      \hat{\beta}_{2,t+h|t},
       \hat{\beta}_{3,t+h|t},
    ]'$
 由~$\mathbf{\hat{B}}_{t+h|t}$~的两个元素得到:
  \begin{align}
   \Delta\mathbf{\hat{B}}_{t+h|t}
    &=\sum_{j=0}^{h-1}\mathbf{\hat{\Gamma}}^j \mathbf{\tilde{\mu}}
     +\mathbf{\hat{\Gamma}}^h \Delta\mathbf{\hat{B}}_t.
 \end{align}

为了对比不同模型的预测效果,下面将介绍其他几种对收益率曲线的预测模型。

\subsection{比较模型}
目前文献资料已有相当多的关于收益率曲线预测的模型,这些模型各有利弊,能够为不同需求的预测目的提供相应的方案。这里仅介绍比较重要的一些\tsm。
 \begin{compactenum}[(1)]
 \item {\bf 随机游走模型}\\
 利率的随机游走模型(Random Walk)假定如果市场是有效的,则长期利率接近于随机游走。那么,提前~$h$~期的国债券收益率则应该为当期水平:
     \begin{align}
       \hat{y}_{t+h|t}(\tau)
       &= y_{t}(\tau).
     \end{align}
也就是说,对未来做最好的预测就是当前的收益率水平。
 \item {\bf 收益率的~AR(1)~模型} \\
 假定收益率服从~AR(1)~随机过程,即~$y_{t+1}(\tau)=\mu+\gamma y_{t}(\tau)+\eps_t$,那么,对未来收益率的预测则为
     \begin{align}
        \hat{y}_{t+h|t}(\tau)
       &= \hat{\mu} + \hat{\gamma}^h y_{t}(\tau)
     \end{align}
 \item {\bf 收益率的~VAR(1)~模型}\\
  在假定收益率服从~VAR(1)~随机过程的情况下,未来收益率可以通过如下方法得到预测:
  \begin{align}
     \hat{y}_{t+h|t}(\tau)
       &= \hat{c} + \hat{\Xi}^h y_{t},
  \end{align}
 式中,$\hat{\Xi}$~是一个协方差矩阵,可以通过对收益率向量做一阶滞后项回归得到。
  \item{\bf 动态~Nelson-Siegel~模型}\\
     \citeai{diebold2006forecasting}论证了\dns 所具有的一些特性,尤其是能够刻画出整个收益率曲线的动态特征,且在一阶向量自回归(VAR(1))的情况下,模型的参数只需经过两步法就能得到估计。同时,该模型在预测方向尤为突出,优于一般的模型如利率随机游走模型(Random Walk Model)、收益率自回归模型(AR(1))等模型,特别是\dns 在预测中长期收益率时均比其他模型有显著的提高。

     在该模型中,对未来~$h$~期的收益率做预测可以通过预测模型参数的动态特征得到,即
     \begin{align}
    \hat{y}_{t+h|t}(\tau) &= \hat{\beta}_{1,t+h|t}
        + \hat{\beta}_{2,t+h|t} \big[\frac{1-e^{-\lambda \tau}} {\lambda \tau} \big]
        + \hat{\beta}_{3,t+h|t}\big[\frac{1-e^{-\lambda \tau}} {\lambda \tau} - e^{-\lambda  \tau} \big], \\
    \mathbf{\hat{\beta}}_{t+h|t} &= \hat{\mathbf{\mu}} + \hat{\mathbf{\Gamma}}\mathbf{\hat{\beta}}_{t}, \nonumber
  \end{align}
  式中,$\mathbf{\beta}_t=(\beta_{1t},\beta_{2t},\beta_{3t})'$。
 \end{compactenum}

 为了比较不同模型的预测效果,本文采用均方根误差(RMSE)作为评判准则:
 \begin{align}
   RMSE &= \sqrt{\frac{\sum \big(y_{t+h}(\tau)-y_{t+h|t}(\tau)\big)^2}{132}},
 \end{align}
 式中~$y_{t+h}(\tau)$~为实际的债券收益率,$y_{t+h|t}(\tau)$~是模型预测的收益率,$n=132$~是样本区间~1990:01~到~ 2000:12~的观察点个数。

%\section{预测结果}
%{\Large \textcolor{red}{$\ldots$有待完善$\ldots$} }

