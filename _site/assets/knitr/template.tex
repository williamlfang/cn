\documentclass[color=green,mathpazo,titlestyle=hang]{elegantbook}\usepackage[]{graphicx}\usepackage[]{color}
%% maxwidth is the original width if it is less than linewidth
%% otherwise use linewidth (to make sure the graphics do not exceed the margin)
\makeatletter
\def\maxwidth{ %
  \ifdim\Gin@nat@width>\linewidth
    \linewidth
  \else
    \Gin@nat@width
  \fi
}
\makeatother

\definecolor{fgcolor}{rgb}{0.251, 0.251, 0.251}
\newcommand{\hlnum}[1]{\textcolor[rgb]{0,0.533,0.298}{#1}}%
\newcommand{\hlstr}[1]{\textcolor[rgb]{0,0.533,0.298}{#1}}%
\newcommand{\hlcom}[1]{\textcolor[rgb]{1,0.314,0.314}{#1}}%
\newcommand{\hlopt}[1]{\textcolor[rgb]{0.251,0.251,0.251}{#1}}%
\newcommand{\hlstd}[1]{\textcolor[rgb]{0.251,0.251,0.251}{#1}}%
\newcommand{\hlkwa}[1]{\textcolor[rgb]{0.502,0.627,0.188}{#1}}%
\newcommand{\hlkwb}[1]{\textcolor[rgb]{0.69,0.424,0.345}{#1}}%
\newcommand{\hlkwc}[1]{\textcolor[rgb]{0.188,0.631,0.533}{#1}}%
\newcommand{\hlkwd}[1]{\textcolor[rgb]{0.6,0,0}{#1}}%

\usepackage{framed}
\makeatletter
\newenvironment{kframe}{%
 \def\at@end@of@kframe{}%
 \ifinner\ifhmode%
  \def\at@end@of@kframe{\end{minipage}}%
  \begin{minipage}{\columnwidth}%
 \fi\fi%
 \def\FrameCommand##1{\hskip\@totalleftmargin \hskip-\fboxsep
 \colorbox{shadecolor}{##1}\hskip-\fboxsep
     % There is no \\@totalrightmargin, so:
     \hskip-\linewidth \hskip-\@totalleftmargin \hskip\columnwidth}%
 \MakeFramed {\advance\hsize-\width
   \@totalleftmargin\z@ \linewidth\hsize
   \@setminipage}}%
 {\par\unskip\endMakeFramed%
 \at@end@of@kframe}
\makeatother

\definecolor{shadecolor}{rgb}{.97, .97, .97}
\definecolor{messagecolor}{rgb}{0, 0, 0}
\definecolor{warningcolor}{rgb}{1, 0, 1}
\definecolor{errorcolor}{rgb}{1, 0, 0}
\newenvironment{knitrout}{}{} % an empty environment to be redefined in TeX

\usepackage{alltt}

\author{方 莲}
\email{william.lian.fang@gmail.com}
\zhtitle{\LaTeX{}}
\zhend{写作模板}
\entitle{\LaTeX{}}
\enend{Template}
\version{0.1}
\myquote{Stay Hungray, Stay Foolish.}
\logo{logo.pdf}
\cover{cover.pdf}
  
%green color
   \definecolor{main1}{RGB}{0,120,2}
   \definecolor{seco1}{RGB}{230,90,7}
   \definecolor{thid1}{RGB}{0,160,152}
%cyan color
   \definecolor{main2}{RGB}{0,175,152}
   \definecolor{seco2}{RGB}{239,126,30}
   \definecolor{thid2}{RGB}{120,8,13}
%blue color
   \definecolor{main3}{RGB}{20,50,104}
   \definecolor{seco3}{RGB}{180,50,131}
   \definecolor{thid3}{RGB}{7,127,128}

\usepackage{makecell}
\usepackage{lipsum}
\usepackage{texnames}

%%%%%%%%%%%%%%%%%%%%%%%%%%%%%%%%%%%%%%%%%%%%%%%%%%%%%%%%%%%%%%%%%%%%%%%%%%%%%%%%%%%%%%%%%%%%%%%%%%%%%%%%%%%%%%%%%%%%%%%%
%%%%%%%%%%%%%%%%%%%%%%%%%%%%%%%%%%%%%%%%%%%%%%%%%%%%%%%%%%%%%%%%%%%%%%%%%%%%%%%%%%%%%%%%%%%%
\usepackage{setspace}           % Allows easy changes to line spacing 
\usepackage{marginnote}         % Used with todonotes package
\usepackage{enumerate}           % List formatting commands
%%%%%%%%%%%%%%%%%%%%%%%%%%%%%%%%%%%%%%%%%%%%%%%%%%%%%%%%%%%%%%%%%%%%%%%%%%%%%%%%%%%%%%%%%%%%

%%%%%%%%%%%%%%%%%%%%%%%%%%%%%%%%%%%%%%%%%%%%%%%%%%%%%%%%%%%%%%%%%%%%%%%%%%%%%%%%%%%%%%%%%%%%%%%%%%%%%%%%%%%%%%%%%%%%%%%%
\IfFileExists{upquote.sty}{\usepackage{upquote}}{}
\begin{document}


%<knitr: GLOBAL SETTING>========================================================




%>knitr: GLOBAL SETTING<========================================================


\maketitle
\tableofcontents
%%%%%%%%%%%%%%%%%%%%%%%%%%%%%%%%%%%%%%%%%%%%%%%%%%%%%%%%%%%%%%%%%%%%%%%%%%%%%%%%%%%%%%%%%%%%%%%%%%%%%%%%%%%%%%%%%%%%%%%%
\mainmatter
%%-> CHAPTER: %%%%%%%%%%%%%%%%%%%%%%%%%%%%%%%%%%%%%%%%%%%%%%%%%%%%%%%%%%%%%%%%%%%%%%%%%%%%%%%%%%%%%%%%%%%%%%%%%%%%%%%%%%

%%<- CHAPTER: %%%%%%%%%%%%%%%%%%%%%%%%%%%%%%%%%%%%%%%%%%%%%%%%%%%%%%%%%%%%%%%%%%%%%%%%%%%%%%%%%%%%%%%%%%%%%%%%%%%%%%%%%%
%%%%%%%%%%%%%%%%%%%%%%%%%%%%%%%%%%%%%%%%%%%%%%%%%%%%%%%%%%%%%%%%%%%%%%%%%%%%%%%%%%%%%%%%%%%%%%%%%%%%%%%%%%%%%%%%%%%%%%%%
%%-> SECTION: %%%%%%%%%%%%%%%%%%%%%%%%%%%%%%%%%%%%%%%%%%%%%%%%%%%%%%%%%%%%%%%%%%%%%%%%%%%%%%%%%%%%%%%%%%%%%%%%%%%%%%%%%%

%%<- SECTOIN: %%%%%%%%%%%%%%%%%%%%%%%%%%%%%%%%%%%%%%%%%%%%%%%%%%%%%%%%%%%%%%%%%%%%%%%%%%%%%%%%%%%%%%%%%%%%%%%%%%%%%%%%%%
%%%%%%%%%%%%%%%%%%%%%%%%%%%%%%%%%%%%%%%%%%%%%%%%%%%%%%%%%%%%%%%%%%%%%%%%%%%%%%%%%%%%%%%%%%%%%%%%%%%%%%%%%%%%%%%%%%%%%%%%
%%-> CHAPTER: %%%%%%%%%%%%%%%%%%%%%%%%%%%%%%%%%%%%%%%%%%%%%%%%%%%%%%%%%%%%%%%%%%%%%%%%%%%%%%%%%%%%%%%%%%%%%%%%%%%%%%%%%%
\chapter{\LaTeX{} 模板}
%%-> SECTION: %%%%%%%%%%%%%%%%%%%%%%%%%%%%%%%%%%%%%%%%%%%%%%%%%%%%%%%%%%%%%%%%%%%%%%%%%%%%%%%%%%%%%%%%%%%%%%%%%%%%%%%%%%
\section{缘由}
我从研究生才开始接触 \LaTeX{} 排版系统,初相识,便一发不可收的爱上了这个伟大的排版工具。

说句惭愧的话,之前一直都在使用 Microsoft 出品的 Word 来编辑所有的文本与文档,也没有觉得有任何的问题,这就像那只坐在井底的青蛙,从小到大甚至老死也不会对井盖形状锅底大小的天空抱有任何的怀疑与诘问。在日常的工作与学习中,我们使用 Word 其实也是够用的,偶尔使用一些数学公式与索引,也都是通过手动的方式来实现的,根本不需要任何更加复杂的工具。可是等到了研究生阶段,我们对文档的处理工作量急剧增加,尤其是相关的课程如果有用到大量的数学公式编辑、图形的索引、排版的标准化,这些已经远远超出了我们手动能够编辑实现的程度了。而如果在处理这些技术类的文档时,使用 Word 来处理是异常的复杂(并非不可实现)。而且还面临者一项更加要命的「风险」:万恶的 MS 基本上会间断性的发作更新版本,美名其曰修改 Bugs,实质上是为了再次收取那高额的软件费。由于各个 Word 排版基本是不兼容的,往往这个文档到啦下次打开的时候就悲剧成了一堆乱码,不可理喻。因此,有了多次受伤的经历之后,前车之鉴的教训使得我只好彻底放弃了 MS 的办公系统,转向 \LaTeX{} 伟大的阵营。可谓「无心」之遇,带来「终身」受用。

%%<- SECTOIN: %%%%%%%%%%%%%%%%%%%%%%%%%%%%%%%%%%%%%%%%%%%%%%%%%%%%%%%%%%%%%%%%%%%%%%%%%%%%%%%%%%%%%%%%%%%%%%%%%%%%%%%%%%
%%----------------------------------------------------------------------------------------------------------------------
%%-> SECTION: %%%%%%%%%%%%%%%%%%%%%%%%%%%%%%%%%%%%%%%%%%%%%%%%%%%%%%%%%%%%%%%%%%%%%%%%%%%%%%%%%%%%%%%%%%%%%%%%%%%%%%%%%%
\section{关于这个文档}
\LaTeX{} 其实是非常结构化的一种排版工具,我们往往只需要一个简单的命令就可以调节一个系列的格式输出。比如,对于章节标题的设置,可以在导言区设定,那么在后面的编辑时便不用再去考虑章节标题的格式化问题了,因为这些都是已经固定好了的。这与 Word 相比具有极大的优势,在那里,我们需要针对每一个地方都重复的做一遍格式的调整,一来琐碎浪费大量时间,二来往往调整的格式并不一致。可以说,正是这种通过对全局格式的精确调整与控制,使得 \LaTeX{} 能够输出整体效果一致的精美文档,同时,也正是这样的特点,使得我们在使用 \LaTeX{} 时不需要太多的去关注格式的问题,从而把主要的精力放在写作内容上。当然,现在有更加便利的工具,Markdown,完全摈弃了对格式的考虑,而是将作者全部的心思都集中到写作上,不过由于其过于简单,依然无法输出较为精确的文档。因此,我在这里主要推荐如何使用 \LaTeX{} 来完成一份页面精美、结构一致、高度模块化的排版输出文件。

因此,这个文档将从整体布局开始,逐渐深入到对每一个细节的处理,我会具体给出每一个效果实现得到的代码,并在旁边附注实现的效果图。由于我现在主要在做数据处理与图形开发,在工作当中大量的使用 R ,所以这个文档也是结合了 R 与 \LaTeX{} 的优质特征,使用 RStudio 编辑器,处理的格式为 .Rnw,最后输出的结果是一个 PDF 格式的文件。
%%<- SECTOIN: %%%%%%%%%%%%%%%%%%%%%%%%%%%%%%%%%%%%%%%%%%%%%%%%%%%%%%%%%%%%%%%%%%%%%%%%%%%%%%%%%%%%%%%%%%%%%%%%%%%%%%%%%%
%%----------------------------------------------------------------------------------------------------------------------
%%<- CHAPTER: %%%%%%%%%%%%%%%%%%%%%%%%%%%%%%%%%%%%%%%%%%%%%%%%%%%%%%%%%%%%%%%%%%%%%%%%%%%%%%%%%%%%%%%%%%%%%%%%%%%%%%%%%%

%%-> CHAPTER: %%%%%%%%%%%%%%%%%%%%%%%%%%%%%%%%%%%%%%%%%%%%%%%%%%%%%%%%%%%%%%%%%%%%%%%%%%%%%%%%%%%%%%%%%%%%%%%%%%%%%%%%%%
\chapter{Note 类}
%%-> SECTION: %%%%%%%%%%%%%%%%%%%%%%%%%%%%%%%%%%%%%%%%%%%%%%%%%%%%%%%%%%%%%%%%%%%%%%%%%%%%%%%%%%%%%%%%%%%%%%%%%%%%%%%%%%
\section{注释}
我们在一些文档中有看到大量的注释。所谓的注释,一方面可能是由于这个信息与正文的节奏不吻合,不大适合放在此处;另一方面,我们又不想让这个信息与读者失之交臂,希望利用这个注释提供更多的信息。我们在阅读时,如果时间匆忙,可以暂时略过此处继续下文;而如果在今后想要更多的获取相关的咨询,则希望该处的信息能给自己多一些思考的素材。因此,注释是一个文档重要的部分。


%%<- SECTOIN: %%%%%%%%%%%%%%%%%%%%%%%%%%%%%%%%%%%%%%%%%%%%%%%%%%%%%%%%%%%%%%%%%%%%%%%%%%%%%%%%%%%%%%%%%%%%%%%%%%%%%%%%%%
%%-> SECTION: %%%%%%%%%%%%%%%%%%%%%%%%%%%%%%%%%%%%%%%%%%%%%%%%%%%%%%%%%%%%%%%%%%%%%%%%%%%%%%%%%%%%%%%%%%%%%%%%%%%%%%%%%%
\section{旁注}
在日常的工作中,我们常常需要对某个文档的细节做一个备注,这个往往体现为 Footnote,即「脚注」,一般位于页面的最底层。可是这个有时候查找不方便,所谓注释的本来目的,就是为了能够在情急之下迅速的找到需要的信息。因此,我们希望能够及时的在相应的文本旁边添加旁注信息。
%%<- SECTOIN: %%%%%%%%%%%%%%%%%%%%%%%%%%%%%%%%%%%%%%%%%%%%%%%%%%%%%%%%%%%%%%%%%%%%%%%%%%%%%%%%%%%%%%%%%%%%%%%%%%%%%%%%%%

%%<- CHAPTER: %%%%%%%%%%%%%%%%%%%%%%%%%%%%%%%%%%%%%%%%%%%%%%%%%%%%%%%%%%%%%%%%%%%%%%%%%%%%%%%%%%%%%%%%%%%%%%%%%%%%%%%%%%




\end{document}
