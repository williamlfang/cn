%% start of file `template.tex'.
%% Copyright 2006-2010 Xavier Danaux (xdanaux@gmail.com).
%
% This work may be distributed and/or modified under the
% conditions of the LaTeX Project Public License version 1.3c,
% available at http://www.latex-project.org/lppl/.

% Version: 20110122-4


\documentclass[11pt,a4paper,nolmodern]{moderncv}

\usepackage{william}
\usepackage{zhfontcfg_article}
\usepackage{comment}
\usepackage{multicol}

\usepackage[english]{babel}
\linespread{1.2}                %%% 行距
% for some reason, lines take up a lot of space in itemize in English...
\newenvironment{tightitemize}
   {\begin{itemize}
   \setlength{\parskip}{0pt}}
   {\end{itemize}}


% personal data
\firstname{方}
\familyname{莲}
%\address{\kai{福建省厦门市思明区}}{\kai{曾厝铵后厝85号},361005}
\address{\kai{福建省厦门市思明南路422号}}{\kai{厦大学生公寓7\#{}206},361005}

\title{\kai{王亚南经济研究院}}
%\extrainfo{%
%\linkedin~\httplink{www.linkedin.com/in/raphink}\\%
%\octocat~\httplink{www.github.com/raphink}%%
%} % optional, remove the line if not wanted

\myquote{『Stay hungry, Stay foolish.』}{史蒂夫·乔布斯(Steve Jobs)}


%\nopagenumbers{}                             % uncomment to suppress automatic page numbering for CVs longer than one page
%----------------------------------------------------------------------------------
%            content
%----------------------------------------------------------------------------------
\begin{document}
\setmainfont{Georgia}
\setsansfont{Cambria}

\hyphenpenalty=10000
\maketitle

%%%%%%%%%%%%%%%%%%%%%%%%%%%%%%%%%%%%%%%%%%%%%%%%%%%%%%%%%%%%%%%%%%%%%%%%%%%%%
\section{教育背景}
%%-------------------------------------------------------------------------
\tlcventry{2010}{0}{厦门大学(XMU),王亚南经济研究院(WISE),硕士研究生}{}{}{}{
\begin{itemize}
  \item 研究方向:宏观金融模型(Macro-Finance),利率期限结构,应用宏观计量经济学 
  \item 学术部部长,校级优秀学生干事 
  \item 最佳助教奖:高级宏观经济学(硕士班,2011,春);资产定价(双学位,2012,秋)
  %%
  \vspace{-2ex}
  %%
  \item[\textcolor{orange}{\textbf{\kai{核心课程}}\hspace{1ex}}]
  	\begin{multicols}{2}
  		\begin{itemize}
  			\item 高级微观经济学
  			\item 高级宏观经济学
  			\item 高级计量经济学
  			\item 金融理论与资产定价
  			\item 时间序列分析与应用
  			\item 宏观经济学专题课程
  			\item 城市经济学
  			\item 微观计量与面板数据
  		\end{itemize}
  	\end{multicols}
  	%%
     \item[\textcolor{orange}{\textbf{\kai{毕业论文}}\hspace{1ex}}] {\kai 人口因素驱动的动态 Nelson-Siegel 利率期限结构模型}
   %%
  	%%
     \item[\textcolor{orange}{\textbf{\kai{工作论文}}\hspace{1ex}}] {\emph{The Fiscal Theory of Price Level}, 2011}
     \item[\textcolor{orange}{\textbf{\kai{    }}\hspace{1ex}}] {\emph{Lecture Notes on the RBC},2011}
     \item[\textcolor{orange}{\textbf{\kai{    }}\hspace{1ex}}] {\kai{货币经济理论学习笔记},2012}
   %%
\end{itemize}}
\vspace{1ex}
%%-------------------------------------------------------------------------
\tlcventry{2006}{2010}{山东大学(SDU),管理学院(SOM),管理学学士,人力资源专业}{}{}{}{
\begin{itemize}
  \item 大学四年排名:4/64,GPA:92.1/100
  \item 2009年,山东大学国家奖学金二等奖
  \item 2007年,2008年,山东大学山东大学国家奖学金三等奖
  %%
  \vspace{.8em}
  \item[\textcolor{orange}{\textbf{\kai{毕业论文}}\hspace{1ex}}] {\kai 企业组织员工职业心理健康问题及其预防措施研究}
   %%
%  \vspace{-2ex}
%  %%
%  \item[\textcolor{orange}{\textbf{\kai{核心课程}}\hspace{1ex}}]
%    \begin{multicols}{2}
%  		\begin{itemize}
%  			\item 管理学及后续课程
%  			\item 微观经济学、宏观经济学
%  			\item 基础会计与财务会计
%  			\item 人力资源管理与员工测评
%  			%\item 心理测评、员工素质拓展
%  			%\item 绩效管理与薪酬管理
%  		\end{itemize}
%  	\end{multicols}
\end{itemize}}
%%%%%%%%%%%%%%%%%%%%%%%%%%%%%%%%%%%%%%%%%%%%%%%%%%%%%%%%%%%%%%%%%%%%%%%%%%%%%
\section{能力}
%%-------------------------------------------------------------------------
\subsection{财经管理}
\cvcomputer{金融分析}{熟练掌握金融知识,银行类金融机构,利率期限理论,宏观金融与货币政策,擅长金融时间序列数据分析与挖掘}
           {财会通识}{较强的财务分析能力,熟悉资产负债表,长期关注互联网金融发展,熟练资本市场运作与企业资金运转}
\cvcomputer{团队协作}{积极参与学院各项组织活动,不断在实践中学习新知识}
           {领导能力}{担任亚南院学术部部长,有效带领团队组织多场学术讲座活动}
%%-------------------------------------------------------------------------
%%-------------------------------------------------------------------------
\subsection{计算机}
\cvcomputer{编程开发}{\textbf{\textsf{R}}, Matlab, Python, Excel, Eviews}
           {独立网站}{HTML, CSS, Markdown}
\cvcomputer{代码管理}{Git}
           {使用工具}{GitHub, Jekyll, Bootstrap, Gitcafe}
%%-------------------------------------------------------------------------
\subsection{办公}          
\cvcomputer{办公套件}{OpenOffice/LibreOffice, Microsoft Office(Words, Excel)}
           {操作系统}{GNU/Linux(Ubuntu, Mint), Windows}
\cvcomputer{排版输出}{\TeX{}, \LaTeX{}, \XeLaTeX{}, Pandoc}
           {编辑器}{Sublime Text 3,Emacs}
%%%%%%%%%%%%%%%%%%%%%%%%%%%%%%%%%%%%%%%%%%%%%%%%%%%%%%%%%%%%%%%%%%%%%%%%%%%%%
\section{经历}
%%-------------------------------------------------------------------------
\subsection{创业经历}
\tlcventry{2012}{0}{厦门『{\kai 咪豆$^+$}』客栈}{}{}{}{
\begin{itemize}
 \item 前期工作准备充足(个人经历,好友讨论可行性,问卷调查,访谈),利用厦大学生的人脉资源,后期根据实际反馈做相应调整
 \item 着眼学生群体与自助游客,提供环境舒适、交流氛围良好的家庭式住宿条件
 \item 根据目标市场定位分析,精心设计与广泛传播清新、亲民的客栈文化,突出客栈独特性
 \item 利用口碑营销与互联网媒体宣传,增进客户的认同感与粘合度
 \item 目前有稳定客源,形成了「互相推荐,评价优惠」的机制;在淘宝网有良好的评价(钻石级);建立QQ联系群,外地游客用此提前预定;与豆瓣网合作,推出了小组、小站等服务;即将建立专属客栈网页,进一步传播
\end{itemize}}
%%-------------------------------------------------------------------------\\
\vspace{.5em}
\subsection{实习经历}
\vspace{-1em}
\tldatelabelcventry{2009}{2009夏}{广东碧桂园集团,总办人力资源部}{}{}{}{
  \begin{itemize}
    \item 负责招聘网站维护,根据各部门要求筛选合格简历,并进一步安排电话面试
    \item 协办完成了两场大型员工素质拓展活动
  \end{itemize}
}
\vspace{-.5em}
\tldatelabelcventry{2009}{2009春}{天狮(济南)有限责任公司,人力资源部}{}{}{}{
  \begin{itemize}
    \item 团队共同完成公司组织结构框架设计与薪酬管理重构,成果获得『优秀奖』
  \end{itemize}
}
\vspace{-.5em}
\tldatelabelcventry{2008}{2008夏}{广东碧桂园集团,统计部}{}{}{}{
  \begin{itemize}
    \item 核查各部门所提交数据,完成数据库录入和整理
    \item 分析各楼盘业主满意度,利用统计模型解释可能存在的影响因素,并生成统计报告上传
  \end{itemize}
}

\vspace{.5em}
\subsection{社会工作}
\tlcventry{2010}{0}{厦门大学王亚南经济研究院学生会}{}{}{}{
\begin{itemize}
 \item 担任学术部长职位,主要负责安排学术类讲座、负责协调其他部门、联系与接待讲座教授
 \item 整理每场讲座报道,协助办公室编辑与制作院刊,并及时给寄送到各单位
 \item 负责全院学生信息管理,辅助建立并继续维护电子档案管理系统
\end{itemize}}
\vspace{.5em}
\tlcventry{2011}{0}{硕士班助教(高级宏观经济学)、双学位助教}{}{}{}{
\begin{itemize}
 \item 辅助任课老师编写课件及课堂所需的展示程序代码(使用\LaTeX{}与Matlab)
 \item 依据课程内容与进度,需及时制作课程笔记(独立完成,英文版本)
 \item 每周两次固定时间提供课程答疑与辅导
 \item 负责整个学期内作业、课堂测试、期中及期末考试试卷的评改,最后提交学期报告至教务处
\end{itemize}}

\vspace{.5em}
\subsection{公益活动}
\vspace{-1em}
\tldatelabelcventry{2011}{2011夏}{WISE暑期学校与夏令营}{}{}{}{
\begin{itemize}
 \item 担任暑期学校新闻报道组组长,负责接待讲座教授与来宾,全程审阅所有学术类采访
 \item 采访 Peter Bossaerts 教授,并与之交流实验经济学前沿课题
 \item 同时兼任夏令营辅导员,出色完成各项任务。小组共有4位成员被录取研究生
\end{itemize}}

\section{个人}
\subsection{语言}
\cvcomputer{汉语}{通过全国大学生普通话标准考试(山东大学)}
           {英语}{六级考试(CET-6,560); GRE(V158,Q170);托福考试(102)}
\subsection{爱好}
\cvcomputer{运动}{慢跑,马拉松(2010、2011年厦门国际马拉松,全程),足球}
           {文学}{写作(独立博客网站),读书(中、英),摄影,电影}
%%-------------------------------------------------------------------------
 
\begin{comment}
\section{语言}
\cvlanguage{汉语}{母语}{通过全国大学生普通话标准考试(山东大学)}
\cvlanguage{英语}{熟练}{全国大学英语六级考试(CET-6,560); GRE(V158,Q170);托福考试(102)}
\section{个人兴趣}

\cvhobby{运动}{慢跑,马拉松(2010、2011年厦门国际马拉松,全程)}
%\cvhobby{互联网}{\href{https://twitter.com/terro1991}{Twitter}, \href{http://facebook.com/dangfan}{facebook}, \href{http://github.com/terro}{GitHub}, stackoverflow}
\cvhobby{其他}{写作(独立博客网站),读书(中、英),摄影}
\end{comment}

\begin{comment}
\section{自我评价}
\hspace{2em}多年在外求学的经历使我获益甚多,不仅磨砺了我独立、勇敢、坚强的性格品质,而且还培育了在不同文化环境中为人处事、有效沟通的能力。在高中期间,我便有幸多次参观广东知名企业美的集团、格兰仕、白云山制药、碧桂园等,并在碧桂园集团总部实习2次总计四个月时间;自大二起,我担任微软中国(山大)研究院新闻记者,负责新闻报道、网站编辑等工作,大三下学期我与团队同学一起参与了天狮(济南)有限公司的组织结构转型与薪酬设计,同时作为课程项目获得优评;进入研究生阶段,我也积极参与学院各项任务,担任学生会学术部部长一职,多次组织大型学术讲座与经验交流会,因出色工作获得『厦门大学优秀学生干事』荣誉称号;期间,我还不断挑战自我,尝试创业,开设了一家客房规模计15间、月营业2万的家庭式客栈,综合能力得到了新的提升。

\hspace{2em}我还是一名极客,相信并坚持『科技改变生活』的理念,长期关注互联网与虚拟金融的动态发展。我使用Linux操作系统,自学C语言编写程序,开发\LaTeX{}宏包编辑文档,使用R完成数据分析,通过Google获取最新资讯,借助Python处理文本,最近在学习Julia新型编程语言。岭南文化讲究『和气生财』,齐鲁文化强调『兼济天下』,闽南文化侧重『勤劳致富』。深深汲取这些独特迥异却又融汇贯通的文化精髓,我自信能够胜任在不同环境、不同文化、不同市场中工作的各项任务要求。『求知若饥,虚心若愚』,我会在新的工作环境中不断成长,不断得到磨砺,工作兢兢业业,为人勤勤恳恳,希望公司因为我的努力而蒸蒸日上,希望社会因为我的工作而更加美好。
 

 您好:

我是一名即将毕业的厦大研究生。现欲申请贵公司职位。
   
    多年在外求学的经历使我获益甚多,不仅磨砺了我独立、勇敢、坚强的性格品质,而且还培育了在不同文化环境中为人处事、有效沟通的能力。在高中期间,我便有幸多次参观广东知名企业美的集团、格兰仕、白云山制药、碧桂园等,并在碧桂园集团总部实习2次总计四个月时间;自大二起,我担任微软中国(山大)研究院新闻记者,负责新闻报道、网站编辑等工作,大三下学期我与团队同学一起参与了天狮(济南)有限公司的组织结构转型与薪酬设计,同时作为课程项目获得优评;进入研究生阶段,我也积极参与学院各项任务,担任学生会学术部部长一职,多次组织大型学术讲座与经验交流会,因出色工作获得『厦门大学优秀学生干事』荣誉称号;期间,我还不断挑战自我,尝试创业,开设了一家客房规模计15间、月营业2万的家庭式客栈,综合能力得到了新的提升。

    我还是一名极客,相信并坚持『科技改变生活』的理念,长期关注互联网与虚拟金融的动态发展。我使用Linux操作系统,自学C语言编写程序,开发\LaTeX{}宏包编辑文档,使用R完成数据分析,通过Google获取最新资讯,借助Python处理文本,最近在学习Julia新型编程语言。岭南文化讲究『和气生财』,齐鲁文化强调『兼济天下』,闽南文化侧重『勤劳致富』。深深汲取这些独特迥异却又融汇贯通的文化精髓,我自信能够胜任在不同环境、不同文化、不同市场中工作的各项任务要求。『求知若饥,虚心若愚』,我会在新的工作环境中不断成长,不断得到磨砺,工作兢兢业业,为人勤勤恳恳,希望公司因为我的努力而蒸蒸日上,希望社会因为我的工作而更加美好。

祝好

方莲, Email: william.lian.fang@gmail.com

\end{comment}
%\renewcommand{\listitemsymbol}{-} % change the symbol for lists

% Publications from a BibTeX file without multibib\renewcommand*{\bibliographyitemlabel}{\@biblabel{\arabic{enumiv}}}% for BibTeX numerical labels
%\nocite{*}
%\bibliographystyle{plain}
%\bibliography{publications}       % 'publications' is the name of a BibTeX file

% Publications from a BibTeX file using the multibib package
%\section{Publications}
%\nocitebook{book1,book2}
%\bibliographystylebook{plain}
%\bibliographybook{publications}   % 'publications' is the name of a BibTeX file
%\nocitemisc{misc1,misc2,misc3}
%\bibliographystylemisc{plain}
%\bibliographymisc{publications}   % 'publications' is the name of a BibTeX file

\end{document}

