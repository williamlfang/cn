% !Mode:: "TeX:UTF-8"

%%%%%%%%%%%%%%%%%%%%%%%%%%%%%%%%%%%%%%%%%%%%%%%%%%%%%%%%%%%%%%%%%%%%%%%%%%%%%%%%%%%%%%%%%%
\usepackage[refpage]{nomencl} %refeq

\makenomenclature
%% \nomenclature{$XXX$}{YYY}%
%%%%%%%%%%%%%%%%%%%%%%%%%%%%%%%%%%%%%%%%%%%%%%%%%%%%%%%%%%%%%%%%%%%%%%%%%%%%%%%%%%%%%%%%%%

%%%%%%%%%%%%%%%%%%%%%%%%%%%%%%%%%%%%%%%%%%%%%%%%%%%%%%%%%%%%%%%%%%%%%%%%%%%%%%%%%%%%%%%%%%
%% XX‘s(2000):author's + year
\makeatletter
 \newcommand{\citex}[1]{%
 \citeauthor{#1}'s\citeyearpar{#1}%
 }
\makeatother
%% XX (2000):author+year
\makeatletter
 \newcommand{\citeay}[1]{% cite author+year
 ~\citeauthor{#1}~\citeyearpar{#1}~%
 }
\makeatother
%% XX (1):author+index
\makeatletter
 \newcommand{\citeai}[1]{% cite author+index
 ~\citeauthor{#1}~(\citeyear{#1})~\cite{#1}~%
 }
\makeatother
%% \cite{}: (1)
%% \upcite{}: (1)
%% XX: author
\makeatletter
\newcommand{\citeath}[1]{~\citeauthor{#1}~}
\makeatother
%%%%%%%%%%%%%%%%%%%%%%%%%%%%%%%%%%%%%%%%%%%%%%%%%%%%%%%%%%%%%%%%%%%%%%%%%%%%%%%%%%%%%%%%%%


%%%%%%%%%%%%%%%%%%%%%%%%%%%%%%%%%%%%%%%%%%%%%%%%%%%%%%%%%%%%%%%%%%%%%%%%%%%%%%%%%%%%%%%%%%
%------------------罗马数字---------------------%
\usepackage{nicefrac}
\usepackage{amsmath,bm} % \bm
\usepackage{algorithmic}
\makeatletter
\newcommand{\rmnum}[1]{\romannumeral #1}
\newcommand{\Rmnum}[1]{\expandafter\@slowromancap\romannumeral #1@}
\makeatother
\usepackage{cases}

\newcommand{\E}{\mathbb{E}}
\newcommand{\eps}{\varepsilon}
%% 图标的数学公式
%\usepackage[font=small,format=plain,labelfont=bf,up,
%  textfont=normal,up,justification=justified,singlelinecheck=false]{caption}
%\usepackage{caption,sansmath}
%\DeclareCaptionFont{sansmath}{\sansmath}
%\captionsetup{textfont={sansmath}}%sf

\usepackage{booktabs}
\renewcommand{\arraystretch}{1.5}
\arrayrulewidth=0.8pt
\tabcolsep=12pt

\usepackage{multicol}
\usepackage{threeparttable}
%%%%%%%%%%%%%%%%%%%%%%%%%%%%%%%%%%%%%%%%%%%%%%%%%%%%%%%%%%%%%%%%%%%%%%%%%%%%%%%%%%%%%%%%%%
%%%%%%%%%%%%%%%%%%%%%%%%%%%%%%%%%%%%%%%%%%%%%%%%%%%%%%%%%%%%%%%%%%%%%%%%%%%%%%%%%%%%%%%%%%
%\usepackage[table]{xcolor}
\definecolor{mygray}{gray}{.9}
\definecolor{mypink}{rgb}{.99,.91,.95}
\definecolor{mycyan}{cmyk}{.3,0,0,0}

\usepackage{datetime}
 
%\xxivtime \, \today
%\ampmtime\, \today
%\oclock\, \today
%\thistime\, \today
%%%%%%%%%%%%%%%%%%%%%%%%%%%%%%%%%%%%%%%%%%%%%%%%%%%%%%%%%%%%%%%%%%%%%%%%%%%%%%%%%%%%%%%%%%
%%=========英文封面=========%%

%%定义设置英文封面内容的命令。
\newcommand\englishtitle[1]{\def\XMUT@value@englishtitle{#1}}
\newcommand\englishauthor[1]{\def\XMUT@value@englishauthor{#1}}
\newcommand\englishadvisor[1]{\def\XMUT@value@englishadvisor{#1}}
\newcommand\englishinstitute[1]{\def\XMUT@value@englishinstitute{#1}}
\newcommand\englishdate[1]{\def\XMUT@value@englishdate{#1}}
\newcommand\englishdegree[1]{\def\XMUT@value@englishdegree{#1}}
\newcommand\englishmajor[1]{\def\XMUT@value@englishmajor{#1}}

%%生成英文封面。
\newcommand\makeenglishtitle{%
  \cleardoublepage
  \thispagestyle{empty}
  \begin{center}
	\vspace*{20pt}
	  \sf\zihao{-1} \XMUT@value@englishtitle
	\vskip \stretch{1}
	  \bf\zihao{4} \XMUT@value@englishauthor
	\vskip \stretch{1}
	  \normalfont\zihao{4} \XMUT@label@englishadvisor
	\vskip 3pt
	  \normalfont\zihao{4} \XMUT@value@englishadvisor
	\vskip \stretch{2}
	  \normalfont\normalsize \XMUT@value@englishinstitute
	\vskip 30pt
	  \normalfont\normalsize \XMUT@value@englishdate
	\vskip 20pt
	  \it\normalsize \XMUT@label@englishstatement
  \end{center}
  \clearpage
  \if@twoside
	\thispagestyle{empty}
	\cleardoublepage
  \fi
}

%% 英文封面上的标签内容。
%%
%% labels in the english title page
%%
\def\XMUT@label@englishadvisor{Supervisor:}
\def\XMUT@label@englishstatement{Submitted in total fulfilment
  of the requirements for the degree of \XMUT@value@englishdegree \\
  in \XMUT@value@englishmajor}
%%
%% 英文封面的填写内容。
%%
%% string values filled in the english title page
%%
\def\XMUT@value@englishtitle{(English Title of Thesis)}
\def\XMUT@value@englishauthor{(Author Name)}
\def\XMUT@value@englishadvisor{(Supervisor's Name)}
\def\XMUT@value@englishinstitute{(Institute Name)}
\def\XMUT@value@englishdate{%
  \ifcase\month\or
    January\or February\or March\or April\or May\or June\or
    July\or August\or September\or October\or November\or December\fi
  , \number\year}
\def\XMUT@value@englishdegree{Ph.D.}
\def\XMUT@value@englishmajor{}
%%%%%%%%%%%%%%%%%%%%%%%%%%%%%%%%%%%%%%%%%%%%%%%%%%%%%%%%%%%%%%%%%%%%%%%%%%%%%%%%%%%%%%%%%%
%%%%%%%%%%%%%%%%%%%%%%%%%%%%%%%%%%%%%%%%%%%%%%%%%%%%%%%%%%%%%%%%%%%%%%%%%%%%%%%%%%%%%%%%%%%%
%%<-		粘贴源代码		<-%%%%%%%%%%%%%%%%%%%%%%%%%%%%%%%%%%%%%%%%%%%%%%%%%%%%%%%%%%%%%%
%%%%%%%%%%%%%%%%%%%%%%%%%%%%%%%%%%%%%%%%%%%%%%%%%%%%%%%%%%%%%%%%%%%%%%%%%%%%%%%%%%%%%%%%%%%%

\usepackage{listings}                % 粘贴源代码
\lstset{
language=R,                          %% R 
basicstyle=\footnotesize\ttfamily,
commentstyle=\ttfamily\color{gray},  %% 注释
numbers=left,                        %% 显示前面的序号
numbersep=1em,
%numberstyle=\ttfamily\color{myorange}\tiny,
numberstyle=\ttfamily\tiny,
stepnumber=1,                        %% 序号增加值
frame=shadowbox,                    %% single,
rulesepcolor=\color{red!20!green!20!blue!20},
escapeinside=``, 
xleftmargin=2em,
xrightmargin=2em, 
aboveskip=1em,
belowskip=-2em,
breaklines=true,
showspaces=false,
showstringspaces=false,   %% 在listing的代码输出中不显示"空格"
showtabs=false,
tabsize=2,
captionpos=b,
breaklines=true,
breakatwhitespace=false,
backgroundcolor=\color{yellow!10},
title=\lstname,
keywordstyle=\color{blue},
columns=flexible,
morekeywords={maketitle},
}
%%%%%%%%%%%%%%%%%%%%%%%%%%%%%%%%%%%%%%%%%%%%%%%%%%%%%%%%%%%%%%%%%%%%%%%%%%%%%%%%%%%%%%%%%%%%
%%<-		粘贴源代码		<-%%%%%%%%%%%%%%%%%%%%%%%%%%%%%%%%%%%%%%%%%%%%%%%%%%%%%%%%%%%%%%
%%%%%%%%%%%%%%%%%%%%%%%%%%%%%%%%%%%%%%%%%%%%%%%%%%%%%%%%%%%%%%%%%%%%%%%%%%%%%%%%%%%%%%%%%%
