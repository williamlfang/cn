% !Mode:: "TeX:UTF-8"

\chapter{经营情况与财务状况分析}
\label{chap02}

\section{公司战略分析}

下面我们将结合公司会计报表与行业情况,对公司整体战略及竞争策略进行分析。通过对公司所在行业的分析,投资者可以明确公司的行业地位以及所采取的竞争策略,权衡风险与收益,了解和掌握公司的发展潜力,特别是公司在价值创造与盈利能力方面的潜力。

\subsection{经营情况分析}

鑫科公司主营业务为房地产开发经营、商品房销售,现有投资建设『鑫科·永安山』、『鑫科·时代广场』共计两个项目。公司目前开发资质为三级。

公司在过去的三个年度间,主营业务(房地产开发与销售)具有显著的增长,其主营业务收入在2011-2012年间翻番,并且在接下来的一个年度间依然保持 108\%的高速增长趋势。这主要得益于公司在过去开发的两个楼盘业已进入收盘阶段,公司通过加大房屋销售量以回笼资金,实现资本的高额回报率。公司在2012年和2013年的销售利润率分别达到了30.8\%与12.7\%,其总资产报酬率分析为13.57\%和12.18\%,这表明公司一方面在扩大资本规模,大幅提升权益资本比例的同时,也积极控制成本,努力将三大费用尽可能的减低,将市场销售量有力的转化为资本的高额回报率。
\begin{note}
营业费用分别上升了11.54\%和53.81\%,说明公司在积极开拓市场份额,加大对市场开发的力度;而管理费用在逐步减少,减少幅度分别为17.04\%和-8.63\%;另外,公司的财务费用具有大幅度的减低,分别为21.69\%与94.39\%,这主要是公司大量减少长期借款从而为公司减少啦相应的利息支出。
\end{note}
%%------------------------------------------------------------------------
\renewcommand*{\arraystretch}{0.8}
\setlength{\tabcolsep}{8pt}
\begin{longtable}{>{\scriptsize}l>{\scriptsize}r>{\scriptsize}r>{\scriptsize}r>{\scriptsize}r>{\scriptsize}r}
\caption[利润表分析]{利润表分析}\\  % 1
&&&&& {\scriptsize 单位:元}\\
\hline\hline
\rowcolor{mycyan}	\hspace{3em} \bfseries 项目 	& \bfseries 2011年度\hspace{1em} & \bfseries 2012年度\hspace{1em} 	& \bfseries 2013年度\hspace{1em} &  \bfseries  11-12(\%)     & \hspace{1em} \bfseries  12-13(\%)  \\ \endfirsthead          % 2
\caption[]{利润表分析(续表)} \\ 
&&&&& {\scriptsize 单位:元}\\                        % 3
\hline\hline
\rowcolor{mycyan}	\hspace{3em} \bfseries 项目 	& \bfseries 2011年度\hspace{1em} & \bfseries 2012年度\hspace{1em} 	& \bfseries 2013年度\hspace{1em} &  \bfseries  11-12(\%)     & \hspace{1em} \bfseries  12-13(\%) \\  \endhead                % 4
\hline
\endfoot
\hline   % 内容开始
一、主营业务收入	& 52,113,958.50	&	156,510,205.50	&	326,399,687.00		&	200.32		&		108.55 \\ 
  \quad     减:主营业务成本		& 42,597,161.00		& 65,956,858.08		& 228,099,120.20		& 	54.84		& 		245.83 \\ 
     \qquad\quad             主营业务税金及附加	& 	1,099,110.33	& 	21,911,699.64	& 	34,809,443.70	& 		1,893.59		& 		58.86 \\ 
二、主营业务利润	& 	8,417,687.17	& 	68,641,647.78	& 	63,491,123.10		& 	715.45		& 		-7.50 \\ 
        \quad    加: 其他业务利润			& 		& 	& 	& 	&  \\ 
       \quad     减:营业费用	& 	2,729,389.47	& 	3,044,307.54	& 	4,682,394.70	& 		11.54		& 		53.81 \\ 
           \qquad\quad            管理费用	& 	4,289,158.89	& 	3,558,323.23	& 	3,251,395.86		& 	-17.04		& 		-8.63 \\ 
              \qquad\quad         财务费用	& 	8,885,656.99	& 	6,958,590.15	& 	390,205.28		& 	-21.69		& 		-94.39 \\ 
三、营业利润	& 	-7,486,518.18	& 	55,080,426.86		& 55,167,127.26		& 	835.73		& 		.16 \\ 
    \quad        加:投资收益		& 	& 	& 	& 	&  \\ 
         \qquad\quad         营业外收入		& 		& 	& 	12,584.00		& 	&  \\ 
        \qquad\quad          补贴收入			& 	& 	& 	& 	&  \\ 
    \quad    减:营业外支出	& 	128.31		& 1,001,073.10	& 	1,679,475.73	& 		780,098.82		& 		67.77 \\ 
四、利润总额	& 	-7,486,646.49		& 54,079,353.76		& 53,500,235.53		& 	 822.34		& 		-1.07 \\ 
    \quad     加:利息支出(财务费用)	& 	8,885,656.99	& 	6,958,590.15	& 	390,205.28		& 	-21.69		& 		-94.39 \\ 
EBIT 	& 	1,399,010.50	& 	61,037,943.91	& 	53,890,440.81		& 	4,262.94	& 			-11.71 \\ 
   \quad     减:所得税	& 	726,924.83	& 	5,869,132.71	& 	12,058,937.85	& 		707.39		& 		105.46 \\ 
五、净利润	& 	-8,213,571.32	& 	48,210,221.05	& 	41,441,297.68		& 	-686.96		& 		-14.04 \\ 
   \quad     减:应交特征税金		& 	& 	& 	& 	&  \\ 
    \quad    加:年初未分配利润	& 	-8,245,161.11	& 	-16,458,732.43	& 	31,751,488.62		& 	99.62	& 			 292.92 \\ 
            \qquad\quad            上年利润调整	& 	& 	& 	& 	&  \\ 
             \qquad\quad            其他				& 	& 	& 	& 	&  \\ 
       \quad    减:上年所得税调整				& 	& 	& 	& 	&  \\ 
七、可供所有者分配的利润	& 	-16,458,732.43	& 	31,751,488.62	& 	73,192,786.30		& 	 292.92	& 			130.52 \\ 
       \quad    加:盈余公积补助	& 	& 	& 	& 	&  \\ 
       \quad    减:提取盈余公积					& 	& 	& 	& 	&  \\ 
       \qquad\quad            其中:公积金					& 	& 	& 	& 	&  \\ 
        \qquad\quad                      任意盈余公积金					& 	& 	& 	& 	&  \\ 
         \qquad\quad                     利润归还投资						& 	& 	& 	& 	&  \\ 
        \qquad\quad                      转增资本						& 	& 	& 	& 	&  \\ 
八、年末未分配利润	& 	-16,458,732.43	& 	31,751,488.62	& 	73,192,786.30		& 	 292.92		& 		130.52 \\ 
	\bottomrule
	\end{longtable}\label{lirun}
%%------------------------------------------------------------------------

从\tabref{cash}中可以看出,公司在本年度内有大量的现金流发生。在本年度会计期末,公司的现金及现金等价物将大幅减少23816千元,同比上期减少幅度为223.36\%。由此可以看出,公司预计在未来一段时期内将严重缺乏可利用现金,资金缺口大约为23 427千元。从各项构成来看
\begin{compactenum}[(1) ]
 \item 经营性活动产生的现金流量净额相对同期减少了5 933 千元,减少幅度为11.32\%。其中,通过销售产品、提供劳务收到的现金流量增加了357 747 
 千元,增加幅度为1 193.84\%,表明公司在主营业务现金收入方面取得大幅增长,而收到的其他与主营业务有关的则是大幅减少,同比降低了78.94\%。同时,由于支付产生的现金流初导致公司在经营性活动的净现金流出共计减少5 933 千元.
 \item 公司本年度内大幅度减少长期借款,从报表上看,公司的「收到的其他与筹资活动有关的现金」出现明显下降,共计减少了69 500 千元,这是主要导致公司在未来出现资金严重缺乏的主要因素。 一方面,则在长期可以减少公司所需要支付的财务费用(利息支出),为企业所有者到来更多的利润分配,而另一方面,可能导致企业在短期内出现资金链紧张,无法满足公司日常的运营资金需要。我们也从鑫科公司的短期偿债能力看出,该公司在流动利率与速动比率方面的指标都远低于行业的阈值,这需要提请投资者留意该公司在未来的短期债务存在无法及时支付的风险。
\end{compactenum}

\renewcommand*{\arraystretch}{0.8}
\setlength{\tabcolsep}{4pt}
\begin{longtable}{>{\scriptsize}l>{\scriptsize}r>{\scriptsize}r>{\scriptsize}r>{\scriptsize}r>{\scriptsize}r}
\caption[现金流量表]{现金流量表}\\  % 1
&&&&& {\scriptsize 单位:元}\\
\hline\hline
\rowcolor{mycyan}	\hspace{3em} \bfseries 项目 	& \bfseries 2011年度\hspace{1em} & \bfseries 2012年度\hspace{1em} 	& \bfseries 2013年度\hspace{1em} &  \bfseries  11-12(\%)      & \hspace{1em} \bfseries  12-13(\%)  \\ \endfirsthead          % 2 \endfirsthead          % 2
\caption[]{现金流量表(续表)} \\ 
&&&&& {\scriptsize 单位:元}\\                        % 3
\hline\hline
\rowcolor{mycyan}	\hspace{3em} \bfseries 项目 	& \bfseries 2011年度\hspace{1em} & \bfseries 2012年度\hspace{1em} 	& \bfseries 2013年度\hspace{1em} &  \bfseries  11-12(\%)    & \hspace{1em} \bfseries  12-13(\%)  \\ \endhead         % 2 \endhead                % 4
\hline
\endfoot
\hline   % 内容开始
一、经营活动产生的现金流量 &  &  &  &  & 	\\		
\hspace{2em}销售产品、提供劳务收到的现金 & 61,544,819.61 & 29,966,152.89 & 387,713,507.50 & -51.31 & 1,193.84 \\
\hspace{2em}收到的税费返还 &   &  &  & & \\
\hspace{2em}收到的其他与经营活动有关的现金 &  & 52,669,711.40	 & 11,091,683.90	 &		 & -78.94 \\
\hspace{2em}\bfseries 现金流入小计 & 61,544,819.61& 82,635,864.29	& 398,805,191.40& 34.27	& 382.61 \\
\hspace{2em}购买商品、接受劳务支付的现金 & 46,039,284.51 & 355,026.66 & 	134,877,838.15	 & -99.23 & 37,890.90\\
\hspace{2em}支付给职工以及为职工支付的现金 & & 1,958,094.53& 	1,852,391.00& 		& 	-5.40 \\
\hspace{2em}支付的各项税费 & 1,826,035.16 & 	27,780,832.35 & 	46,868,381.55 & 		1,421.37	 & 		68.71\\
\hspace{2em}支付的其他与经营活动有关的现金 & -4,012,724.49 & 145,196.34 & 	168,743,380.34	 & 	-103.62	 & 		116,117.38\\
\hspace{2em}\bfseries 现金流出小计 & 43,852,595.18 & 	30,239,149.88 & 	352,341,991.04	 & 	-31.04	 & 		1,065.18\\
\hspace{4em}\bfseries 经营活动产生的现金流量净额 & 17,692,224.43	 & 52,396,714.41 & 	46,463,200.36 & 	196.16	 & 	-11.32 \\
\midrule
二、投资活动产生的现金流量 &  &  &  &  &  \\
\hspace{2em}收回投资所收到的现金 &  &   &  &  &  \\
\hspace{2em}取得投资收益所收到的现金 &  &  & 	 &   & \\
\hspace{2em}处置固定资产、无形资产和长期资产的现金净额 &  &  &  &  &  \\
\hspace{2em}收到的其他与投资活动有关的现金 &  &  &  & 	  & \\
\hspace{2em}\bfseries 现金流入小计 &  &   &  &  &  \\
\hspace{2em}购建固定资产、无形资产和其期资产所支付的现金 &  287,850.00 & 	2,748,650.00   &  & 854.89	 & 		-100.00 \\
\hspace{2em}投资所支付的现金 &  &  &  &   & \\
\hspace{2em}支付的其他与投资活动有关的现金 &  &  &  & 	 &  \\
\hspace{2em}\bfseries 现金流出小计&  287,850.00 & 	2,748,650.00   &  & 854.89	 & 		-100.00 \\
\hspace{4em}\bfseries 投资活动产生的现金流量净额&  287,850.00 & 	2,748,650.00   &  & 854.89	 & 		-100.00 \\
\midrule
三、筹资活动产生的现金流量 &  &  &  &  \\
\hspace{2em}吸收投资所收到的现金 &  &  &  &  \\
\hspace{2em}借款所收到的现金 &   & 69,500,000.00  &   &   & 	-100.00 \\
\hspace{2em}收到的其他与筹资活动有关的现金 &  &  &  & 	  &  \\
\hspace{2em}\bfseries 现金流入小计 &   & 	69,500,000.00  &   &   & 	-100.00\\
\hspace{2em}偿还债务所支付的现金 &13,200,000.00  & 	111,800,000.00  & 	69,500,000.00  & 		746.97	  & 		-37.84\\
\hspace{2em}分配股利、利润或偿付利息所产生的现金 &8,885,656.99  & 	6,958,590.15  & 	390,205.28	  & 	-21.69	  & 		-94.39 \\
\hspace{2em}支付的其他与筹资活动有关的现金 &  &  &  &    & \\
\hspace{2em}\bfseries 现金流出小计 & 8,885,656.99	  & 6,958,590.15  & 	390,205.28	  & 	-21.69	  & 		-94.39\\
\hspace{4em}\bfseries 筹资活动产生的现金流量净额 &-22,085,656.99	  & -49,258,590.15  & 	-69,890,205.28	  & 	-123.03	  & 		-41.88 \\
\midrule
四、汇率变动对现金的影响 &  &  &  &   & 	 \\
五、现金及现金等价物净增加额 & -4,681,282.56  & 	389,474.26  & 	-23,427,004.92  & 		-108.32	  & 		-6,115.03
 \\
\bottomrule
\end{longtable}\label{cash}

\subsection{行业分析}

公司所属房地产行业受国家宏观调控影响较大,国家宏观调控以及房地产市场形势的变化可能会对项目产生不利影响。基金期限内,国家对房地产政策、税收政策和金融政策等相关政策的调整可能会对房地产市场产生不利影响,从而影响基金财产的收益的实现。

\subsection{公司行业地位分析(暂缺)}

\subsection{竞争策略分析}


\section{财务状况分析}
鑫科公司采用公历年作为会计年度,即1月1日至12月31日为一个完整的会计年度。根据目前也已披露的报表信息,我们对公司的财务状况做如下分析。

%\begin{longtable}{|c|c|c|c|}
%\caption[cap in list]{long table first caption}\\  % 1
%\hline hf & hf & hf & hf \\ \endfirsthead          % 2
%\caption[]{(continued)} \\                         % 3
%\hline hs & hs & hs & hs \\ \endhead                % 4
%\hline       % 内容开始
%content 1 & cont 2 & cont 3 & cont 4   \\
%....................
%\hline 368,852,226.45
%\end{longtable}

\renewcommand*{\arraystretch}{0.8}
\setlength{\tabcolsep}{5pt}
\begin{longtable}{>{\scriptsize}p{8em}>{\scriptsize}r>{\scriptsize}r>{\scriptsize}r>{\scriptsize}r>{\scriptsize}r>{\scriptsize}r>{\scriptsize}r}
\caption[资产负债表水平分析表]{资产负债表水平分析表}\\  % 1
&&&&&&& {\scriptsize 单位:元}\\
\hline\hline
\rowcolor{mycyan}	\hspace{3em} \bfseries 项目 	& \bfseries 2011年度\hspace{1em} & \bfseries 2012年度\hspace{1em} 	& \bfseries 2013年度\hspace{1em} &  \bfseries  11-12(\%)  & {\bfseries \scriptsize 影响($\%$)}    & \hspace{1em} \bfseries  12-13(\%) & {\bfseries \scriptsize 影响($\%$)} \\  \endfirsthead          % 2
\caption[]{资产负债表水平分析表(续表)} \\ 
&&&&&&& {\scriptsize 单位:元}\\                        % 3
\hline\hline
\rowcolor{mycyan}	\hspace{3em} \bfseries 项目 	& \bfseries 2011年度\hspace{1em} & \bfseries 2012年度\hspace{1em} 	& \bfseries 2013年度\hspace{1em} &  \bfseries  11-12(\%)  & {\bfseries \scriptsize 影响($\%$)}    & \hspace{1em} \bfseries  12-13(\%) & {\bfseries \scriptsize 影响($\%$)} \\  \endhead                % 4
\hline
\endfoot
\hline   % 内容开始
流动资产:	&		&		&		&		&		&		&		\\
\hspace{2ex}货币资金	&	513,774.42	&	30,903,248.68	&	8,286,243.76	&	5,914.94	&	38.56	&	-73.19	&	150.45	\\
\hspace{2ex}短期投资	&		&		&		&	——	&	0.00	&	——	&	0.00	\\
\hspace{2ex}应收票据	&		&	550,000.00	&		&	——	&	0.70	&	-100.00	&	3.66	\\
\hspace{2ex}应收股利	&		&		&		&	——	&	0.00	&	——	&	0.00	\\
\hspace{2ex}应收利息	&		&		&		&	——	&	0.00	&	——	&	0.00	\\
  应收账款	&	5,000,000.00	&	116,945,122.50	&	57,900,479.00	&	2,238.90	&	142.06	&	-50.49	&	392.78	\\
\hspace{2ex}其他应收款	&	643,657.85	&	1,391,969.24	&	190,755,751.51	&	116.26	&	0.95	&	13,604.02	&	-1,259.69	\\
\hspace{2ex}预付帐款	&		&	9,700,000.00	&	3,000,000.00	&	——	&	12.31	&	-69.07	&	44.57	\\
\hspace{2ex}应收补贴款	&		&		&		&	——	&	0.00	&	——	&	0.00	\\
\hspace{2ex}存货	&	269,697,008.51	&	193,803,969.01	&	78,582,672.56	&	-28.14	&	-96.31	&	-59.45	&	766.48	\\
\hspace{2ex}待摊费用	&		&		&		&	——	&	0.00	&	——	&	0.00	\\
\hspace{2ex}一年内到期长期债券投资	&		&		&		&	——	&	0.00	&	——	&	0.00	\\
\hspace{2ex}其他流动资产	&		&		&		&	——	&	0.00	&	——	&	0.00	\\
\hspace{2ex}\hspace{2ex}\hspace{2ex}   流动资产合计	&	275,854,440.78	&	353,294,309.43	&	338,525,146.83	&	28.07	&	98.27	&	-4.18	&	98.25	\\
\midrule
长期投资:	&		&		&		&	——	&	0.00	&	——	&	0.00	\\
\hspace{2ex}长期股权投资	&	200,000.00	&	200,000.00	&	200,000.00	&	0.00	&	0.00	&	0.00	&	0.00	\\
\hspace{2ex}长期债权投资	&		&		&		&	——	&	0.00	&	——	&	0.00	\\
\hspace{2ex}长期投资合计	&	200,000.00	&	200,000.00	&	200,000.00	&	0.00	&	0.00	&	0.00	&	0.00	\\
固定资产:	&		&		&		&	——	&	0.00	&	——	&	0.00	\\
\hspace{2ex}固定资产原价	&	451,685.00	&	3,200,335.00	&	3,200,335.00	&	608.53	&	3.49	&	0.00	&	0.00	\\
\hspace{2ex}\hspace{2ex}减:累计折旧	&	105,971.11	&	1,493,161.03	&	1,756,527.90	&	1,309.03	&	1.76	&	17.64	&	-1.75	\\
\hspace{2ex}固定资产净值	&	345,713.89	&	1,707,173.97	&	1,443,807.10	&	393.81	&	1.73	&	-15.43	&	1.75	\\
\hspace{2ex}\hspace{2ex}减:固定资产减值准备	&		&		&		&	——	&	0.00	&	——	&	0.00	\\
\hspace{2ex}固定资产净额	&	345,713.89	&	1,707,173.97	&	1,443,807.10	&	393.81	&	1.73	&	-15.43	&	1.75	\\
\hspace{2ex}工程物资	&		&		&		&	——	&	0.00	&	——	&	0.00	\\
\hspace{2ex}在建工程	&		&		&		&	——	&	0.00	&	——	&	0.00	\\
\hspace{2ex}固定资产清理	&		&		&		&	——	&	0.00	&	——	&	0.00	\\
\hspace{2ex}\hspace{2ex}\hspace{2ex}  固定资产合计	&	345,713.89	&	1,707,173.97	&	1,443,807.10	&	393.81	&	1.73	&	-15.43	&	1.75	\\
\midrule
无形资产及其他资产:	&		&		&		&	——	&	0.00	&	——	&	0.00	\\
\hspace{2ex}无形资产	&		&		&		&	——	&	0.00	&	——	&	0.00	\\
\hspace{2ex}长期待摊费用	&		&		&		&	——	&	0.00	&	——	&	0.00	\\
\hspace{2ex}其他长期资产	&		&		&		&	——	&	0.00	&	——	&	0.00	\\
\hspace{2ex}无形资产及其他资产合计	&		&		&		&	——	&	0.00	&	——	&	0.00	\\
递延税项:	&		&		&		&	——	&	0.00	&	——	&	0.00	\\
\hspace{2ex}递延税款借项	&		&		&		&	——	&	0.00	&	——	&	0.00	\\
\hspace{2ex}\hspace{2ex}\hspace{2ex}资产总计	&	276,400,154.67	&	355,201,483.40	&	340,168,953.93	&	28.51	&	100.00	&	-4.23	&	100.00	\\
\midrule
负债及所有者权益	&		&		&		&	——	&	0.00	&	——	&	0.00	\\
流动负债:	&		&		&		&	——	&	0.00	&	——	&	0.00	\\
\hspace{2ex}短期借款	&		&		&		&	——	&	0.00	&	——	&	0.00	\\
\hspace{2ex}应付票据	&		&	30,000,000.00	&		&	——	&	38.07	&	-100.00	&	199.57	\\
\hspace{2ex}应付帐款	&	132,861.00	&	841,652.92	&	2,141,638.52	&	533.48	&	0.90	&	154.46	&	-8.65	\\
\hspace{2ex}预收帐款	&	14,430,861.11	&	381,931.00	&	2,101,108.00	&	-97.35	&	-17.83	&	450.13	&	-11.44	\\
\hspace{2ex}应付工资	&		&		&		&	——	&	0.00	&	——	&	0.00	\\
\hspace{2ex}应付股利	&		&		&		&	——	&	0.00	&	——	&	0.00	\\
\hspace{2ex}应交税金	&		&	2,020,708.58	&	1,779,008.86	&	——	&	2.56	&	-11.96	&	1.61	\\
\hspace{2ex}其他应交款	&		&		&		&	——	&	0.00	&	——	&	0.00	\\
\hspace{2ex}其他应付款	&	22,405,164.99	&	76,615,702.28	&	36,864,412.25	&	241.96	&	68.79	&	-51.88	&	264.44	\\
\hspace{2ex}预提费用	&		&		&		&	——	&	0.00	&	——	&	0.00	\\
\hspace{2ex}预计负债	&		&		&		&	——	&	0.00	&	——	&	0.00	\\
\hspace{2ex}一年内到期的长期负债	&		&		&		&	——	&	0.00	&	——	&	0.00	\\
\hspace{2ex}其他流动负债	&		&		&		&	——	&	0.00	&	——	&	0.00	\\
\hspace{2ex}\hspace{2ex}\hspace{2ex}  流动负债合计	&	36,968,887.10	&	109,859,994.78	&	42,886,167.63	&	197.17	&	92.50	&	-60.96	&	445.53	\\
\midrule
长期负债:	&		&		&		&	——	&	0.00	&	——	&	0.00	\\
\hspace{2ex}长期借款	&	111,800,000.00	&	69,500,000.00	&		&	-37.84	&	-53.68	&	-100.00	&	462.33	\\
\hspace{2ex}应付债券	&		&		&		&	——	&	0.00	&	——	&	0.00	\\
\hspace{2ex}长期应付款	&		&		&		&	——	&	0.00	&	——	&	0.00	\\
\hspace{2ex}专项应付款	&		&		&		&	——	&	0.00	&	——	&	0.00	\\
\hspace{2ex}其他长期负债	&		&		&		&	——	&	0.00	&	——	&	0.00	\\
\hspace{2ex}\hspace{2ex}长期负债合计	&	111,800,000.00	&	69,500,000.00	&		&	-37.84	&	-53.68	&	-100.00	&	462.33	\\
\midrule
递延税项:	&		&		&		&	——	&	0.00	&	——	&	0.00	\\
\hspace{2ex}递延税项贷项	&		&		&		&	——	&	0.00	&	——	&	0.00	\\
\hspace{2ex}\hspace{2ex}\hspace{2ex}负债总计	&	148,768,887.10	&	179,359,994.78	&	42,886,167.63	&	20.56	&	38.82	&	-76.09	&	907.86	\\
\midrule
少数股东权益	&		&		&		&	——	&	0.00	&	——	&	0.00	\\
所有者权益(或股东权益):	&		&		&		&	——	&	0.00	&	——	&	0.00	\\
\hspace{2ex}实收资本(或股本)	&	20,000,000.00	&	20,000,000.00	&		&	0.00	&	0.00	&	-100.00	&	133.04	\\
\hspace{2ex}\hspace{2ex}减:已归还投资	&		&		&		&	——	&	0.00	&	——	&	0.00	\\
\hspace{2ex}实收资本(或股本)净额	&	20,000,000.00	&	20,000,000.00	&	20,000,000.00	&	0.00	&	0.00	&	0.00	&	0.00	\\
\hspace{2ex}资本公积	&	124,090,000.00	&	124,090,000.00	&	204,090,000.00	&	0.00	&	0.00	&	64.47	&	-532.18	\\
\hspace{2ex}盈余公积	&		&		&		&	——	&	0.00	&	——	&	0.00	\\
\hspace{2ex}\hspace{2ex}其中:法定公益金	&		&		&		&	——	&	0.00	&	——	&	0.00	\\
\hspace{2ex}未分配利润	&	-16,458,732.43	&	31,751,488.62	&	73,192,786.30	&	-292.92	&	61.18	&	130.52	&	-275.68	\\
\midrule
所有者权益(或股东权益)合计	&	127,631,267.57	&	175,841,488.62	&	297,282,786.30	&	37.77	&	61.18	&	69.06	&	-807.86	\\
  负债和所有者权益(或股东权益)总计	&	276,400,154.67	&	355,201,483.40	&	340,168,953.93	&	28.51	&	100.00	&	-4.23	&	100.00	\\
\bottomrule
\end{longtable}\label{asset-debt}

\subsection{资产状况分析}
从\tabref{asset-debt}中,鑫科公司在上一个会计年度内发生的资产变动情况如下:
\begin{compactenum}[(1) ]
 \item 本期公司总资产规模减少 15 032千元,减少幅度为4.23\%,主要由于本期内短期流动性较强的货币资金出现大幅度的减少,实际减少量 22 617千元,减少幅度为73.19\%,其变动对总资产的影响为150.45\%。这些对公司在短期内的支付能力产生一定的影响。同时,我们发现公司由于资金回笼速度提升,导致本期内应收账款大量减少,幅度达到50.49\%,对总资产变动的影响为392.78\%。同样,鑫科公司的存货(楼盘)减少 59 044千元,减少幅度为59.45\%,对总资产的影响达到766.48\%。由于流动资产具有较强的资金流动性,因此公司流动资产的增加能够保证公司的偿还能力,满足对资金流动性的需要。从这些代表公司短期偿还能力的财务指标上看,公司在近期内存在无法兑现支付的风险。另一方面,公司的其他应收款项目在过去的会计年度内出现大幅度的增加,由于应收账款在未来的偿付具有不确定的特征,这进一步增加了鑫科公司在未来的一个偿付风险。
 \item 从负债角度看,公司的流动负债减少 66 973千元,同比减少 60.96\%,对总资产的影响为445.53\%。特别的,公司的应付票据减少17 296千元,降幅为100.00\%;其他应付款减少了 39 751千元,减少幅度51.88\%,对总资产的影响为264.44\%。从短期负债压力上分析,鑫科公司虽然在流动资产方面存在一定的资金缺口,但其相应的短期负债也在缩小,在一定程度上缓解了公司的短期偿还压力。而且从公司的长期负债上看,鑫科公司在过去的一个会计年度内已经没有长期负债的影响,这为公司减少财务费用(利息支付)起到啦重要的作用,间接提升了公司股东对公司权益资产的控制能力。
 \item 对所有者权益资产的分析可以看到,公司逐年增加权益资产的比重,连续两年分别增长了37.77\%、69.06\%。公司的权益资产率在2013年达到了0.87,这能够保证公司股东对公司投资决策的控制,但不利于公司使用财务杠杆扩大对资金的利用率。因此,鑫科公司可以采用财务杠杆对资金利用的放大作用,增加资产负债比率,通过债权融资等形式扩大公司资产规模,科学、有效使用外部资金。
\end{compactenum}
