\chapter{字体}

英文中的typeface和font一般都被翻译为字体,传统印刷业通常使用typeface,电脑字体通常使用font;当然也有很多人混用这两个词。包老师认为电脑字体的诸多相关概念可以划分为三个层次:

\begin{compactenum}
    \item 编码层,字符 (包括字母、数字、符号、控制码等) 的索引和编码,也就是字符集 (character set) 和字符编码 (character encoding) 。
    \item 格式层,字形 (glyph) 的定义描述方法,以及字体的文件存储格式。
    \item 显示层,字体的外在表现形式,比如字体的样式,或具体的字体。
\end{compactenum}

看到这里好奇的读者可能会问,为什么微软\indexMSFT{} Word和其他所见即所得软件可以直接用下拉菜单设置字体,同样是标记语言的HTML也可以方便地设置字体,而 \LaTeX 就要牵扯这么多概念?我们不得不承认,这是 \LaTeX 的历史局限性之一。Knuth\indexKnuth 当初设计 \TeX 时,既没有Unicode字符集和UTF-8编码,也没有TrueType和OpenType字体格式。

包老师语重心长地说,道路是曲折的,前途是光明的。这不马上就要侃到 \XeTeX 了,这些困难只是黎明前的黑暗。

\section{字符集和编码}
\label{sec:encoding}

众所周知电脑内部采用二进制编码,因为它易于用电子电路实现。所有字符在电脑内部都是用二进制表示的,字符集的二进制编码被称为字符编码,有时人们也会混用这两个术语。

1963年ANSI\indexANSI 发布了基于电报码的ASCII,这就是最早的字符编码,它用7位 (bit) 表示了$2^7=128$个字符,只能勉强覆盖英文字符。

美国人发明了电脑,他们优先考虑英语是可以理解的。后来随着电脑技术的传播,人们呼吁把字符编码扩充到8位也就是一个字节 (byte) ,可以涵盖$2^8=256$个字符。

于是ISO\indexISO 在1980年代中期推出了ISO 8859,256个字符显然也不能满足需要,所以8859被分为十几个部分:从8859-1 (西欧语言) 、8859-2 (中欧语言) ,直到8859-16 (东南欧语言) ,覆盖了大部分使用拉丁字母的语言文字。

在ISO标准完全定型之前,IBM就有一系列自己的字符编码,也就是代码页 (code page) ,比如437 (扩展ASCII) 、850 (西欧语言) 、852 (东欧语言) 。IBM\indexIBM 代码页通常被用于控制台 (console) 环境,也就是MS-DOS或Unix Shell那样的命令行环境。

微软\indexMSFT 将IBM代码页称为OEM代码页,自己定义的称为ANSI代码页,比如1252 (西欧语言) 、1250 (东欧语言) 、936 (GBK简体中文) 、950 (Big5繁体中文) 、932 (SJIS日文) 、949 (EUC-KR韩文) 等。

1981年,中国大陆推出了第一个自己的字符集标准GB2312,它是一个94×94的表,包括7445个字符。GB2312通常采用双字节的EUC-CN编码,所以后者也常常被称为GB2312编码;其实GB2312还有另一种编码方式HZ,只是不常用。

GB2312中没有朱镕基的“镕”字,于是它在1993年被扩展为GBK,包含21886个字符。GBK不是正式标准。2000年发布的GB18030包含70244个字符,采用四字节编码。GB18030之前还出现过一个GB13000,但是没有形成气候。

1990年ISO推出了通用字符集 (universal character set,UCS) ,即ISO 10646,意图一统江湖。它的容量是一百多万个字符,目前实际使用的有十万个左右。UCS有两种编码:双字节的UCS-2 和四字节的UCS-4。

ISO之外还有个希望一统江湖的组织:统一码联盟 (The Unicode Consortium)\indexUnicode ,它于1991 年推出了Unicode 1.0。后来两家组织意识到没必要做重复工作,于是双方开始合并成果,携手奔小康。Unicode从2.0版开始采用与ISO 10646-1相同的编码。

Unicode主要有三种编码:UTF-8、UTF-16、UTF-32。UTF-8使用一至四个8位编码。UTF-16用一或两个16位编码,基本上是UCS-2的超集,和ASCII不兼容。UTF-32用一个32位编码,它是UCS-4的一个子集。

IETF\indexIETF 要求所有网络协议都支持UTF-8,互联网电子邮件联盟 (Internet Mail Consortium, IMC)\indexIMC 也建议所有电子邮件软件都支持UTF-8,所以它已成为互联网上的事实标准。

\section{字体格式}

\subsection{点阵和矢量字体}

电脑字体的数据格式可以分为三大类:点阵 (bitmap) 字体、轮廓 (outline) 字体和笔画 (stroke-based) 字体。

点阵字体通过点阵来描述字形。早期的电脑受到容量和绘图速度的限制,多采用点阵字体。点阵字体后来渐渐被轮廓字体所取代,但是很多小字号字体仍然使用它,因为这种情况下轮廓字体缩放太多会导致笔画不清晰。

轮廓字体又称作矢量字体,它通过一组直线段和曲线来描述字形。轮廓字体易于通过数学函数进行缩放等变换,形成平滑的轮廓。轮廓字体的主要缺陷在于它所采用的贝塞尔曲线 (Bézier curves) 在光栅设备 (比如显示器和打印机) 上不能精确渲染,因而需要额外的补偿处理比如字体微调 (font hinting) 。但是随着电脑硬件的发展,人们一般不在意它比点阵字体多出的处理时间。

笔画字体其实也是轮廓字体,不过它描述的不是完整的字形,而是笔画。它多用于东亚文字。

\subsection{常见字体格式}

当前常见的轮廓字体格式有:Type 1、TrueType、OpenType。

1984年Adobe\indexAdobe 推出PostScript时,同时支持两种字体格式:Type 1和Type 3,它们都采用三次贝塞尔曲线。Type 1支持微调,它使用一个简化的PostScript子集;Type 3不支持微调,但它可以使用全部PostScript功能,因此既可包含轮廓字体也可包含点阵字体信息。
    
1991年Apple\indexApple 发布了TrueType,它采用二次贝塞尔曲线。二次曲线处理起来比三次曲线快,但是需要更多的点来描述。所以从TrueType到Type 1的转换是无损的,反之是有损的。1994年Apple开始研究TrueType的下一代技术:TrueType GX,它后来演变为Apple advanced typography (AAT) 。

1996年微软和Adobe联合发布了OpenType,它可以被认为是Type 1和TrueType的超集,既可使用二次曲线,也可使用三次曲线。它比起TrueType和AAT的优势还有:平台独立、开放、易于开发,并且支持更多的语言比如阿拉伯语。

早在1984年Knuth就发布了 \MF ,它与TrueType和OpenType的区别是,不直接描述字形轮廓,而描述生成轮廓的笔的轨迹。笔的形状可以是椭圆形或多边形,尺寸缩放自如,字形边缘也柔和一些。两种字体可以用同一个 \MF 文件,当然还有不同的参数。\MF 技术如此先进,却没有流行开来。对此Knuth解释道,要求一位设计字体的艺术家掌握60个参数太变态了,那是用来折磨数学家的。

Type 1和Type 3把字体的尺寸 (metrics) 信息和字形 (glyph) 信息分别存储。字体尺寸文件有AFM (Adobe font metrics) 和PFM (printer font metrics) ,字形文件有PFA (printer font ASCII) 和PFB (printer font binary) 。\LaTeX 使用的尺寸格式是TFM (TeX font metrics) 。

TrueType和OpenType则将字体数据都存在一个文件里,它们的文件后缀分别是是.ttf和.otf。\MF{}虽然用矢量图形来定义字形,实际输出的却是一种点阵格式:PK (packed raster) 。

这些字体格式按照技术先进性,从高到低依次为:OpenType、TrueType、Type 1、Type 3、PK,所以我们要优先选用OpenType和TrueType。

PostScript文件可以包含Type 1和Type 3字体,而PDF除了这两种还支持TrueType和OpenType字体。

\subsection{合纵连横}

当年Adobe\indexAdobe{}推出Type 1和Type 3时,前者收费,后者是公开的自由规范。Type 1专利许可费十分昂贵,穷人们只好用免费的Type 3。为了打破这种垄断,Apple开发了TrueType。1991年TrueType发布之后,Adobe随即公开了Type 1的规范,它从贵族堕落为平民,因而流行开来。

1980年代中后期,Adobe的大部分盈利来自于PostScript解释器的许可费。面对这种垄断局面,微软和Apple联合了起来。微软把买来的PostScript解释器TrueImage授权给Apple,Apple则把TrueType授权给微软。

微软得陇望蜀,又企图获得AAT的许可证,未遂。为了打破Apple的垄断,微软联合Adobe在1996年发布了OpenType。Adobe在2002年末将其字体库全面转向OpenType。

上面这几出精彩好戏充分展示了商场上的勾心斗角、尔虞我诈,没有永恒的伙伴,只有永恒的利益。但它同时也告诉我们,市场竞争中受益的还是广大的消费者。

\section{常见字体}

在 \ref{sec:font_style} 节中我们提到每种字体样式可以包含很多种具体的字体。为了方便读者,\autoref{tab:font} 列出一些最常见的字体。\TeX 的缺省字体是Knuth用 \MF 生成的Computer Modern;\XeTeX 的缺省字体是1997年 \AmS 发布的Latin Modern,它基于Computer Modern,但是扩展了其字符集,其封装格式有Type 1和OpenType。

\begin{table}[htbp]
\caption{常见字体}
\label{tab:font}
\centering
\begin{tabular}{llll}
    \toprule
    操作系统 & 衬线字体        & 无衬线字体 & 等宽字体 \\
    \midrule
    \multirow{3}{*}{Windows}   
             & Times New Roman & Tahoma     & Courier New \\
             & Georgia         & Verdana    & Lucida Console \\
             & Palatino Linotype & Arial    & Consolas \\
    \midrule
    \multirow{3}{*}{Mac OS}   
             & Times           & Helvetica  & Monaco \\
             & Georgia         & Lucida Grande & Courier \\
             & Times New Roman & Geneva     & Courier New \\
    \bottomrule
\end{tabular}
\end{table}

\section{字体的应用}

从理论上讲,任何电脑字体只要有TFM,\TeX 就可以使用它。然而早期的 \TeX 只能使用 \MF 生成的字体。直到 \LaTeXe 时代NFSS的出现后,Type 1和Type 3才在 \LaTeX 中得到广泛应用。后起之秀 \XeTeX 则极大程度地简化了TrueType和OpenType的配置,而且它还支持Unicode。

\texttt{latex}, \texttt{pdflatex}, \texttt{xelatex} 编译程序,\texttt{dvips} 和 \texttt{dvipdfmx} 驱动,DVI浏览器等分别采用不同的字体技术路线。本文主推 \XeLaTeX ,对其他早期技术只作简要介绍,聊以忆苦思甜。

\subsection{早期技术}

\subsubsection{latex和DVI}

我们用 \texttt{latex} 生成DVI时只需要TFM文件,因为DVI并不包含字形信息,而只包含对字体的引用。DVI浏览器显示DVI时一般使用PK,它在系统中查找相应的 \texttt{.pk} 文件,若找不到就调用 \MF 在后台自动生成。

\subsubsection{dvips}

缺省情况下,\texttt{dvips} 也会查找 \texttt{.pk},或调用 \MF 自动生成;然后把PK转换成包含点阵字体的Type 3,它的参数 \texttt{-D} 可以用来控制该点阵字体的分辨率。用 \texttt{ps2pdf} 处理含Type 3的PostScript时,输出的自然是含Type 3的PDF。

GSview在低分辨率下可以很好地渲染Type 3,Adobe Reader或Acrobat却不能,因为它们使用的Adobe Type Manager不支持包含完整PostScript的Type 3。因此含Type 3的PDF看起来会有些模糊,所以应尽量避免使用。

\texttt{dvips} 的另一个参数 \texttt{-Ppdf} 把Type 1嵌入生成的PostScript,这样再 \texttt{ps2pdf} 就能生成含Type 1的PDF。

\texttt{dvips} 不支持真正的 (native) TrueType,用户只能把TrueType先转成PK或Type 1,这样绕了个弯效果总会打些折扣。

\texttt{dvips} 的字体详细使用方法可查阅其手册\citep{Rokicki_2005}第六章,此处不赘述。

\subsubsection{dvipdfm(x)和pdflatex}

\texttt{dvipdfm} 支持PK和Type 1,它可以用一个 \texttt{t1fonts.map} 文件建立PK文件和Type 1文件之间的映射,这样生成的PDF用的就是Type 1。\texttt{dvipdfm} 也不支持真正的TrueType。\texttt{dvipdfmx} 通过正确的设置可以使用真正的TrueType,它对中日韩等东亚文字的支持也较好;只是其安装配置较繁琐,此处不赘述,实在有兴趣考古的读者可以参考lnotes第一版\citep{Huang_2008}。

pdflatex支持Type 1, TrueType,也在一定程度上支持OpenType。它的配置也很繁琐,可以参考其用户手册\citep{Han_2007}第五章。

\subsection{XeTeX}

\XeTeX 可以直接使用电脑系统字体,不再需要TFM文件。我们首先需要知道电脑上有哪些字体,\XeTeX 用一个XML文件记录系统字体路径,MikTeX用的是\texttt{localfonts.conf},TeXlive用的是 \texttt{fonts.conf}。

我们设置字体时需要字体的引用名,它和字体的文件名是不同的概念。\texttt{fc-list} 程序可以用来获取字体引用名,比如下面命令生成的 \texttt{myfonts.txt} 文件就是一份字体引用名列表。

\begin{Code}[]
fc-list > myfonts.txt
\end{Code}

带 \XeTeX 的发行包首次安装时会自动扫描这些字体目录,生成字体的缓存 (cache) 。每次系统安装了新字体时,我们需要手工运行字体缓存命令 \texttt{fc-cache},生成新的缓存。

\begin{Code}[]
fc-cache -r
\end{Code}

关于字体路径,\texttt{fc-list} 和 \texttt{fc-cache} 命令等的详细信息,可以参考Michel Goossens (1951--)\indexGoossens{} \footnote{布鲁塞尔自由大学 (Free University of Brussels) 物理系1972年学士,1978年博士。1979年加入CERN。合著\emph{\LaTeX{} Companion}、\emph{\LaTeX{} Graphics Companion}、\emph{\LaTeX{} Web Companion}等书。曾任TUG总统。} 编辑的\emph{\XeTeX{} Companion}\citep{Goossens_2010}。

\XeTeX 提供的字体命令比较原始、繁琐,Will Robertson (1981--)\indexRobertson{} \footnote{阿德雷德大学 (University of Adelaide) 机械系博士生,80后帅哥。} 的 \texttt{fontspec} 宏包提供了较好的封装。\XeLaTeX 下字体常用设置方法如下,详细信息可以参考 \texttt{fontspec} 的用户手册\citep{Robertson_2008}。

\begin{example}[h]
\begin{Code}[]
\usepackage{fontspec}
\setmainfont[Mapping=tex-text]{Times New Roman}
\setsansfont[Mapping=tex-text]{Tahoma}
\setmonofont{Courier New}
\end{Code}
\caption{\XeLaTeX 字体设置}
\label{exa:xelatex_font}
\end{example}

\autoref{exa:xelatex_font} 中的代码分别设置了衬线、无衬线和等宽字体样式对应的字体。在 \ref{sec:char_input} 节 \autoref{exa:dash} 中我们学过中划线和短划线可以用 \texttt{--} 和 \texttt{---} 来输入。\TeX 中几个短划线是相连的,\XeTeX 中缺省它们之间是有空隙的。\autoref{exa:xelatex_font} 用 \texttt{Mapping} 参数指示改回 \TeX 的方式,即去掉短划线之间的空隙。

\section{中文解决方案}

\LaTeX 对中文的支持主要有两种方法:张林波\indexZhangLinbo{} \footnote{中科院数学与系统科学研究院研究员。} 的CCT和Werner Lemberg (1968--)\indexLemberg{} \footnote{从维也纳音乐和表演艺术大学 (University of Music and Performing Arts, Vienna) 获得作曲、指挥、钢琴、乐团管理、歌手教练等五个专业文凭,后自学中文和数学。曾任职于奥地利和德国多家剧院和乐团,现任德国科布伦茨某剧院指挥。} 的CJK宏包。早期CCT比较流行,CJK后来居上;新版CCT也可以和CJK配合使用。

支持简体中文的 \LaTeX 发行版有吴凌云等人的CTeX和李树钧\indexLiShujun{} \footnote{西安交大电子系1997年学士,2003年博士。2003年香港城市大学博士后,2005年香港理工大学博士后,2007年哈根函授大学 (Distance University of Hagen) 研究员,2008年康斯坦茨大学 (University of Konstanz) 研究员。} 的ChinaTeX,支持繁体中文的有吴聪敏 (1952--)\indexWuCongmin{} \footnote{台湾大学电机学士,罗彻斯特大学经济学博士,台湾大学经济系教授。}、吴聪慧\indexWuConghui{} \footnote{新泽西理工学院电脑硕士,嘉南药理科技大学医务管理系助理教授。} 兄弟和翁鸿翎\indexWongHongling 的cwTeX,蔡奇伟\indexCaiQiwei{} \footnote{犹他电脑博士,静宜大学资讯工程系副教授。} 的PUTeX。这两个繁体中文的发行版本人都不熟悉,前两个简体中文发行版都包含MikTeX, CCT, CJK, WinEdt等。

\XeTeX 问世后,孙文昌 (1970--)\indexSunWenchang{} \footnote{南开大学数学系1993年学士,1998年博士,毕业后留校作博士后,2000年副教授,2002年教授。} 推出了 \texttt{xeCJK} 宏包,用于排版中日韩等文字,包括字体选择,标点、文字间距调整等功能。它可以被认为是CCT和CJK在某种程度上的结合。

在 \autoref{exa:xecjk} 中引用宏包时,\texttt{CJKchecksingle} 参数防止段落最后一行只有一个汉字;\texttt{CJKnumber} 参数自动载入另一个宏包 \texttt{CJKnumber},它提供的 \verb|\CJKnumber| 命令可以把阿拉伯数字转换为中文数字。

\verb|\setCJKmainfont| 命令设置了中文正文字体,它的两个参数 \texttt{BoldFont} 和 \texttt{ItalicFont} 分别设置了粗体、斜体样式对应的字体。\verb|\setCJKsansfont|, \verb|\setCJKmonofont| 命令分别设置了无衬线和等宽字体样式对应的字体。

\begin{example}[h]
\begin{Code}[numbers=left]
\usepackage[CJKchecksingle,CJKnumber]{xeCJK}
\setCJKmainfont[BoldFont={Adobe Heiti Std}, 
    ItalicFont={Adobe Kaiti Std}]{Adobe Song Std}
\setCJKsansfont{Adobe Heiti Std}
\setCJKmonofont{Adobe Fangsong Std}
\punctstyle{hangmobanjiao}
\end{Code}
\caption{xeCJK}
\label{exa:xecjk}
\end{example}

其实严格地讲中文字体并没有衬线、无衬线、等宽、斜体等概念,只是习惯上宋体用得最多,辅以黑体、仿宋、楷体,文档会显得疏落有致,不至于太沉闷。如果我们把 \autoref{exa:xelatex_font} 和 \autoref{exa:xecjk} 中的命令结合起来,就可以为中英文分别设置字体。

xeCJK宏包的详细用法可参考其用户手册\citep{Sun_2009}。另外Yin Dian的zhspacing宏包也可以完成类似功能,具体用法可参考其文档\citep{Yin_2007}。

\bibliographystyle{unsrtnat}
\bibliography{lnotes2}
