% !Mode:: "TeX:UTF-8"

\chapter{公司概况}

\section{公司基本情况}
{\bfseries 厦门怡惟服饰股份有限公司}(以下简称『厦门怡惟』)坐落于厦门市湖里区,系台湾怡惟服装有限公司的大陆独家合资子公司,首期投资超过一千万元,全权运营怡惟(EVEI)时尚女装品牌。

台湾怡惟服饰有限公司总部位于台湾高雄,自2003年成立以来,发展成为一家集研发、生产、营销、物流、信息一体化的品牌女装公司。秉承『质量先行,顾客至上,追求创新,双赢互惠』品牌经营理念,专业化的研发设计,规模化的生产物流,建立起合理的分销体系、市场网络和全方位的营销服务体系。为了向顾客提供满意的服饰产品,怡惟不惜重金力邀国内外行业内高精尖人物——新锐时尚大师、品牌运营大师等专业人士携手,共筑怡惟品牌的运营。通过8年的市场历练,精准的定位、时尚的款式、精良的工艺,EVEI时尚女装正被愈来愈多的都市女性所青睐,如今,销售门店已近200家,销售区域遍及国内各地大中城市,公司准确市场定位,发展迅速,致力打造中国强势时尚品牌女装。2011年,公司被厦门市政府评为『成长型企业』并给予政策扶持。

怡惟品牌主要面向25-40岁之间、月收入在3000以上的成熟、时尚、知性的二、三级都市女性。时尚女装是享受生活的推动力,怡惟(EVEI)致力于“Evoke:Vogue  Elegance  Intellectuality”——『引领:时尚、优雅、 知性』的设计理念,演绎着现代都市女性的时尚、优雅、知性,深受中国女性消费者喜爱。

% \smalltodol{信息来源:厦门怡惟公司网站}{厦门怡维}

厦门怡维服饰作为福建知名品牌女装,主营业务涉及品牌女装的设计、生产及销售。公司注册资本 2 245.5万元,法人代表为吴雪颖。目前公司以联营为主要合作模式,经营网点遍布福建、四川、重庆、广西、湖南、广东、江西、陕西、江苏、天津等省份及直辖市。入驻了淘宝商城、淘宝C店、京东商城、一号店、卓越亚马逊、唯品会等网络销售平台,并取得卓越的销售业绩。2011年度至2012年度,公司连续2次荣获了『厦门市成长型企业』的荣誉称号。2012年,公司品牌被中国服装协会评为『2012年度最具成长性品牌』,受到厦门市政府优惠政策的大力扶植。

\renewcommand*{\arraystretch}{1}
\setlength{\tabcolsep}{8pt}
\begin{longtable}{>{\footnotesize}r>{\footnotesize}p{10cm}}
\caption[公司概况]{厦门怡惟公司概况}\\  % 1
\hline\hline
\rowcolor{mycyan} \bfseries 项目 	& \bfseries 说明 \\  \endfirsthead          % 2 \endfirsthead          % 2
\caption[]{厦门怡惟公司概况(续表)} \\ 
\hline\hline
\rowcolor{mycyan} \bfseries 项目 	& \bfseries 说明 \\ \endhead         % 2 \endhead                % 4
\hline
\endfoot
\hline   % 内容开始
公司名称    &   厦门怡惟服饰股份有限公司 \\
英文名称	&	Xiamen Evei Garment Co., Ltd.	\\
公司类型	&	股份有限公司(自然人控股、非上市)\\
注册资本	&	 2 245.5万元 \\
法人代表	&	吴雪颖	\\
信息披露	&	滕明涛	\\
所属行业	&	纺织服装、服饰业	\\
注册地址	&	厦门市湖里区湖里大道54号第八层S1单元	\\
办公地址	&  厦门市湖里区湖里大道54号第八层S1单元	\\
经营范围	& 1、生产、加工、销售:服装、鞋帽、皮包;\newline 2、批发、零售:针纺织品、日用品、床上用品、办公用品、酒店用品、家居用品、装饰品。\\
主营业务	&	品牌女装的设计、生产及销售	\\
\bottomrule
\end{longtable}


\subsection{股本结构}
截止2013年07月18日,公司前十大股东的股本结构如下: %\smalltodol{厦门市同邦资产管理有限公司,成立于 2006 年,注册资本 5000 万元,是一家专注于挂牌上柜保荐、股权融资、项目融资、财务顾问、直接投资的综合型非银行金融服务机构。}{公司前十大股东的股本结构如下:}
%%------------------------------------------------------------------------
\renewcommand*{\arraystretch}{1}
\setlength{\tabcolsep}{8pt}
\begin{longtable}{>{\footnotesize}c>{\footnotesize}c>{\footnotesize}r>{\footnotesize}r>{\footnotesize}c}
\caption[公司前十大股东]{公司前十大股东持股情况}\\  % 1
\hline\hline
\rowcolor{mycyan} \bfseries 股东名称	&	\bfseries 股东性质	&	\bfseries 持股数量(股)	&	\bfseries 持股比例(\%)	&	\bfseries 交易是否受限	\\  \endfirsthead          % 2 \endfirsthead          % 2
\caption[]{公司前十大股东持股情况(续表)} \\ 
\hline\hline
\rowcolor{mycyan} \bfseries 股东名称	&	\bfseries 股东性质	&	\bfseries 持股数量(股)	&	\bfseries 持股比例(\%)	&	\bfseries 交易是否受限	\\   \endhead         % 2 \endhead                % 4
\hline
\endfoot
\hline   % 内容开始
吴雪颖	&	国内自然人	&	765万	&	34.06	&	是	\\
翁良庆	&	国内自然人	&	735万	&	32.73	&	是	\\
蔡亚潮	&	国内自然人	&	150万	&	6.68	&	否	\\
翁立芳	&	国内自然人	&	100万	&	4.45	&	否	\\
厦门同邦	&	其他组织	&	75万	&	3.34	&	否	\\
杨美香	&	国内自然人	&	50万	&	2.22	&	否	\\
曾桑潭	&	国内自然人	&	50万	&	2.22	&	否	\\
吴春红	&	国内自然人	&	40万	&	1.78	&	否	\\
滕明涛	&	国内自然人	&	30万	&	1.34	&	否	\\
谢翠玲	&	国内自然人	&	30万	&	1.34	&	否	\\
\midrule
合计	&	——	&	2 025万	&	90.18	&		—— 	\\
\bottomrule
\end{longtable}
\begin{note}
其中,厦门市同邦资产管理有限公司作为机构管理入股,成立于2006年,注册资本5000万元,是一家专注于挂牌上柜保荐、股权融资、项目融资、财务顾问、直接投资的综合型非银行金融服务机构。
\end{note}
%%------------------------------------------------------------------------
\subsection{组织结构}
目前,公司法定代表人为吴雪颖。公司的管理层组织架构为
%%------------------------------------------------------------------------
\renewcommand*{\arraystretch}{1}
\setlength{\tabcolsep}{8pt}
\begin{longtable}{>{\footnotesize}c>{\footnotesize}l>{\footnotesize}c>{\footnotesize}c>{\footnotesize}r>{\footnotesize}r>{\footnotesize}r}
\caption[公司主要管理者]{公司主要管理者介绍}\\  % 1
\hline\hline
\rowcolor{mycyan} \bfseries 高管名称	&	\bfseries 职位	&	\bfseries 出生年份	& \bfseries	性别	& \bfseries	学历	&	\bfseries 所持股份(股)	& \bfseries	持股比例(\%)	\\  \endfirsthead          % 2 \endfirsthead          % 2
\caption[]{公司主要管理者介绍(续表)} \\ 
\hline\hline
\rowcolor{mycyan} \bfseries 高管名称	&	\bfseries 职位	&	\bfseries 出生年份	& \bfseries	性别	& \bfseries	学历	&	\bfseries 所持股份(股)	& \bfseries	持股比例(\%)	\\   \endhead         % 2 \endhead                % 4
\hline
\endfoot
\hline   % 内容开始
吴雪颖	&	董事长	&	1973年	&	女	&	本科	&	765万	&	34.06	\\
蔡亚潮	&	副董事长	&	1968年	&	男	&	本科	&	150万	&	6.68	\\
翁立芳	&	总经理兼董事	&	1975年	&	男	&	大专	&	100万	&	4.45	\\
匡全军	&	总经理兼董事	&	1979年	&	男	&	大专	&	无	&	——	\\
郑\hspace{2ex}桦	&	董事	&	1969年	&	女	&	本科	&	30万	&	1.34	\\
曾根福	&	董事	&	1963年	&	男	&	本科	&	10万	&	0.46	\\
滕明涛	&	董事会秘书	&	1969年	&	男	&	本科	&	30万	&	1.34	\\
林明光	&	董事	&	1978年	&	男	&	中专	&	无	&	——	\\
翁良庆	&	监事会主席	&	1950年	&	男	&	初中	&	735万	&	32.73	\\
吴永会	&	监事	&	1979年	&	男	&	大专	&	无	&	——	\\
兰\hspace{2ex}丽	&	监事	&	1986年	&	女	&	大专	&	无	&	——	\\
陈群英	&	财务总监	&	1971年	&	女	&	本科	&	无	&	——	\\
\bottomrule
\end{longtable}
%%------------------------------------------------------------------------

%%------------------------------------------------------------------------
\subsection{主要管理者介绍}
厦门怡惟公司现有管理层,其具体情况如下。\\
\vspace{1ex}

\begin{mdfbox}[吴雪颖]
\end{mdfbox}

从怡惟公司主要管理层成员结构上看
\section{公司经营与财务情况}
%%------------------------------------------------------------------------
\subsection{经营情况介绍}
厦门怡惟公司的主营业务涉及品牌女装的设计、生产及销售,来自主营业务的收入在过去的两个会计年度间表现出强劲的增长态势。公司注重品牌的建设及推广,采用以联营为主,自营、电商、加盟相结合的商业运营模式,目前已将业务扩展至福建、四川等多个省份及直辖市,迅速取得了一定的市场份额。同时,公司也在积极开拓新型销售渠道,建立、完善了一套多层次、一体化的完整销售体系,不仅充分发挥实体店面客户亲和力、品牌忠诚感、口碑知名度的传统渠道的营销能力,还积极拓展网络销售渠道,入驻了淘宝商城、淘宝C店、京东商城、一号店、卓越亚马逊、唯品会等网络销售平台,并取得卓越的销售业绩。

近几年来怡惟公司在资产规模、市场占有率、盈利能力、净利润等方面增长显著,正逐步跻身为女装服饰行业的第一梯队。公司整体资产总额保持高速增长,在2011年至2013年间,分别增长了121.62\%、50.62\%,公司的总资产报酬率在三年间分别达到了15.20\%、9.03\%与18.35\%,这主要是由于公司通过减少公司负债总额,同时通过大量的权益性资产融资,增强自身的盈利创造能力。公司具有较强的研发设计团队,目前能够较好的预测和把握服饰流行动向,其主营业务收入分别上升了40.74\%、278.08\%,也显示了公司在未来面临着十分广阔的发展空间。公司主要产品的原材料主要为各种面、辅料。公司拥有系统的采购管理办法和外包业务操作流程,与一些主要供应商和外包厂商建立了稳定良好的合作关系。此外,公司通过优化工艺流程降低了单位产品原料用量,降低了原材料价格波动给公司带来的经营压力,维持着稳定的毛利率,公司在2011年至2013年间,销售毛利率分别达到41.14	\%、37.18\%和35.66\%。


与此同时,需要提请投资者关注公司的存货资产规模增长速度。由于服饰市场具有较强的时尚性与流行性,如果公司对流行时尚和市场需求判断失误,不能及时引导时尚潮流、推出迎合时尚趋势的产品,从而不能全面满足消费者需求,则其大规模的存货积压有可能无法通过市场销售转化为公司的利润收入。产品的大量滞销与积压,将对公司未来的经营业绩产生不利影响。
%%------------------------------------------------------------------------


%%------------------------------------------------------------------------

\subsection{财务情况介绍}
根据现有企业年度会计报表资料,我们将部分重要财务指标罗列如下,具体财务分析将在下一章节分析。

从盈利能力指标分析,怡惟公司在过去的三年时间内呈现出『U型』的特征,即在2011年期间,公司来自主营业务取得的收入使得其保持较为强劲的盈利能力,权益资本投资回报率(ROE)为27.28\%,往后的一年内虽然主营业务依然发展迅速,但由于公司规模扩大导致其管理费用大幅增加,由此造成权益资本回报率减低为8.83\%,其后的2013年度则通过减低相应的管理费用,增加权益资本比重,使得该年度的权益资本回报率上升为15.44\%。同样,税前投入资本利润率分别为25.55\%、7.47\%、16.65\%。

针对公司的资金流动性与债权偿还能力的分析显示,怡惟公司具备较为稳健的偿还债务能力,公司总资产负债率持续减低,流动比率与速动比率在显著增加,这保证了公司在偿还债务方面的能力,不存在无法实现兑付偿还的风险。而公司的资金周转速度逐年加快,资产使用效率也在逐步提高,为公司未来发展提供充分的资金流动性。公司的权益资产比值从2011年的40\%上升为89\%,权益乘数达到1.13,这显示了公司股东对公司管理决策权的绝对控制权,增强了公司在重大战略决策的宏观把握。
%%------------------------------------------------------------------------
%%------------------------------------------------------------------------
\renewcommand*{\arraystretch}{0.8}
\setlength{\tabcolsep}{5pt}
\begin{longtable}{>{\footnotesize}l>{\footnotesize}r>{\footnotesize}r>{\footnotesize}r>{\footnotesize}r>{\footnotesize}r}
\caption[主要财务指标]{主要财务指标}\\  % 1
\hline\hline
\rowcolor{mycyan} {\bfseries \footnotesize  项目} & {\bfseries \footnotesize  2011}\hspace{2ex} & {\bfseries \footnotesize   2012}\hspace{2ex} &   {\bfseries \footnotesize   2013}\hspace{2ex}  &{\bfseries \footnotesize  11-12(\%)} &{\bfseries \footnotesize  12-13(\%)} \\  \endfirsthead          % 2
\caption[]{主要财务指标(续表)} \\ 
\hline\hline
\rowcolor{mycyan} {\bfseries \footnotesize  项目} & {\bfseries \footnotesize  2011}\hspace{2ex} & {\bfseries \footnotesize   2012}\hspace{2ex} &   {\bfseries \footnotesize   2013}\hspace{2ex}  &{\bfseries \footnotesize  11-12(\%)} &{\bfseries \footnotesize  12-13(\%)} \\ \endhead                % 4
\hline
\endfoot
\hline   % 内容开始
总资产	&	17,931,067.33	&	39,739,359.18	&	59,853,895.87	&	121.62	&	50.62	\\
负债总额	&	10,696,565.35	&	5,775,686.01	&	5,445,518.93	&	-46.00	&	-5.72	\\
所有者权益	&	7,234,501.98	&	33,963,673.17	&	54,349,779.31	&	369.47	&	60.02	\\
\midrule
主营收入	&	8,406,216.30	&	11,831,083.55	&	44,731,367.24	&	40.74	&	278.08	\\
营业利润	&	2,668,082.02	&	2,401,723.12	&	9,140,389.39	&	-9.98	&	280.58	\\
利润总额	&	2,595,487.43	&	2,437,480.83	&	9,089,971.70	&	-6.09	&	272.92	\\
投入资本	&	10,670,756.42	&	34,869,112.38	&	54,877,398.54	&	226.77	&	57.38	\\
EBIT	&	2,726,228.00	&	2,604,015.66	&	9,138,701.35	&	-4.48	&	250.95	\\
\midrule
销售毛利率	&	41.14	&	37.18	&	35.66	&	-9.61	&	-4.09	\\
销售利润率	&	23.47	&	15.38	&	15.24	&	-34.50	&	-0.88	\\
成本费用利润率	&	47.55	&	25.87	&	25.70	&	-45.59	&	-0.66	\\
盈余现金保障倍数	&	——	&	-532.33	&	-17.60	&	——	&	-96.69	\\
总资产报酬率	&	15.20	&	9.03	&	18.35	&	-40.60	&	103.22	\\
总资产收益率	&	11.00	&	6.31	&	13.69	&	-42.67	&	117.01	\\
营业利润率	&	32.43	&	22.01	&	20.43	&	-32.13	&	-7.18	\\
\midrule
税前总资产盈利力	&	15.20	&	9.03	&	18.35	&	-40.60	&	103.22	\\
税后总资产盈利力	&	12.62	&	7.50	&	15.23	&	-40.60	&	103.22	\\
权益资本利润率(ROE)	&	27.28	&	8.83	&	15.44	&	-67.62	&	74.82	\\
税前(ROIC)	&	25.55	&	7.47	&	16.65	&	-70.77	&	122.99	\\
税后(ROIC)	&	21.21	&	6.20	&	13.82	&	-70.77	&	122.99	\\
\midrule
流动比率	&	1.50	&	5.50	&	8.01	&	267.71	&	45.62	\\
速动比率	&	1.21	&	2.44	&	3.02	&	101.62	&	23.60	\\
现金流动负债比率	&	——	&	-167.67	&	-22.03	&	——	&	-86.86	\\
已获利息倍数	&	20.85	&	15.64	&	187.54	&	-25.01	&	1,099.37	\\
营运资本需求量比率	&	29.53	&	65.37	&	63.74	&	121.34	&	-2.49	\\
总资产负债率	&	0.60	&	0.29	&	0.11	&	-52.12	&	-60.55	\\
\midrule
权益资产比	&	0.40	&	0.71	&	0.89	&	77.06	&	24.13	\\
权益乘数	&	2.48	&	1.40	&	1.13	&	-43.52	&	-19.44	\\
基于EBIT的利息保障倍数	&	20.85	&	15.64	&	187.54	&	-25.01	&	1,099.37	\\
总资产周转次数	&	0.47	&	0.41	&	0.90	&	-12.48	&	118.93	\\
固定资产周转次数	&	4.40	&	2.99	&	6.74	&	-32.07	&	125.11	\\
\bottomrule
\end{longtable}
