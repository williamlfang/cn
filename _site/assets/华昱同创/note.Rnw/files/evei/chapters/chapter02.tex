% !Mode:: "TeX:UTF-8"

\chapter{经营情况与财务状况分析}
\label{chap02}

\section{公司战略分析}

下面我们将结合公司会计报表与行业情况,对公司整体战略及竞争策略进行分析。通过对公司所在行业的分析,投资者可以明确公司的行业地位以及所采取的竞争策略,权衡风险与收益,了解和掌握公司的发展潜力,特别是公司在价值创造与盈利能力方面的潜力。

\subsection{经营情况分析}
厦门怡惟公司的主营业务涉及品牌女装的设计、生产及销售。公司近年来大举扩展业务规模,采用以联营为主,自营、电商、加盟相结合的商业运营模式,公司通过大力拓展销售网络,提升了公司的影响力。同时,怡惟公司目前拥有58家联营店,并能够保证各个联营网点均能较快速地销售各个批次的服饰产品,进一步提升了公司的盈利创造能力。
\begin{compactenum}[(1) ]
\item 公司的主营业务发展迅速,在发展初期便通过合理高效的商业运营模式,提高了公司的营业规模。从2011年至2013年,怡惟公司的主营业务收入为840.6万元、1 183.1万元和4 473.1万元,分别增长了40.74\%、278.08
\%,表现出强劲的发展态势。怡惟公司的盈利能力主要依靠产能规模的扩大,在扣除掉主营业务成本与相关费用后的营业利润在2012至2013年度间增长了280.58\%,说明公司能够将所获得的营业收入转化为利润水平,并进而由此带来了利润总额的强劲增长,增加幅度为272.92\%。
\item 厦门怡惟公司作为台湾怡惟服饰品牌的全资子公司,公司注重品牌的建设及推广,建立、完善了多元化的商业运营模式。在生产上,公司完全自制生产的比例较小,委托加工模式为公司目前的生产规模提供了很好的补充。在委托加工模式上,公司向外包厂商提供产品设计样板、工艺单以及剪裁好的面、辅料,外包厂商按照公司要求进行缝纫流程。该部分流程占服饰生产流程的80\%以上。同时,怡惟公司也在近几年开始逐步投资建立厂房、购置相关生产设备,在企业生产车间负责对车缝后的半成品进行钉扣、整烫等整体装饰流程,控制产成品质量。这种形式为公司节省了大量的固定资本和人工成本,有利于公司集中资源于高附加值的核心业务环节,提高公司在行业中的竞争力,不过会相应增加公司的管理费用,从2011年的31.7万元锐增到2012年的142.7万元,增加幅度为349.99\%,并继续增加至2013年的298.6万元,增加幅度为109.21\%。
\item 公司的盈利能力在过去的三年间保持较为稳定的发展节奏。销售毛利率在2011至2013年分别为41.14\%、37.18\%、35.66\%,销售利润率为23.47\%、15.38\%、12.24\%。不过由于怡惟公司主要依靠规模的增加,通过新设立销售网点等形式来提高市场销售份额,由此带来了管理费用的大幅度增加,从公司的成本费用利润率上看,其资金使用情况呈现下调的趋势。虽然公司在近两年内保持资产规模与营业量的大幅度提高,但公司的营业利率相比2011年分别下降了32.13\%、7.18\%。因此,如果怡惟公司在未来继续保持现有营业能力扩展的速度,同时通过优化企业管理、合理调整相关费用支出情况,则能够继续保持稳健的发展趋势,在公司规模增长的同时为投资者带来较高的投资回报。 
\end{compactenum}

%%------------------------------------------------------------------------
\renewcommand*{\arraystretch}{0.8}
\setlength{\tabcolsep}{2pt}
\begin{longtable}{>{\scriptsize}l>{\scriptsize}r>{\scriptsize}r>{\scriptsize}r>{\scriptsize}r>{\scriptsize}r>{\scriptsize}r>{\scriptsize}r}
\caption[利润表分析]{利润表分析}\\  % 1
&&&&&&& {\scriptsize 单位:元}\\
\hline\hline
\rowcolor{mycyan}	\hspace{3em} \bfseries 项目 	& \bfseries 2011年度\hspace{1em} & \bfseries 2012年度\hspace{1em} 	& \bfseries 2013年度\hspace{1em} &  \bfseries  11-12    & \hspace{1em} \bfseries  11-12(\%) &  \bfseries  12-13     & \hspace{1em} \bfseries  12-13(\%)  \\ \endfirsthead          % 2
\caption[]{利润表分析(续表)} \\ 
&&&&&&& {\scriptsize 单位:元}\\                        % 3
\hline\hline
\rowcolor{mycyan}	\hspace{3em} \bfseries 项目 	& \bfseries 2011年度\hspace{1em} & \bfseries 2012年度\hspace{1em} 	& \bfseries 2013年度\hspace{1em} &  \bfseries  11-12    & \hspace{1em} \bfseries  11-12(\%) &  \bfseries  12-13     & \hspace{1em} \bfseries  12-13(\%)  \\  \endhead                % 4
\hline
\endfoot
\hline   % 内容开始
	一、主营业务收入	&	8406216.3	&	11831083.55	&	44731367.24	&	3424867.25	&	40.74	&	32900283.69	&	278.08	\\
\hspace{2ex}	      减:主营业务成本	&	4821379.67	&	7286658.64	&	28757055.8	&	2465278.97	&	51.13	&	21470397.16	&	294.65	\\
\hspace{2ex}	          主营业务税金及附加	&	126870.75	&	145268.2	&	22683.58	&	18397.45	&	14.5	&	-122584.62	&	-84.39	\\
	二、主营业务利润(亏损以“-”号填列)	&	3457965.88	&	4399156.71	&	15951627.86	&	941190.83	&	27.22	&	11552471.15	&	262.61	\\
\hspace{2ex}	              加:其他业务利润(亏损以“-”号填列)	&		&		&		&	0	&	——	&	0	&	——	\\
\hspace{2ex}	      减:营业费用 	&	341897.59	&	403335.26	&	3775876.13	&	61437.67	&	17.97	&	3372540.87	&	836.16	\\
\hspace{2ex}	          管理费用	&	317245.7	&	1427563.5	&	2986632.69	&	1110317.8	&	349.99	&	1559069.19	&	109.21	\\
\hspace{2ex}	          财务费用	&	130740.57	&	166534.83	&	48729.65	&	35794.26	&	27.38	&	-117805.18	&	-70.74	\\
	三、营业利润(亏损以“-”号填列)	&	2668082.02	&	2401723.12	&	9140389.39	&	-266358.9	&	-9.98	&	6738666.27	&	280.58	\\
\hspace{2ex}	     加:投资收益(损失以“-”号填列)	&	-72581.66	&	35757.71	&	-150418.33	&	108339.37	&	-149.27	&	-186176.04	&	-520.66	\\
\hspace{2ex}	         补贴收入	&		&		&		&	0	&	——	&	0	&	——	\\
\hspace{2ex}	         营业外收入	&	0.48	&		&	100000.64	&	-0.48	&	-100	&	100000.64	&	——	\\
\hspace{2ex}	     减:营业外支出    	&	13.41	&		&		&	-13.41	&	-100	&	0	&	——	\\
	四、利润总额(亏损以“-”号填列)	&	2595487.43	&	2437480.83	&	9089971.7	&	-158006.6	&	-6.09	&	6652490.87	&	272.92	\\
\hspace{2ex}	      减:所得税	&	622191.16	&	618309.64	&	2272492.93	&	-3881.52	&	-0.62	&	1654183.29	&	267.53	\\
\hspace{2ex}	          少数股东权益	&		&		&		&	0	&	——	&	0	&	——	\\
	五、净利润(亏损以“-”号填列)	&	1973296.27	&	1819171.19	&	6817478.77	&	-154125.08	&	-7.81	&	4998307.58	&	274.76	\\
\hspace{2ex}	     加:年初未分配利润	&	37755.51	&	2011051.78	&	697211.69	&	1973296.27	&	5226.51	&	-1313840.09	&	-65.33	\\
\hspace{2ex}	        其他转入	&		&		&		&	0	&	——	&	0	&	——	\\
	六、可供分配的利润	&	2011051.78	&	3830222.97	&	7393620.51	&	1819171.19	&	90.46	&	3563397.54	&	93.03	\\
\hspace{2ex}	     减:提取法定盈余公积	&		&	77467.97	&	681747.88	&	77467.97	&	——	&	604279.91	&	780.04	\\
\hspace{2ex}	         提取法定公益金	&		&		&		&	0	&	——	&	0	&	——	\\
\hspace{2ex}	         提取职工奖励及福利基金	&		&		&		&	0	&	——	&	0	&	——	\\
\hspace{2ex}	         提取储备基金	&		&		&		&	0	&	——	&	0	&	——	\\
\hspace{2ex}	         提取企业发展基金	&		&		&		&	0	&	——	&	0	&	——	\\
\hspace{2ex}	         利润归还投资	&		&		&		&	0	&	——	&	0	&	——	\\
	七、可供投资者分配的利润	&	2011051.78	&	3752755	&	6711872.63	&	1741703.22	&	86.61	&	2959117.63	&	78.85	\\
\hspace{2ex}	     减:应付优先股股利	&		&		&		&	0	&	——	&	0	&	——	\\
\hspace{2ex}	         提取任意盈余公积	&		&		&		&	0	&	——	&	0	&	——	\\
\hspace{2ex}	         应付普通股股利	&		&		&	6653275	&	0	&	——	&	6653275	&	——	\\
\hspace{2ex}	         转作资本(或股本)的普通股股利	&		&	3055543.31	&		&	3055543.31	&	——	&	-3055543.31	&	-100	\\
	八、未分配利润	&	2011051.78	&	697211.69	&	58597.63	&	-1313840.09	&	-65.33	&	-638614.06	&	-91.6	\\
	\bottomrule
	\end{longtable}\label{lirun}
%%------------------------------------------------------------------------

从怡惟公司的现金流角度分析,可以看出
\begin{compactenum}[(1) ]
\item 公司的整体的现金流为负值,同比上年度减少了108.73\%,在2013年现金缺口为1.3万元,显示怡惟公司在近年内缺乏一定额度的可调用运营资金。结合公司创立年限分析,我们发现怡惟公司已经历发展初期,正逐步进入发展的高速阶段,此时对现金流的依赖性较大,公司各方面的发展均需要有一定可调用资金。因此,公司在现有阶段需要通过外部融资渠道获得一定数量的现金流入,以支撑企业的战略发展。
\item 从公司的自身发展能力分析,我们看到其在连续两年内经营性活动所产生的现金净流出,表明企业无法依靠自身经营状况维持目前的发展。虽然公司在主营业务方面表现强劲,来自商品销售收入的现金流增加了154.77\%,但由于其用于产品开发的相关费用增长迅速,导致公司的经营性现金流为负。结合公司的存货分析,我们看到怡惟公司在过去的两年内存货速度增长迅速,公司用于购买产品原材料与产品积压滞销影响了公司的资金运转。
\item  与此同时,公司用于投资的现金净流出,该部分资金主要用于投资设立厂房设备、销售网点等,同比增长了35.88\%。由于服饰市场的行业特殊情况,具有时尚性、流行性等特征,如果怡惟公司对市场行情判断较为准确,深入了解目标客户群体对产品的需求,准确定位公司在整个市场中的战略地位,则怡惟公司通过投资建立厂房、大量购买服装原材料、增加可供销售产品,那么,我们可以预见公司在未来的营业收入将有大幅度的提升,通过增加产能、扩大市场份额等提高公司的盈利能力。反之,如果该公司管理层对市场未来行情判断失误,所开发产品无法满足客户需要,则由于存货规模的增加将进一步拖累公司的现金流使用,影响企业后期发展。
\item 怡惟公司在近几年主要通过权益资本的累计来扩大公司规模,其权益资本从2011起增长了369.47\%、60.02\%。这主要是公司通过筹资活动获得的现金流,公司的权益乘数达到了1.13。这有利于增强公司股东对企业管理决策的控制,加强公司的战略执行力度,但不利于公司通过财务杠杆使用外部资金,为股东创造更多的盈利。
\end{compactenum}

\renewcommand*{\arraystretch}{0.8}
\setlength{\tabcolsep}{4pt}
\hspace*{-1cm}
\begin{longtable}{>{\scriptsize}l>{\scriptsize}r>{\scriptsize}r>{\scriptsize}r>{\scriptsize}r>{\scriptsize}r}
\caption[现金流量表]{现金流量表}\\  % 1
&&&& {\scriptsize 单位:元}\\
\hline\hline
\rowcolor{mycyan}	\hspace{3em} \bfseries 项目 	& \bfseries 2012年度\hspace{1em} & \bfseries 2013年度\hspace{1em} 	& \bfseries 变动\hspace{1em} &  \bfseries  增减(\%)    \\ \endfirsthead          % 2 \endfirsthead          % 2
\caption[]{现金流量表(续表)} \\ 
&&&& {\scriptsize 单位:元}\\                        % 3
\hline\hline
\rowcolor{mycyan}	\hspace{3em} \bfseries 项目 	& \bfseries 2012年度\hspace{1em} & \bfseries 2013年度\hspace{1em} 	& \bfseries 变动\hspace{1em} &  \bfseries  增减(\%)    \\ \endhead         % 2 \endhead                % 4
\hline
\endfoot
\hline   % 内容开始
	一、经营活动产生的现金流量:	&		&		&		&		\\
\hspace{2ex}	    销售商品、提供劳务收到的现金	&	15,704,519.95	&	40,010,667.27	&	24,306,147.32	&	154.77	\\
\hspace{2ex}	    收到的税费返还	&	  -           	&	  -           	&	——	&	——	\\
\hspace{2ex}	    收到的其他与经营活动有关的现金	&	7,745,486.02	&	414,021.58	&	-7,331,464.44	&	-94.65	\\
\hspace{3ex}\bfseries 	现金流入小计	&	23,450,005.97	&	40,424,688.85	&	16,974,682.88	&	72.39	\\
\hspace{2ex}	    购买商品、接受劳务支付的现金	&	30,227,112.68	&	35,057,742.70	&	4,830,630.02	&	15.98	\\
\hspace{2ex}	    支付给职工以及为职工支付的现金	&	538,467.87	&	  -           	&	——	&	——	\\
\hspace{2ex}	    支付的各项税费	&	  -           	&	  -           	&	——	&	——	\\
\hspace{2ex}	    支付的其他与经营活动有关的现金	&	2,368,358.59	&	6,566,729.26	&	4,198,370.67	&	177.27	\\
\hspace{3ex}\bfseries 	现金流出小计	&	33,133,939.14	&	41,624,471.96	&	8,490,532.82	&	25.62	\\
\hspace{3ex}\bfseries 经营活动产生的现金流量净额	&	-9,683,933.17	&	-1,199,783.11	&	8,484,150.06	&	87.61	\\
\midrule	二、投资活动产生的现金流量:	&		&		&	0.00	&	——	\\
\hspace{2ex}	    收回投资所收到的现金	&		&		&	0.00	&	——	\\
\hspace{2ex}	    取得投资收益所收到的现金	&	  -           	&	  -           	&	——	&	——	\\
\hspace{2ex}	 处置固定资产、无形资产和其他长期资产所收回的现金净额	&	  -           	&	  -           	&	——	&	——	\\
\hspace{2ex}	    收到的其他与投资活动有关的现金	&	  -           	&	  -           	&	——	&	——	\\
\hspace{3ex}\bfseries 	现金流入小计	&	  -           	&	  -           	&	——	&	——	\\
\hspace{2ex}	 购建固定资产、无形资产和其他长期资产所支付的现金	&	6,057,524.53	&	8,230,974.21	&	2,173,449.68	&	35.88	\\
\hspace{2ex}	    投资所支付的现金	&	  -           	&	  -           	&	——	&	——	\\
\hspace{2ex}	    支付的其他与投资活动有关的现金	&	  -           	&	  -           	&	——	&	——	\\
\hspace{3ex}\bfseries 	现金流出小计	&	6,057,524.53	&	8,230,974.21	&	2,173,449.68	&	35.88	\\
\hspace{3ex}\bfseries 	投资活动产生的现金流量净额	&	-6,057,524.53	&	-8,230,974.21	&	-2,173,449.68	&	35.88	\\
\midrule
	三、筹资活动产生的现金流量:	&		&		&	0.00	&	——	\\
\hspace{2ex}	    吸收投资所收到的现金	&	17,455,000.00	&	10,000,000.00	&	-7,455,000.00	&	-42.71	\\
\hspace{2ex}	    借款所收到的现金	&	-1,394,839.79	&	-537,474.03	&	857,365.76	&	-61.47	\\
\hspace{2ex}	    收到的其他与筹资活动有关的现金	&		&		&	0.00	&	——	\\
\hspace{3ex}\bfseries 	现金流入小计	&	16,060,160.21	&	9,462,525.97	&	-6,597,634.24	&	-41.08	\\
\hspace{2ex}	    偿还债务所支付的现金	&		&		&	0.00	&	——	\\
\hspace{2ex}	    分配股利、利润或偿付利息所支付的现金	&	163,311.62	&	45,337.31	&	-117,974.31	&	-72.24	\\
\hspace{2ex}	    支付的其他与筹资活动有关的现金	&		&		&	0.00	&	——	\\
\hspace{3ex}\bfseries 	现金流出小计	&	163,311.62	&	45,337.31	&	-117,974.31	&	-72.24	\\
\hspace{3ex}\bfseries 	筹资活动产生的现金流量净额	&	15,896,848.59	&	9,417,188.66	&	-6,479,659.93	&	-40.76	\\
\midrule
	四、汇率变动对现金的影响	&		&		&	0.00	&	——	\\
	五、现金及现金等价物净增加额	&	155,390.89	&	-13,568.66	&	-168,959.55	&	-108.73	\\
\bottomrule
\end{longtable}\label{cash}
\hspace*{-1cm}

\subsection{行业分析}

怡惟公司所处行业为服装产业(女装市场),该产业为劳动密集型产业,行业门槛较低,业内竞争十分激烈,市场化程度较高。其竞争特点表现为:在中低端产品市场,由于服装行业壁垒很低、门槛不高,目前国内大多数小型服装企业集中在该领域,市场竞争激烈;在高端产品市场,虽然进入门槛较高,但随着国际高端品牌对中国市场日益重视并加强渗透,也使得该领域竞争日趋激烈。

在女装服饰行业,哥弟、卓影、秋水伊人等与公司的消费受众、产品定位比较接近。虽然近几年来公司在市场占有率、营收规模、净利润等方面增长显著,但并未取得绝对领先的市场龙头地位。随着越来越多的国外品牌进入国内市场,公司面临行业竞争风险。

同时,我国服饰市场属于垄断竞争市场,也是典型的买方市场。随着中高端女装市场流行趋势变化速度加快,以及国内不同区域消费市场的差异,能否准确把握目标消费群的偏好变化趋势,持续开发出市场需求的产品,已成为市场竞争的关键所在。公司具有较强的研发设计团队,目前能够较好的预测和把握服饰流行动向。但如果公司对流行时尚和市场需求判断失误,不能及时引导时尚潮流、推出迎合时尚趋势的产品,从而不能全面满足消费者需求,导致消费者对本公司品牌认同度降低,开发的产品大量滞销,将对公司经营业绩产生不利影响。

\subsection{公司行业地位分析(暂缺)}

\subsection{竞争策略分析}


\section{财务状况分析}
怡惟公司采用公历年作为会计年度,即1月1日至12月31日为一个完整的会计年度。根据目前也已披露的报表信息,我们对公司的财务状况做如下分析。

%\begin{longtable}{|c|c|c|c|}
%\caption[cap in list]{long table first caption}\\  % 1
%\hline hf & hf & hf & hf \\ \endfirsthead          % 2
%\caption[]{(continued)} \\                         % 3
%\hline hs & hs & hs & hs \\ \endhead                % 4
%\hline       % 内容开始
%content 1 & cont 2 & cont 3 & cont 4   \\
%....................
%\hline 368,852,226.45
%\end{longtable}

\renewcommand*{\arraystretch}{0.8}
\setlength{\tabcolsep}{5pt}
\begin{longtable}{>{\scriptsize}p{9em}>{\scriptsize}r>{\scriptsize}r>{\scriptsize}r>{\scriptsize}r>{\scriptsize}r>{\scriptsize}r>{\scriptsize}r}
\caption[资产负债表分析表]{资产负债表水平分析表}\\  % 1
&&&&&&& {\scriptsize 单位:元}\\
\hline\hline
\rowcolor{mycyan}	\hspace{3em} \bfseries 项目 	& \bfseries 2011年度\hspace{1em} & \bfseries 2012年度\hspace{1em} 	& \bfseries 2013年度\hspace{1em} &  \bfseries  11-12(\%)  & {\bfseries \scriptsize 影响($\%$)}    & \hspace{1em} \bfseries  12-13(\%) & {\bfseries \scriptsize 影响($\%$)} \\  \endfirsthead          % 2
\caption[]{资产负债表水平分析表(续表)} \\ 
&&&&&&& {\scriptsize 单位:元}\\                        % 3
\hline\hline
\rowcolor{mycyan}	\hspace{3em} \bfseries 项目 	& \bfseries 2011年度\hspace{1em} & \bfseries 2012年度\hspace{1em} 	& \bfseries 2013年度\hspace{1em} &  \bfseries  11-12(\%)  & {\bfseries \scriptsize 影响($\%$)}    & \hspace{1em} \bfseries  12-13(\%) & {\bfseries \scriptsize 影响($\%$)} \\  \endhead                % 4
\hline
\endfoot
\hline   % 内容开始
	流动资产:	&		&		&		&		&		&		&		\\
\hspace{2ex}	    货币资金	&	216028.91	&	371419.8	&	357851.14	&	71.93	&	0.71	&	-3.65	&	-0.07	\\
\hspace{2ex}	    短期投资	&		&		&		&	——	&	- 	&	——	&	- 	\\
\hspace{2ex}	    应收票据	&		&		&		&	——	&	- 	&	——	&	- 	\\
\hspace{2ex}	    应收股利	&		&		&		&	——	&	- 	&	——	&	- 	\\
\hspace{2ex}	    应收利息	&		&		&		&	——	&	- 	&	——	&	- 	\\
\hspace{2ex}	    应收帐款	&	9026489.71	&	4834277.44	&	9566033.83	&	-46.44	&	-19.22	&	97.88	&	23.52	\\
\hspace{2ex}	    其他应收款	&	3718168.77	&	4847880.11	&	4727318.78	&	30.38	&	5.18	&	-2.49	&	-0.6	\\
\hspace{2ex}	    预付帐款	&	278874.31	&	4056262.6	&	1791875.28	&	1354.51	&	17.32	&	-55.82	&	-11.26	\\
\hspace{2ex}	    应收补贴款	&		&		&		&	——	&	- 	&	——	&	- 	\\
\hspace{2ex}	    存货	&	2752330.93	&	17641759.43	&	27151825.1	&	540.98	&	68.27	&	53.91	&	47.28	\\
\hspace{2ex}	    待摊费用	&		&		&		&	——	&	- 	&	——	&	- 	\\
\hspace{2ex}	    一年内到期长期债券投资	&		&		&		&	——	&	- 	&	——	&	- 	\\
\hspace{2ex}	    其他流动资产	&		&		&		&	——	&	- 	&	——	&	- 	\\ 
\hspace{2ex}	      流动资产合计	&	15991892.63	&	31751599.38	&	43594904.13	&	98.55	&	72.26	&	37.3	&	58.88	\\
\midrule
\hspace{2ex}	长期投资:	&		&		&		&	——	&	- 	&	——	&	- 	\\
\hspace{2ex}	    长期股权投资	&		&		&		&	——	&	- 	&	——	&	- 	\\
\hspace{2ex}	    长期债权投资	&		&		&		&	——	&	- 	&	——	&	- 	\\
\hspace{2ex}	    长期投资合计	&	- 	&	- 	&	- 	&	——	&	- 	&	——	&	- 	\\
\hspace{2ex}	固定资产:	&		&		&		&	——	&	- 	&	——	&	- 	\\
\hspace{2ex}	    固定资产原价	&	1908438.53	&	5999963.04	&	7282601.26	&	214.39	&	18.76	&	21.38	&	6.38	\\
\hspace{2ex}	        减:累计折旧	&		&		&		&	——	&	- 	&	——	&	- 	\\
\hspace{2ex}	    固定资产净值	&	1908438.53	&	5999963.04	&	7282601.26	&	214.39	&	18.76	&	21.38	&	6.38	\\
\hspace{2ex}	        减:固定资产减值准备	&		&		&		&	——	&	- 	&	——	&	- 	\\
\hspace{2ex}	    固定资产净额	&	1908438.53	&	5999963.04	&	7282601.26	&	214.39	&	18.76	&	21.38	&	6.38	\\
\hspace{2ex}	    工程物资	&		&		&		&	——	&	- 	&	——	&	- 	\\
\hspace{2ex}	    在建工程	&		&		&		&	——	&	- 	&	——	&	- 	\\
\hspace{2ex}	    固定资产清理	&		&		&		&	——	&	- 	&	——	&	- 	\\
\hspace{2ex}	              固定资产合计	&	1908438.53	&	5999963.04	&	7282601.26	&	214.39	&	18.76	&	21.38	&	6.38	\\
\midrule
\hspace{2ex}	无形资产及其他资产:	&		&		&		&	——	&	- 	&	——	&	- 	\\
\hspace{2ex}	    无形资产	&		&		&		&	——	&	- 	&	——	&	- 	\\
\hspace{2ex}	    长期待摊费用	&		&	1966000.02	&	8914336.01	&	——	&	9.01	&	353.43	&	34.54	\\
\hspace{2ex}	    其他长期资产	&		&		&		&	——	&	- 	&	——	&	- 	\\
\hspace{2ex}	    无形资产及其他资产合计	&	- 	&	1966000.02	&	8914336.01	&	——	&	9.01	&	353.43	&	34.54	\\
\hspace{2ex}	递延税项:	&		&		&		&	——	&	- 	&	——	&	- 	\\
\hspace{2ex}	    递延税款借项	&	30736.17	&	21796.74	&	62054.47	&	-29.08	&	-0.04	&	184.7	&	0.2	\\
\hspace{2ex}	            资产总计	&	17931067.33	&	39739359.18	&	59853895.87	&	121.62	&	100	&	50.62	&	100	\\
\midrule	负债及所有者权益	&		&		&		&		&		&		&		\\
\hspace{2ex}	流动负债:	&		&		&		&		&		&		&		\\
\hspace{2ex}	    短期借款	&	2300000	&	905160.21	&	367686.18	&	-60.65	&	-6.4	&	-59.38	&	-2.68	\\
\hspace{2ex}	    应付票据	&		&		&		&	——	&	- 	&	——	&	- 	\\
\hspace{2ex}	    应付帐款	&	4606501.95	&	332864.7	&	1277856.15	&	-92.77	&	-19.6	&	283.9	&	4.71	\\
\hspace{2ex}	    预收帐款	&	319054.87	&	279	&	11335.42	&	-99.91	&	-1.46	&	3962.87	&	0.06	\\
\hspace{2ex}	    应付工资	&	108451.43	&		&		&	-100	&	-0.5	&	——	&	- 	\\
\hspace{2ex}	    应付福利费	&		&		&		&	——	&	- 	&	——	&	- 	\\
\hspace{2ex}	    应付股利	&		&		&		&	——	&	- 	&	——	&	- 	\\
\hspace{2ex}	    应交税金	&	2653808.96	&	4537382.1	&	3698641.18	&	70.98	&	8.64	&	-18.49	&	-4.18	\\
\hspace{2ex}	    其他应交款	&		&		&		&	——	&	- 	&	——	&	- 	\\
\hspace{2ex}	    其他应付款	&	708748.14	&		&	90000	&	-100	&	-3.25	&	——	&	0.45	\\
\hspace{2ex}	    预提费用	&		&		&		&	——	&	- 	&	——	&	- 	\\
\hspace{2ex}	    预计负债	&		&		&		&	——	&	- 	&	——	&	- 	\\
\hspace{2ex}	    一年内到期的长期负债	&		&		&		&	——	&	- 	&	——	&	- 	\\
\hspace{2ex}	    其他流动负债	&		&		&		&	——	&	- 	&	——	&	- 	\\
\hspace{2ex}	 流动负债合计	&	10696565.35	&	5775686.01	&	5445518.93	&	-46	&	-22.56	&	-5.72	&	-1.65	\\
\midrule
\hspace{2ex}	长期负债:	&		&		&		&	——	&	- 	&	——	&	- 	\\
\hspace{2ex}	    长期借款	&		&		&		&	——	&	- 	&	——	&	- 	\\
\hspace{2ex}	    应付债券	&		&		&		&	——	&	- 	&	——	&	- 	\\
\hspace{2ex}	    长期应付款	&		&		&		&	——	&	- 	&	——	&	- 	\\
\hspace{2ex}	    专项应付款	&		&		&		&	——	&	- 	&	——	&	- 	\\
\hspace{2ex}	    其他长期负债	&		&		&		&	——	&	- 	&	——	&	- 	\\
\hspace{2ex}	        长期负债合计	&	- 	&	- 	&	- 	&	——	&	- 	&	——	&	- 	\\
\midrule
\hspace{2ex}	递延税项:	&		&		&		&	——	&	- 	&	——	&	- 	\\
\hspace{2ex}	    递延税项贷项	&		&		&		&	——	&	- 	&	——	&	- 	\\
\hspace{2ex}	            负债总计	&	10696565.35	&	5775686.01	&	5445518.93	&	-46	&	-22.56	&	-5.72	&	-1.65	\\
\midrule
\hspace{2ex}	少数股东权益	&		&		&		&	——	&	- 	&	——	&	- 	\\
\hspace{2ex}	所有者权益(或股东权益):	&		&		&		&	——	&	- 	&	——	&	- 	\\
\hspace{2ex}	    实收资本(或股本)	&	5000000	&	22455000	&	32455000	&	349.1	&	80.04	&	44.53	&	49.86	\\
\hspace{2ex}	        减:已归还投资	&		&		&		&	——	&	- 	&	——	&	- 	\\
\hspace{2ex}	    实收资本(或股本)净额	&	5000000	&	22455000	&	32455000	&	349.1	&	80.04	&	44.53	&	49.86	\\
\hspace{2ex}	    资本公积	&		&	10733993.51	&	21149015.68	&	——	&	49.22	&	97.03	&	51.93	\\
\hspace{2ex}	    盈余公积	&	223450.2	&	77467.97	&	745763.63	&	-65.33	&	-0.67	&	862.67	&	3.33	\\
\hspace{2ex}	        其中:法定公益金	&		&		&		&	——	&	- 	&	——	&	- 	\\
\hspace{2ex}	    未分配利润	&	2011051.78	&	697211.69	&	- 	&	-65.33	&	-6.02	&	-100	&	-3.48	\\
\midrule
\hspace{2ex}	所有者权益(或股东权益)合计	&	7234501.98	&	33963673.17	&	54349779.31	&	369.47	&	122.56	&	60.02	&	101.65	\\
\hspace{2ex}	  负债和所有者权益(或股东权益)总计	&	17931067.33	&	39739359.18	&	59795298.24	&	121.62	&	100	&	50.47	&	100	\\
\bottomrule
\end{longtable}\label{asset-debt}

\subsection{资产状况分析}
从\tabref{asset-debt}中,怡惟公司在上一个会计年度内发生的资产变动情况如下:
\begin{compactenum}[(1) ]
 \item 截至2013年底,公司的资产规模为 5985.4万元,连续两年保持快速增长,涨幅分别为121.62\%、50.62\%,这主要是由于公司在近几年不断增加权益资本规模,扩大股东资本比例,使得公司的权益比率达到0.89。在2013年度间,公司的流动比率为8.01,速动比率为3.02,。其中,公司流动资产中占主要部分的是存货资产,从2011年的275.2万元增加到2012年1764.1,增加幅度为540.98\%,并继续增加至2013年2715.2万元,同比增加了53.91\%,由于存货的增加对总资产规模的影响分别为68.27\%与53.91\%。公司存货的增长表现为公司有大量产品的滞销,造成公司在短期内可使用资金缺乏;同时,公司在另一方面又继续扩大企业生产规模,增加投资活动资金。这两方面的资金变动导致怡惟公司在2013年末出现现金流短缺。
 \item 怡惟公司的资产项主要构成为短期流动资产,显示其具有较强的短期偿还与兑现能力。同时,公司没有进行长期资产投资,这不利于公司在未来扩展自身战略领域,为公司在未来的盈利能力提供后续发展潜力。
 \item 从负债与权益资本的结构上分析,怡惟公司的权益资产比率逐年上升。公司主要通过吸收股东投资扩大规模。至2013年末,公司的实收资本规模为3245.5万元,在过去的会计年度间,实收资本增加幅度分别为349.10\%和44.53\%,其对公司总资产规模变动的影响分别为80.04\%和49.86\%。公司在过去三个年度间的权益资本回报率分别为27.28\%、8.83\%、15.44\%,在经历2012年短暂的下降后重新保持较高的增速,显示公司能够为投资者带来丰厚的投资利益回报。从本项指标看,怡惟公司的历史负债水平较低,偿债压力相对较小,具有较为充足的偿还企业债务能力,能够充分保证股东的权益资本利益。
\end{compactenum}















