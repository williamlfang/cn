% !Mode:: "TeX:UTF-8"

\chapter{时间序列分析}

这一章主要介绍如何在 \R 中处理{\kai 时间序列}。金融数据主要是时间序列数据,即在一个时间范围内按照固定时间间隔对所研究的变量进行观察并逐步记录而得到的数据集。这种通过有规律的观测得到的数据往往表现为在时间上的依赖性,即各个时点的数据具有时间依存性。这与传统统计的『独立不相关』的基本假设不同,我们往往在处理时间序列数据时需要将数据的依赖性加以研究,金融变量往往表现为内在变化规律的一致性,这往往成为时间序列分析的研究重点。

\R 已经成为研究和分析金融时间序列数据的首要选择软件,目前已有的 packages 有
\begin{compactenum}
\item \rcode{zoo}
\item \rcode{xts}(extensive time series)
\item \rcode{TRR}
\item \rcode{quantmod}:主要用于从网上下载金融数据,并提供了一套完整的作图方法
\end{compactenum}
%%
\begin{mdfbox}[quantmod]

\end{mdfbox} 

