% !Mode:: "TeX:UTF-8"

\chapter{项目主要风险及防范措施} 
\label{chap05}

\section{政策与行业风险}
房地产行业受国家宏观调控影响较大,国家宏观调控以及房地产市场形势的变化可能会对项目产生不利影响。基金期限内,国家对房地产政策、税收政策和金融政策等相关政策的调整可能会对房地产市场产生不利影响,从而影响基金财产的收益的实现。

\begin{mdfbox}[防范措施]
\hspace{2em}本项目南区是市政府所在地,也是永安新城市发展的热点。项目地处于南区中心区域,依山近水,周边自然环境优美,项目北侧为行政中心、商务中心、鑫科时代广场,西面紧临福建水利电力学院,东面与汽车工业园、永安八中遥相呼应。该项目距离高速公路南出口仅1公里,即将开工建设的高铁动车站仅1.5公里,项目地理位置得天独厚,具有广阔的发展前景。随着永安城市的对外拓展,该区域已逐渐成为永安的政治、文化、商业中心,成为永安企业的总部。

项目手续齐全,符合基金发行条件,不存在合规风险。基金管理人将及时关注政策、法规变换,及时向受益人披露,要求项目相关主题履行职责,并按照基金文件的规定采取相应措施。
\end{mdfbox}


\section{兑付风险及其控制}
兑付风险,也称作『违约风险』,指的是在基金投资期限内,由于鑫科公司面临管理经营不善、项目无法产生预计的现金流、企业资金缺口过大导致公司无法在合约期限内偿付基金投资本息的风险。一方面,该风险与整个房地产行业的周期性波动有关,收到行业不景气因素的影响;另一方面,该风险可能源自公司内部管理欠佳,公司无法在规定期限内兑现支付,从而导致投资者利益受到损坏。

\begin{mdfbox}[防范措施]
\hspace{2em}针对鑫科公司所做的尽职调查表明,公司在过去年度内采取积极的资产管理策略,以增加权益资产的方式扩大企业总资产规模,并努力减少公司对外部债务的依赖,公司在盈利能力创造方面表现突出,在过去两年保持主营业务收入的高速增长。鑫科公司财务状况简单,无不良贷款,资产负债率低于行业平均水平,实际偿付压力较小,流动比率及速动比率严格控制正常范围内。

同时,针对本项目的投资分析认为,该项目由于属于合同签约投资建设方案,具有较为稳定的项目未来现金流入,公司定向为华电提供46\%的楼盘销售,为公司未来的市场销售奠定良好的基础。预计公司在未来不存在较大概率出现现金流大幅波动的情况。
\end{mdfbox}

\section{还款来源分析}
本项目预计总投资额度为 49 870 万元,项目住宅49幢,共有875套。贷款期间,预计可销售65\%,实现现金流入4.2亿元。借款12 000万,完全能覆盖本息安全。

\section{保障措施分析}
本次融资采用本项目土地抵押与公司开发楼盘『永安山庄』共同作为抵押物。该项目土地面积74 586 平方米,各项手续齐全,符合施工条件;『永安山庄』系鑫科公司在建楼盘,根据行业一般数据分析,贷款期间,预计可销售70\%,实现现金流入2.8亿元。














