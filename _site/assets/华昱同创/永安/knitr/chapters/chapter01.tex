% !Mode:: "TeX:UTF-8"

\chapter{公司概况}

\section{公司基本情况}
福建鑫科置业有限公司(以下简称「鑫科公司」或「本公司」)系由福建省永安市工商行政管理局审核批准,由公司三位原始股东林青水、刘宗厚、赖贞旺共同认缴出资组建,于2009年09月27日正式成立,取得企业法人营业执照注册号为 35048 11000 22639。公司当前法定代表人为林青水,注册资本:20,000,000 元,注册经营期限:30年,注册地址:福建省水安市巴溪大道 1369 号 1 幢 2-323 室。

鑫科公司主营业务为房地产开发经营、商品房销售(以上经营范围涉及许可经营项目的,应在取得有关部门的许可后方可经营。),现有投资建设『鑫科·永安山』、『鑫科·时代广场』共计两个项目。公司目前无对外融资负债及对外担保,开发资质为三级。

\subsection{股本结构}
截止2014年03月10日,公司的股本结构如下:
%%------------------------------------------------------------------------
  \begin{center}
  \begin{threeparttable}\vspace{-1.0cm}
  %%
 %\caption{股本构成}
 \renewcommand{\arraystretch}{1.1} \arrayrulewidth=0.8pt \tabcolsep=8pt
 	 \begin{tabular}{cccrr}
 	 &&&& {\small 单位:元}\\
	\hline\hline
\rowcolor{mycyan}	股东名称 	& 股东性质 & 出资方式 &  出资金额      & 持股比例  \\
	\hline \renewcommand{\arraystretch}{1}
	林青水   & 国内自然人 & 货币    &  40,000,000 &  80\% \\
	刘宗厚   & 国内自然人 & 货币    &   5,000,000 &  10\% \\ 
	赖贞旺   & 国内自然人 & 货币    &   5,000,000 &  10\% \\ 
	\midrule
	合计    &           &        &  50,000,000 &  100\% \\
	\bottomrule
	\end{tabular}
\end{threeparttable}
\end{center}
%%------------------------------------------------------------------------
\subsection{组织结构}
公司于2014年01月24日进行股权变更登记,并于同年02月26日更新依据新修改的公司章程而设立的组织结构。目前,公司法定代表人为林青水。新设立的管理层组织架构为
\begin{compactitem}
\item 执行董事兼总经理:林青水
\item 董事、副总经理:刘宗厚
\item 监事:许传福
\end{compactitem}

%%------------------------------------------------------------------------
\subsection{主要管理者介绍}
鑫科公司现有管理层,其具体情况如下。\\
\vspace{1ex}

\begin{mdfbox}[林青水]
林青水,男,1974年12月出生,永安市小陶镇人。1994年7月毕业于福建工程学校工业与民用建筑专业,同年分配到永安市供销社房地产开发公司工作。

具体简历如下:
\begin{compactitem}
\item 1994年9月—2000年8月:永安市供销社房地产开发公司;
\item 2000年8月—2009年9月:福建东泉建筑有限公司永安分公司总经理;
\item 2009年9月—现在:福建鑫科置业有限公司董事、总经理,公司法人代表。
\end{compactitem}

从事房地产开发业绩:
\begin{compactenum}
\item 2001年3—2002年11月:开发石门花园二区11、12号楼,总占地面积2000多平方米,总建筑面积7000多平方米。
\item 2004年5—2005年10月:开发小陶供销社农资综合楼,总占地面积800多平方米,总建筑面积3000多平方米。
\item 2006年6—2008年5月:开发小陶供销社果品仓储综合楼,总占地面积1200多平方米,总建筑面积6002平方米。
\item 2008年3月:开发景祥佳苑楼盘,总占地面积8370平方米,总建筑面积42540平方米。
\item 2009年3月至今开发“永安山庄、时代广场”,总建筑面积12.3万平方米。
\end{compactenum}
\end{mdfbox}
%%------------------------------------------------------------------------
\begin{mdfbox}[刘宗厚]
刘宗厚,男,1972年12月生。永安市小陶人。1994年7月毕业于福建建筑高等专科学校,房地产经营管理专业。同年分配到永安市燕江住宅建设开发公司工作。已取得资称证书:土建高级工程师。

具体简历如下:
\begin{compactitem}
\item 1994年9月~1997年6月:永安市燕江住宅建设开发公司施工管理员;
\item 1997年7月~1999年7月:中外合资永安市安捷房地产开发有限公司董事、工程部经理;
\item 1999年8月~2009年8月:永安市建兴工程监理咨询有限责任公司董事、副总经理、沙县分公司负责人;
\item 2009年9月至今:福建鑫科置业有限公司董事、副总经理。
\end{compactitem}

从事房地产开发业绩:
\begin{compactenum}
\item 1994年7月~1996年8月,开发永安市五四小区商住楼,建筑面积12000平方米。
\item 1996年9月~1999年8月,永安市大小街一期商住楼,建筑面积28500平方米。
\item 2009年3月至今开发“永安山庄、时代广场”,总建筑面积12.3万平方米。
\end{compactenum}
\end{mdfbox}
%%------------------------------------------------------------------------
\begin{mdfbox}[赖贞旺]
赖贞旺,男,1975年3月出生,永安市人。1996年7月毕业于福建建筑高等专科学校土木工程专业,同年分配到永安市市政工程处工作。已取得资称证书:土建助理工程师。

具体简历如下:
\begin{compactitem}
\item 1996年9月—2000年8月:永安市市政工程处;
\item 2000年8月—2009年9月:福建东泉建筑有限公司永安分公司;
\item 2009年9月—现在:福建鑫科置业有限公司。
\end{compactitem}

从事房地产开发业绩:
\begin{compactenum}
\item 001年3—2002年11月:开发石门花园二区11、12号楼,总占地面积2000多平方米,总建筑面积7000多平方米。
\item 004年5—2005年10月:开发小陶供销社农资综合楼,总占地面积800多平方米,总建筑面积3000多平方米。
\item 2006年6—2008年5月:开发小陶供销社果品仓储综合楼,总占地面积1200多平方米,总建筑面积6002平方米。
\item 008年3月:开发景祥佳苑楼盘,总占地面积8370平方米,总建筑面积42540平方米。
\item 2009年3月至今开发“永安山庄、时代广场”,总建筑面积12.3万平方米。
\end{compactenum}
\end{mdfbox}

从鑫科公司主要管理层成员结构上看,公司股东均从事房地产行业多年,有着丰富的行业经验,且大多数出生于上世纪七八十年代,年轻有为,思维敏捷,有极强的开拓创新能力。
\section{公司经营与财务情况}
%%------------------------------------------------------------------------
\subsection{经营情况}

鑫科公司目前主营业务收入来自房地产项目开发与楼盘销售,业已完成或接近完成投资项目有两个。公司首个项目(简称「永安」)位于位于永安城市南部,总占地107.654亩,总建筑面积123 739.2 $m^2$。项目分为两个部份,B地块『鑫科•永安山庄』为独幢、联排别墅,共计241户,建成后可满足中高档住宅市场需求,为入住居民提供一个高雅、舒适、安静的生活环境;C地块『鑫科•时代广场』系集购物、餐饮、娱乐、休闲为一体的商贸城。

截止今年初,时代广场已完成总体的开发及销售。已售面积:26 250.03 $m^2$,实现销售金额:16 500万元。永安山庄组团2、5、6已完成总体的开发及销售。已售面积:30 842.81 $m^2$,实现销售金额:20 000万元,其中未收购房款(购房户按揭贷款和抵押贷款银行未放贷)约 5 000万元。
%%------------------------------------------------------------------------
  \begin{center}
  \renewcommand*{\arraystretch}{0.6}
  \setlength{\tabcolsep}{8pt}
  \begin{threeparttable}\vspace{-1.0cm}
  %%
 \caption{永安项目的基本情况}
 \renewcommand{\arraystretch}{1.1} \arrayrulewidth=0.8pt \tabcolsep=10pt
 	 \begin{tabular}{>{\footnotesize}c>{\footnotesize}r>{\footnotesize}r}
	\hline\hline
\rowcolor{mycyan}	\bfseries 项目 	& \bfseries 『时代广场』 & \bfseries 『永安山庄』\\
	\hline \renewcommand{\arraystretch}{.8}
规划占地	&	13 364.8 $m^2$	&	58 405.1 $m^2$ \\
总建筑面积	&	33 017.6 $m^2$	&	90 721.6 $m^2$ \\
地下室	&	6 288.0 $m^2$	&	23 951.0 $m^2$ \\
人防	&	3 328.5 $m^2$	&	1 964.0 $m^2$ \\
会所	&		&	2 106.0 $m^2$ \\
不计容面积	&	6 288.0 $m^2$	&	26 471.0 $m^2$ \\
\midrule 
容积率	&	2	&	1.0 $\geq$ 1.1 \\
绿化率	&	20.00\%	&	30.00\% \\
建筑高度	&	17.7m	&	9.6m \\
建筑密度	&	45.00\%	& 	36.0\% \\
\midrule
总投资	&	\multicolumn{2}{c}{\hspace{2em}56 156万元}  \\
	\bottomrule
	\end{tabular}
\end{threeparttable}
\end{center}
%%------------------------------------------------------------------------
「永安项目」在2013年11月组团三(71-73\#、75\#)、组团四、独幢及东侧基座开盘,进行现房销售。并预计于2014年8月组团一、组团三(76\#、77\#、79\#、81\#—83\#)、南侧基座开盘,进行预售。预计到2015年底实现全盘销售完毕。预计销售收入如下:
%%------------------------------------------------------------------------
  \begin{center}
  \begin{threeparttable}\vspace{-1.0cm}
  %%
 \caption{永安项目预计现金流}
 \renewcommand{\arraystretch}{1.1} \arrayrulewidth=0.8pt \tabcolsep=8pt
 	 \begin{tabular}{>{\footnotesize}c>{\footnotesize}c>{\footnotesize}r>{\footnotesize}r>{\footnotesize}r}
	\hline\hline
\rowcolor{mycyan}	\bfseries 类型 	& \bfseries 单位   &  \bfseries 数量 \hspace{2ex}      & \bfseries 单价(元\textbackslash $m^2$) & \bfseries 总价(万元)  \\
	\hline \renewcommand{\arraystretch}{1}
住宅	&	$m^2$	&	41,268.70	&	8,500.00	&	35,000.00	\\
商业	&	$m^2$	&	3,035.38	&	10,000.00	&	 3,035.00	\\
车库	&	$m^2$	&	10,016.29	&	2,000.00	&	 2,000.00	\\
\midrule
合计	&		&		&		&	40,035.00	\\	
	\bottomrule
	\end{tabular}
\end{threeparttable}
\end{center}
\begin{note}
根据行业一般数据分析,贷款期间,预计可销售70\%,实现现金流入2.8亿元。经分析,本项目第一还款来源充分,安全边际较高。
\end{note}

%%------------------------------------------------------------------------


%%------------------------------------------------------------------------

\subsection{财务情况}
根据现有企业年度会计报表资料,我们将部分重要财务指标罗列如下,具体财务分析将在下一章节分析。由于房地产行业具有前期投入资本大、投资周期相对长、资本回笼速度慢等特征,鑫科公司在创立之处尚处于楼盘开发阶段,公司主要表现为资金输出为主、主营业务收入不稳定的特点。到2012会计年度期间,公司各项业务步入正轨,相应的资金运转也逐渐趋于平稳运作。因此,我们从鑫科公司公开的部分会计报表可以看出,公司在前面几年的时间资金流动明显,投资回报波动较大;而在最近的两年里,公司逐步趋于具有稳定现金流的房地产市场中,逐渐开始形成稳定的资金运转机制。

在过去的三个会计年度里,鑫科公司的总资产规模实现了较大幅度的增长,从2011年至2012年间,总资产增长了 28.51\%,但是在最近的一个年度里有小幅的降低。从其资本结构上看,资产规模的增长既有来自企业负债规模的扩大,进一步增大财务杠杆的作用,另一方面也来之公司所有者权益资本的增加。而在2013年度间,由于公司通过减少流动负债数量,并进一步增大股东权益资本比例,两方面综合影响导致公司的总资产规模有小幅度的减少,股东对公司资本的控制得到提升,从正向增加了公司抵御由于短期资金紧张造成的经营风险的能力。


%%------------------------------------------------------------------------
%%------------------------------------------------------------------------
\renewcommand*{\arraystretch}{0.8}
\setlength{\tabcolsep}{6pt}
\begin{longtable}{>{\footnotesize}l>{\footnotesize}r>{\footnotesize}r>{\footnotesize}r>{\footnotesize}r>{\footnotesize}r}
\caption[主要财务指标]{主要财务指标}\\  % 1
\hline\hline
\rowcolor{mycyan} {\bfseries \footnotesize  项目} & {\bfseries \footnotesize  2011}\hspace{2ex} & {\bfseries \footnotesize   2012}\hspace{2ex} &   {\bfseries \footnotesize   2013}\hspace{2ex}  &{\bfseries \footnotesize  11-12(\%)} &{\bfseries \footnotesize  12-13(\%)} \\  \endfirsthead          % 2
\caption[]{主要财务指标(续表)} \\ 
\hline\hline
\rowcolor{mycyan} {\bfseries \footnotesize  项目} & {\bfseries \footnotesize  2011}\hspace{2ex} & {\bfseries \footnotesize   2012}\hspace{2ex} &   {\bfseries \footnotesize   2013}\hspace{2ex}  &{\bfseries \footnotesize  11-12(\%)} &{\bfseries \footnotesize  12-13(\%)} \\ \endhead                % 4
\hline
\endfoot
\hline   % 内容开始
总资产		& 276,400,154.67		& 355,201,483.40		& 	340,168,953.93		& 28.51		& -4.23\\
负债总额	& 148,768,887.10		& 	179,359,994.78		& 	42,886,167.63		& 20.56		& -76.09\\
所有者权益	& 127,631,267.57		& 	175,841,488.62		& 	297,282,786.30		& 37.77		& 69.06\\
\midrule
主营收入		& 		52,113,958.50	& 	156,510,205.50	& 	326,399,687.00		& 	200.32		& 	52.05\\
营业利润		& 		-7,486,518.18	& 	55,080,426.86	& 	55,167,127.26		& 835.73		& 	0.16\\
利润总额		& 		-7,486,646.49	& 	54,079,353.76	& 	53,500,235.53	& 	822.34		& -1.08\\
\midrule
销售利润率	& -15.76	& 30.80	& 	12.70	& 		295.44		& 		-58.78 \\
营业利润率	& 2.68	& 	39.00	& 16.51		& 	1352.75		& 		-57.66 \\
总资产利润率	& -2.97	& 	13.57	& 	12.18	& 		556.74		& 		-10.24 \\
税前总资产盈利力	& 0.51	& 	17.18	& 	15.84		& 	3295.02			& 	-7.81 \\
权益资本利润率(ROE)	& -6.44		& 27.42	& 	13.94	& 		526.03		& 		-49.16 \\
税前投入资本利润率(ROIC)	& 0.58	& 	24.88	& 	18.13	& 		4157.83		& 		-27.14 \\
\midrule
流动比率	& 7.46	& 3.22	& 	7.89	& 		-56.90		& 		145.46 \\
速动比率	& 0.17	& 1.45	& 	6.06	& 		771.63		& 		317.51 \\
营运资本需求量比率	& 0.86	& 	0.69	& 	0.87		& 	-20.70	& 			26.81 \\
\midrule
总资产负债率	& 0.54	& 	0.50	& 	0.13	& 		-6.18		& 		-75.03 \\
权益资产比	& 0.46	& 	0.50	& 	0.87	& 		7.21		& 		76.53 \\
权益乘数	& 2.17	& 	2.02	& 	1.14	& 		-6.72	& 			-43.35 \\
权益负债比	& 1.14	& 	2.53	& 	—— & 		121.63		& 	——	  \\
基于EBIT的利息保障倍数	& 0.16	& 	8.77	& 	138.11		& 	5471.18			& 	1474.49 \\
\midrule
总资产周转次数	& 0.19	& 		0.44	& 		0.96	& 			133.70		& 			117.76 \\
固定资产周转次数	& 150.74	& 		91.68	& 		226.07		& 		-39.18			& 		146.59 \\
存货周转天数	& 1888.93	& 		451.97	& 		87.88		& 		-76.07  & -80.56 \\
应收账款周转天数 & 	35.02	& 		272.73 &	64.75	& 			678.80		& 			-76.26 \\

\bottomrule
\end{longtable}
