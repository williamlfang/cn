\begin{abstract}
本文根据已有的\tsm 的理论文献,以\citeath{diebold2006forecasting}拓展的\dns 为基础,提出了一个以人口因素驱动的动\ts 拓展模型。文章将介绍\dns 的发展过程及其后的拓展模型,探讨\ts 动态行为可分解为具有平稳特征的长期均衡状态下的利率均值回复部分与非平稳的时变部分,并指出长期均衡利率水平的平稳组成部分由具有稳定可预测性的\dsf 决定,从而在现有模型的基础上引入\dsf 的分析。在直观上,当~$MY_t$~ 比较大(小),即在特定的时点上一个地区的中年群体(年轻群体)占总人口的比重较大,则这个中年群体由于收入大于支出从而推动了对储蓄(消费)的需求上升。在保证市场出清的条件下,利率则随之下降(上升)。

本文将试图建立一个以人口因素驱动的动\ts 拓展模型。一方面,该模型不仅能够理论上为理解\ts 的长期均衡水平波动提供很好的宏观经济解释,阐述了\dsf 在低频度层次上(low-frequency)影响未来利率的预期水平与变动方向,而且能够提供一个分析宏观经济基础变量与\ts 之间稳定关系的桥梁;另一方面,该模型不仅继承了\dns 的简约 特征(Principle of Parsimony),模型的估计参数较少,实证估计与拟合回归更加灵活,而且\dsf 本身所具有的的稳定的可预测性特征有利于增强对未来\ts 的水平变动的预测能力。

\keywords{利率期限结构;动态~Nelson-Siegel~模型;人口年龄结构;债券市场}
\end{abstract}

