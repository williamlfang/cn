% !Mode:: "TeX:UTF-8"

\chapter{结论}{Conclusion}
\label{chap06}

在直观上,当~$MY_t$~ 比较大(小),即在特定的时点上一个地区的中年群体(年轻群体)占总人口的比重较大,则这个中年群体由于收入大于支出从而推动了对储蓄(消费)的需求上升。在保证市场出清的条件下,利率则随之下降(上升)。GMQ~模型的实证研究也证实了理论研究的结论。即:股票实际回报率与~MY~存在显著的统计关系。\dsf 与不同期限利率的变动之间存在着紧密的关系,这也符合经济学的直觉推理。同时,在一定的时期内,\ds 具有相对可靠的预测性,可被视为一个外生变量,例如美国人口调查局(BoC)每五年都会进行一次人口预测项目(Population Projection)以提供未来的人口结构走势,该项目提供的报告具有一定的准确性。因此,建立一个由人口因素驱动的\tsm 不仅能够有助于增强\tsm 的理论解释能力,为理解影响长期均衡状态利率水平提供良好的理论支持,而在由于\dsf 的可预测性,将进一步提高对未来\tsm 变动的预测能力,从而为政策制定和市场投资提供有效的信息价值。

本文根据已有的\tsm 的理论文献,以\citeath{diebold2006forecasting}拓展的\dns 为基础,提出了一个以人口因素驱动的动态\ts 拓展模型。一方面,该模型不仅能够理论上为理解\ts 的长期均衡水平波动提供很好的宏观经济解释,阐述了\dsf 在低频度层次上(low-frequency)影响未来利率的预期水平与变动方向,而且能够提供一个分析宏观经济基础变量与\ts 之间稳定关系的桥梁;另一方面,该模型不仅继承了\dns 的简约 特征(Principle of Parsimony),模型的估计参数较少,实证估计与拟合回归更加灵活,而且\dsf 本身所具有的的稳定的可预测性特征有利于增强对未来\ts 的水平变动的预测能力。

由于目前本文在\ts 预测方面所做的工作仍然有待改进,故此未给出结果报告。这一点将在今后的工作中进一步的完善。
