% !Mode:: "TeX:UTF-8"

\chapter{经营情况与财务状况分析}{}
\label{chap02}

\section{公司战略分析}{}

下面我们将结合公司会计报表与行业情况,对公司整体战略及竞争策略进行分析。通过对公司所在行业的分析,投资者可以明确公司的行业地位以及所采取的竞争策略,权衡风险与收益,了解和掌握公司的发展潜力,特别是公司在价值创造与盈利能力方面的潜力。

\subsection{经营情况分析}{}

中恒通公司经报武平县工商行政管理局,取得法定一般性经营项目为:工程机械、矿山机械、汽车配件等机械产品生产与销售。公司目前已经于武平县十方工业集中区建立厂房三处,主要生产商用车制动鼓、轮毂、铸钢和冲焊驱动桥壳、工程机械桥壳、平衡悬架总成、刹车盘、刹车片等载重汽车底盘零部件,并设有销售部门提供产品销售与服务。

\subsubsection{盈利能力}{}
从\tabref{lirun}中,公司在2013年度实现净利润37 651千元,同比增长15 439千元,增长率为69.51\%,涨幅较大。从水平分析表上看,该部分净利润增长部分主要来自公司在过去的一个年度里主营业务利润增长了15 746千元,获得补贴收入5 960千元,以及主营业务税金及附加减少475千元,扣除相应成本后,共计增长15 439千元。

\subsubsection{业务发展}{}
以上分析可以认为,中恒通公司在主营业务增长较为强劲,同比上年度增长了24.15\%。正是由于主营业务的大幅度增长,使得税后净利润大幅增加。结合主营业务成本分析,公司在不断开拓产品市场,增加销售业绩,争取更多的市场占有率。汽配市场在国内目前尚处于新兴行业,因此公司目前正处于较为快速发展的阶段,未来面临着良好的发展前景。

同时,提醒投资者注意,\tabref{lirun}显示公司在其他业务利润方面表现欠佳,亏损440千元。这可能是该公司在聚焦主营业务,收缩其他市场。另外,需要注意的是,公司没有任何通过资本投资而获得收益,即由投资活动产生的现金流量净额为负,这表明公司在资本运作与套期保值方面尚未开始启动,在对抗资本市场风险方面有待加强。
%%------------------------------------------------------------------------
\renewcommand*{\arraystretch}{0.8}
\setlength{\tabcolsep}{8pt}
\begin{longtable}{>{\footnotesize}l>{\footnotesize}r>{\footnotesize}r>{\footnotesize}r>{\footnotesize}r}
\caption[利润水平分析表]{利润水平分析表}\\  % 1
&&&& {\scriptsize 单位:元}\\
\hline\hline
\rowcolor{mycyan}	\hspace{3em} \bfseries 项目 	& \bfseries 2013年度\hspace{1em} & \bfseries 2012年度\hspace{1em} &  \bfseries 增减额\hspace{1.5em}      & \hspace{1em} \bfseries 增减($\%$)  \\ \endfirsthead          % 2
\caption[]{利润水平分析表(续表)} \\ 
&&&& {\scriptsize 单位:元}\\                        % 3
\hline\hline
\rowcolor{mycyan}	\hspace{3em} \bfseries 项目 	& \bfseries 2013年度\hspace{1em} & \bfseries 2012年度\hspace{1em} &  \bfseries 增减额\hspace{1.5em}      & \hspace{1em} \bfseries 增减($\%$)  \\  \endhead                % 4
\hline
\endfoot
\hline   % 内容开始
一、主营业务收入 & 402,880,199.53 & 324,498,460.53 & 78,381,739.00 & 24.15\\
       \quad 减:主营业务成本 & 330,211,535.38 & 262,726,315.82 & 67,485,219.56 & 25.69\\
              \qquad\quad 税金及附加 & 1,076,457.79 & 5,926,712.09 & -4,850,254.30 & -81.84\\
二、主营业务利润 & 71,592,206.36 & 55,845,432.62 & 15,746,773.74 & 28.20\\
        \quad 加: 其他业务利润 & -440,309.67 & 35,411.79 & -475,721.46 & -1,343.40\hspace{2ex}\\
        \quad 减:营业费用 & 10,389,610.59 & 9,212,361.74 & 1,177,248.85 & 12.78\\
\qquad\quad 管理费用 & 12,950,707.12 & 8,894,394.13 & 4,056,312.99 & 45.61\\
\qquad\quad 财务费用 & 9,810,807.91 & 8,158,335.98 & 1,652,471.93 & 20.26\\
三、营业利润 & 38,000,771.07 & 29,615,752.56 & 8,385,018.51 & 28.31\\
        \quad 加:投资收益 & - & - & -	 & - \\
\qquad\quad 营业外收入 & 36,359.12 & - & 36,359.12	 & -\\
\qquad\quad 补贴收入 & 5,960,460.00 & - & 5,960,460.00 & - 	\\
       \quad 减:营业外支出 & 124,404.08	& - & 124,404.08 & 	- \\
四、利润总额 & 43,873,186.11 & 29,615,752.56 & 14,257,433.55 & 48.14\\
       \quad 减:所得税 & 6,221,883.46 & 7,403,938.14 & -1,182,054.68 & -15.97\\
五、净利润 & 37,651,302.65 & 22,211,814.42 & 15,439,488.23 & 69.51\\
	\bottomrule
	\end{longtable}\label{lirun}
%%------------------------------------------------------------------------
\subsubsection{现金流}{}
从\tabref{cash}中可以看出,公司在本年度内有大量的现金流发生。在本年度会计期末,公司的现金及现金等价物净增加额为$-$17 2296千元,同比上期减少了31 317,减少幅度为223.36\%。由此可以看出,公司在未来一段时期内将严重缺乏可利用现金。从各项构成来看
\begin{compactenum}[(1) ]
 \item 经营性活动产生的现金流量净额相对同期减少了47 672千元,减少幅度为93.82\%。其中,通过销售产品、提供劳务收到的现金流量增加了128 407千元,增加幅度为42.86\%,表明公司在主营业务现金收入方面取得大幅增长,而收到的具有其他非主营业务的现金流则是大幅减少,同比降低了50 025千元,两项一共导致公司现金流入增长了84 466千元,增加幅度为24.09\%。这表明公司的盈利方面具有不错的表现。同时,公司的现金流出量同比增加了44.06\%,教上一期增加132 139千元,这主要是由于公司在支付购买商品与接受劳务方面有了大幅的增长。
 \item 公司在本年度内由投资活动产生的现金流量净额为$-$5 946千元,同比增加20 506千元,增加幅度为77.52\%。这主要是由于公司在「购建固定资产、无形资产和其期资产所支付的现金」项目上的现金流出减少24 151千元,减少幅度为80.24\%。该部分长期投资有利于企业扩大生产规模,增强自身的长期持续盈利能力。
 \item 本年度内公司通过借款筹集到的现金同比减少3 298千元,减少幅度为20.85\%;同时,偿还债务所支付的现金增加15 200千元,以及用于分配股利、利润或偿付利息所产生的现金减少14 347千元,两项一共导致筹集资金的现金流出增加了852千元,增加幅度为3.26\%。一减一增从而引起筹资活动产生的现金流量净额减少了4 151千元,减少幅度为40.15\%。
\end{compactenum}


\renewcommand*{\arraystretch}{0.6}
\setlength{\tabcolsep}{4pt}
\begin{longtable}{>{\scriptsize}l>{\scriptsize}r>{\scriptsize}r>{\scriptsize}r>{\scriptsize}r}
\caption[现金流量表]{现金流量表}\\  % 1
&&&& {\scriptsize 单位:元}\\
\hline\hline
\rowcolor{mycyan} {\bfseries \scriptsize 项目} & {\bfseries \scriptsize 2013年度}\hspace{2ex} & {\bfseries \scriptsize  2012年度}\hspace{2ex} &  {\bfseries \scriptsize 增减额}\hspace{4ex}      & {\bfseries \scriptsize 增减($\%$)} \\  \endfirsthead          % 2
\caption[]{现金流量表(续表)} \\ 
&&&& {\scriptsize 单位:元}\\                        % 3
\hline\hline
\rowcolor{mycyan} {\bfseries \scriptsize 项目} & {\bfseries \scriptsize 2013年度}\hspace{2ex} & {\bfseries \scriptsize  2012年度}\hspace{2ex} &  {\bfseries \scriptsize 增减额}\hspace{4ex}      & {\bfseries \scriptsize 增减($\%$)}  \\  \endhead                % 4
\hline
\endfoot
\hline   % 内容开始
一、经营活动产生的现金流量 &  &  &  & 	\\		
\hspace{2em}销售产品、提供劳务收到的现金 & 427,999,599.69 & 299,591,963.81 & 128,407,635.88 & 42.86 \\
\hspace{2em}收到的税费返还 & 6,084,802.80  &  & 6,084,802.80	  & \\
\hspace{2em}收到的其他与经营活动有关的现金 & 1,079,210.41 & 51,105,148.94 & -50,025,938.53 & -97.89 \\
\hspace{2em}\bfseries 现金流入小计 & 435,163,612.90 & 350,697,112.75 & 84,466,500.15 & 24.09 \\
\hspace{2em}购买商品、接受劳务支付的现金 & 414,254,319.23 & 266,712,731.17 & 147,541,588.06 & 55.32 \\
\hspace{2em}支付给职工以及为职工支付的现金 & 11,176,445.13 & 9,517,204.10 & 1,659,241.03 & 17.43 \\
\hspace{2em}支付的各项税费 & 6,468,063.83 & 23,654,028.87 & -17,185,965.04	 & -72.66 \\
\hspace{2em}支付的其他与经营活动有关的现金 & 124,403.08 &  & 124,403.08 & 	 \\
\hspace{2em}\bfseries 现金流出小计 & 432,023,231.27 & 299,883,964.14 & 132,139,267.13 & 44.06 \\
\hspace{4em}\bfseries 经营活动产生的现金流量净额 & 3,140,381.63 & 50,813,148.61 & -47,672,766.98 & -93.82 \\
\midrule
二、投资活动产生的现金流量 &  &  &  &  \\
\hspace{2em}收回投资所收到的现金 &  & 3,645,266.79 & -3,645,266.79 & -100.00 \\
\hspace{2em}取得投资收益所收到的现金 &  &  & 	 &  \\
\hspace{2em}处置固定资产、无形资产和长期资产的现金净额 &  &  &  &  \\
\hspace{2em}收到的其他与投资活动有关的现金 &  &  &  & 	 \\
\hspace{2em}\bfseries 现金流入小计 &  & 3,645,266.79 & -3,645,266.79 & -100.00 \\
\hspace{2em}购建固定资产、无形资产和其期资产所支付的现金 & 5,946,276.61 & 30,098,108.87 & -24,151,832.26 & -80.24 \\
\hspace{2em}投资所支付的现金 &  &  &  &  \\
\hspace{2em}支付的其他与投资活动有关的现金 &  &  &  & 	 \\
\hspace{2em}\bfseries 现金流出小计 & 5,946,276.61 & 30,098,108.87 & -24,151,832.26 & -80.24 \\
\hspace{4em}\bfseries 投资活动产生的现金流量净额 & -5,946,276.61 & -26,452,842.08 & 20,506,565.47 &  77.52 \\
\midrule
三、筹资活动产生的现金流量 &  &  &  &  \\
\hspace{2em}吸收投资所收到的现金 &  &  &  &  \\
\hspace{2em}借款所收到的现金 & 12,520,352.33 & 15,819,054.20 & -3,298,701.87 & -20.85 \\
\hspace{2em}收到的其他与筹资活动有关的现金 &  &  &  & 	 \\
\hspace{2em}\bfseries 现金流入小计 & 12,520,352.33 & 15,819,054.20 & -3,298,701.87 & -20.85 \\
\hspace{2em}偿还债务所支付的现金 & 15,200,000.00 &  & 15,200,000.00 & 	 \\
\hspace{2em}分配股利、利润或偿付利息所产生的现金 & 11,810,803.28 & 26,158,335.98 & -14,347,532.70 & -54.85 \\
\hspace{2em}支付的其他与筹资活动有关的现金 &  &  &  &  \\
\hspace{2em}\bfseries 现金流出小计 & 27,010,803.28 & 26,158,335.98 & 852,467.30 & 3.26 \\
\hspace{4em}\bfseries 筹资活动产生的现金流量净额 & -14,490,450.95 & -10,339,281.78 & -4,151,169.17 & -40.15 \\
\midrule
四、汇率变动对现金的影响 &  &  &  & 	 \\
五、现金及现金等价物净增加额 & -17,296,345.93 & 14,021,024.75 & -31,317,370.68 & -223.36 \\
\bottomrule
\end{longtable}\label{cash}

\subsection{行业分析}{}

公司所属机械行业是整个工业的核心,承担着为工业、农业、交通运输业和国防等部门提供技术装备的重任,具有产业关联度高、产品链条长、带动能力强和技术含量高等特点。近年来,我国机械行业发展迅猛,行业产值规模迅速扩大,行业工业总产值增速明显高于同期 GDP 增速;行业工业总产值在 GDP 中的占比明显提高,从 2006 年的 25.30\%上升至 2012 年的35.45\%(《2013中国行业年度报告系列之机械》)。

当前公司的主营业务为汽车配件类机械产品,对上下游行业的发展具有高度的依赖性。汽车配件作为汽车整车行业的上游,是汽车工业发展的基础,其上游行业主要时钢材、石油、有色金属、天然橡胶等材料行业,受宏观经济波动影响较大;其下游行业涵盖了整车装配行业和维修服务业,其自身发展状况取决于下游整车市场和服务维修市场的发展。根据中国汽车工业协会统计,截止2013年,我国汽车产销量双双突破2100万辆,全年累计生产汽车2211.68万辆,同比增长14.76\%;销售汽车2198.41万辆,同比增长13.87\%。于此同时,我国的汽车保有量也有强劲发展,至2013年底,全国机动车数量突破2.5亿辆,其中,汽车达1.37亿辆,扣除报废量,增加1651万辆,增长了13.7\%,占全部机动车比率达到54.9\%。全国有31个城市的汽车数量超过100万辆,其中北京、天津、成都、深圳、上海、广州、苏州、杭州等8个城市汽车数量超过200万辆,北京市汽车超过500万辆。\footnote{部分资料参考「汽车配件网」:http://info.qipei.hc360.com/2014/02/261749601313.shtml}当前国内市场巨大的整车消费市场与汽车保有量为汽车配件行业的发展提供了强劲的动力。目前,我国汽车配件产业已经形成了环渤海地区、长三角地区、珠三角地区、湖北地区、中西部地区五大板块,整个行业呈现快速增长趋势,部分国内汽车配件企业实力大幅提升,出现了一些在细分市场具有全球竞争力的企业。同时,许多国际知名汽配厂商也开始关注中国市场。根据《全球汽车零部件供应商研究报告2013的报告,全球汽车零部件供应业盈利性稳定在高位。数据显示,2012年与2013年的息税前利润利润率都为6.5\%。供应商获利最大的领域为底盘、传动系统与轮胎,而内饰部件领域利润却会进一步萎缩。技术含量高,高附加值带来的高利润,突出地表现在底盘、传动系统及轮胎行业。这意味着汽车后市场规模巨大,利润不断攀升,很多产品如轮胎在后市场上的利润远高于整车市场。在利润的驱使下,国际零部件企业如博世、电装、大陆等也纷纷加快进驻中国售后市场的步伐。

未来随着更多的车主加大对汽车保养与维护的重视,汽车配件行业将面临更多的发展前景。同时,由于国际汽车配件供应商的加入,竞争也将更加激烈,行业未来面临的风险也会随之上升。

\subsection{公司行业地位分析(暂缺)}{}

\subsection{竞争策略分析}{}
中恒通公司集成产业链一体化,定位于「科研、设计、生产、销售、贸易于一体汽车底盘制造商」。在未来的汽车配件行业,将更多的强调以客户需求为导向,突出「私人定制」的个性化服务理念。因此,加大对市场的研究与加强对产品的研发是企业制胜的关键。

选择正确的竞争战略,是企业保持持久竞争优势和高盈利能力的首要问题。目前在汽配行业,主要有低成本策略和产品差异化策略。前者更多的是以低成本取得客户好感,争取一定的市场份额,却往往不注重对现有顾客关系的维护与二次开发,导致前期客户大量流失。与此不同,产品差异化策略把客户的体验置于重心,突出企业产品的独特性以争取在相同的价格甚至是较高价格的基础上获得更大的利润,取得竞争优势与超额利润。

中恒通公司目前已经建立三个生产厂房,配置了先进的生产设备及检验设备,专业生产汽车底盘零部件。在生产技术方面,公司拥有PR-E、AUTOCAD、CAXA等专业设计开发软件和配套科研设施,并采用消失模和V法相结合的铸造工艺,结合数控机床和加工中心等高精度加工工艺。因此,公司产能能够保证市场需求,具有一定的以客户定制为单位进行生产的能力。同时,公司还强化与上下游行业的战略关系,运用自主设计制造的产品匹配沃尔沃,奔驰等国外知名汽车,并与东风汽车、长春一汽、金龙客车等国内外汽车物流制造企业形成长期稳定的合作关系。此外,我们还看到中恒通公司投资建设吉林分公司,项目分两期建设,总投资达十亿元,一期项目总投资五亿元,占地面积83239平方米,总建筑面积57000平方米,用于建设年产6万吨汽车零部件铸锻铸造生产基地和年产10万套汽车桥总成4条装配机加生产线,投产后,年产值达十亿元以上;二期项目计划投资五亿元,建设年产十万吨汽车零部件铸锻铸造生产线,投产后,年产值达十五亿元以上。两期项目全部建成投产后年产值可实现25亿元以上,年上缴税收实现2亿元以上。

综上分析,公司将在未来经营中进一步采取产品差异化的策略,虽然前期可能会流失部分对汽车配件价格较为敏感的客户群体,但从长远角度看,将有利于强化公司的品牌价值,增强自身的持续盈利能力,从而提升公司的投资潜力。

\section{财务状况分析}{}
中恒容公司采用公历年作为会计年度,即1月1日至12月31日为一个完整的会计年度。根据目前也已披露的报表信息,我们对公司的财务状况做如下分析。


%\begin{longtable}{|c|c|c|c|}
%\caption[cap in list]{long table first caption}\\  % 1
%\hline hf & hf & hf & hf \\ \endfirsthead          % 2
%\caption[]{(continued)} \\                         % 3
%\hline hs & hs & hs & hs \\ \endhead                % 4
%\hline       % 内容开始
%content 1 & cont 2 & cont 3 & cont 4   \\
%....................
%\hline 368,852,226.45
%\end{longtable}

\renewcommand*{\arraystretch}{0.6}
\setlength{\tabcolsep}{8pt}
\begin{longtable}{>{\scriptsize}p{8em}>{\scriptsize}r>{\scriptsize}r>{\scriptsize}r>{\scriptsize}r>{\scriptsize}r}
\caption[资产负债表水平分析表]{资产负债表水平分析表}\\  % 1
&&&&& {\scriptsize 单位:元}\\
\hline\hline
\rowcolor{mycyan} {\bfseries \scriptsize 项目} & {\bfseries \scriptsize 2013年度}\hspace{2ex} & {\bfseries \scriptsize  2012年度}\hspace{2ex} &  {\bfseries \scriptsize 增减额}\hspace{4ex}      & {\bfseries \scriptsize 增减($\%$)}  & {\bfseries \scriptsize 影响($\%$)}\\  \endfirsthead          % 2
\caption[]{资产负债表水平分析表(续表)} \\ 
&&&&& {\scriptsize 单位:元}\\                        % 3
\hline\hline
\rowcolor{mycyan} {\bfseries \scriptsize 项目} & {\bfseries \scriptsize 2013年度}\hspace{2ex} & {\bfseries \scriptsize  2012年度}\hspace{2ex} &  {\bfseries \scriptsize 增减额}\hspace{4ex}      & {\bfseries \scriptsize 增减($\%$)}  & {\bfseries \scriptsize 影响($\%$)}\\  \endhead                % 4
\hline
\endfoot
\hline   % 内容开始
流动资产:    & 	 &  &  &  & 	\\	
    \hspace{2ex}货币资金 & 22,151,000.11 & 39,447,347.04 & -17,296,346.93 & -43.85 & -4.69\\
    \hspace{2ex}短期投资 &  &  &  &  & 					\\
    \hspace{2ex}应收票据 & 1,800,000.00  &  & 1,800,000.00 &  & 0.49\\
    \hspace{2ex}应收股利  &  &  &  &  & 					\\
    \hspace{2ex}应收利息 &  &  &  &  & 					\\
  \hspace{2ex}应收账款 & 69,654,898.92 & 45,243,085.98 & 24,411,812.94 & 53.96 & 6.62\\
    \hspace{2ex}其他应收款 & 37,405,898.94 & 6,654,775.10 & 30,751,123.84 & 462.09 & 8.34\\
    \hspace{2ex}预付帐款 & 15,412,763.26 & 14,218,790.60 & 1,193,972.66 & 8.40 & 0.32\\
    \hspace{2ex}应收补贴款 &  &  &  &  & 					\\
    \hspace{2ex}存货 & 33,011,452.64 & 31,254,210.11 & 1,757,242.53 & 5.62 & 0.48\\
    \hspace{2ex}待摊费用 & 0.00 & 35,425.62 & -35,425.62 & -100.00 & -0.01\\
    \hspace{2ex}{\scriptsize 一年内到期长期债券投资} &  &  &  &  & 					\\
    \hspace{2ex}其他流动资产 &   &  &  &  & 								\\	
    \hspace{2ex}\bfseries 流动资产合计 & 179,436,013.87 & 136,853,634.45 & 42,582,379.42 & 31.12 & 11.54\\
长期投资: &  &  &  &  & 					\\
    \hspace{2ex}长期股权投资 & 62,354,733.21 & 62,354,733.21 &  &  & \\
    \hspace{2ex}长期债权投资 &  &  &  &  & 				\\
    \hspace{2ex}\bfseries 长期投资合计 & 62,354,733.21 & 62,354,733.21 &  &  & \\
固定资产: &  &  &  &  & 					\\
    \hspace{2ex}固定资产原价 & 162,248,865.28 & 136,437,284.86 & 25,811,580.42 & 18.92 & 7.00\\
        \hspace{3ex}减:累计折旧 & 41,726,115.28 & 28,092,340.84 & 13,633,774.44 & 48.53 & 3.70\\
    \hspace{2ex}固定资产净值 & 120,522,750.00 & 108,344,944.02 & 12,177,805.98 & 11.24 & 3.30\\
        \hspace{3ex}减:固定资产减值准备 &  &  &  &  & 			\\
    \hspace{2ex}固定资产净额 & 120,522,750.00 & 108,344,944.02 & 12,177,805.98 & 11.24 & 3.30\\
    \hspace{2ex}工程物资 &  &  &  &  & 				\\
    \hspace{2ex}在建工程 & 16,427,288.42 & 35,577,262.79 & -19,149,974.37 & -53.83 & -5.19\\
    \hspace{2ex}固定资产清理 &  &  &  &  & 			\\
    \hspace{2ex}\bfseries 固定资产合计 & 136,950,038.42 & 143,922,206.81 & -6,972,168.39 & -4.84 & -1.89\\
无形资产及其他资产: &  &  &  &  & 					\\
    \hspace{2ex}无形资产 & 24,677,461.73 & 24,987,020.01 & -309,558.28 & -1.24 & -0.08\\
    \hspace{2ex}长期待摊费用 & 328,860.83 & 734,631.97 & -405,771.14 & -55.23 & -0.11\\
    \hspace{2ex}其他长期资产 &  &  &  &  & 					\\
    \hspace{2ex}\bfseries 无形资产及其他资产合计 & 25,006,322.56 & 25,721,651.98 & -715,329.42 & -2.78 & -0.19\\
递延税项: &  &  &  &  & 					\\
    \hspace{2ex}递延税款借项 &  &  &  &  & 			\\
    \hspace{2ex}\bfseries 资产总计 & 403,747,108.06 & 368,852,226.45 & 34,894,881.61 & 9.46 & 9.46\\
    \midrule
流动负债: &  &  &  &  & \\					
    \hspace{2ex}短期借款 & 84,550,000.00 & 78,600,000.00 & 5,950,000.00 & 7.57 & 1.61\\
    \hspace{2ex}应付票据 & 19,500,000.00 & 57,000,000.00 & -37,500,000.00 & -65.79 & -10.17\\
    \hspace{2ex}应付帐款 & 22,250,369.01 & 9,407,245.43 & 12,843,123.58 & 136.52 & 3.48\\
    \hspace{2ex}预收帐款 &  &  &  &  & 	\\
    \hspace{2ex}应付工资 & 1,088,406.93 & 785,242.22 & 303,164.71 & 38.61 & 0.08	\\
    \hspace{2ex}应付股利 &  &  &  &  & 	\\
    \hspace{2ex}应交税金 & 5,995,173.49 & 226,409.71 & 5,768,763.78 & 2,547.93 & 1.56\\
   \hspace{2ex} 其他应交款 & 304,816.66 & 2,735.65 & 302,081.01 & 11,042.39 & 0.08\\
    \hspace{2ex}其他应付款 & 2,745,416.36 & 2,745,416.36	 & 0.74\\
    \hspace{2ex}预提费用 & 1,109,807.62 & 1,109,807.62 & 0.30\\
    \hspace{2ex}预计负债 &  &  &  &  & 			\\
    \hspace{2ex}一年内到期的长期负债 &  &  &  &  & 				\\
    \hspace{2ex}其他流动负债 &  &  &  &  & 			\\
 \hspace{2ex}\bfseries 流动负债合计 & 137,543,990.07 & 146,021,633.01 & -8,477,642.94 & -5.81 & -2.30\\
长期负债: &  &  &  &  & 	\\
    \hspace{2ex}长期借款 & 0.00 & 13,057,738.05 & -13,057,738.05 & -100.00 & -3.54\\
    \hspace{2ex}应付债券 &  &  &  &  & 	\\
    \hspace{2ex}长期应付款 & 19,628,090.38  & & 19,628,090.38 & - & 5.32\\
    \hspace{2ex}专项应付款 &  &  &  &  & 		\\
    \hspace{2ex}其他长期负债 &  &  &  &  & 	\\
    \hspace{2ex}\bfseries 长期负债合计 & 19,628,090.38 & 13,057,738.05 & 6,570,352.33 & 50.32 & 1.78\\
递延税项: &  &  &  &  & 	\\
    \hspace{2ex}递延税项贷项 &  &  &  &  & \\	
    \hspace{2ex}\bfseries 负债总计 & 157,172,080.45 & 159,079,371.06 & -1,907,290.61 & -1.20 & -0.52\\
少数股东权益 &  &  &  &  & 	\\
所有者权益(或股东权益): &  &  &  &  & 	\\
    \hspace{2ex}实收资本(或股本) & 63,000,000.00 & 63,000,000.00  & 0.00 & 0.00 & 0.00 \\
       \hspace{3ex} 减:已归还投资 &  &  &  &  & 	\\
    \hspace{2ex}实收资本(或股本)净额 & 63,000,000.00 & 63,000,00  & 0.00 & 0.00 & 0.00 \\
    \hspace{2ex}资本公积 & 80,643,863.00 & 80,643,863.00 & 0.00 & 0.00 & 0.00\\
    \hspace{2ex}盈余公积 & 13,245,491.66 & 9,480,361.40 & 3,765,130.26 & 39.72 & 1.02\\
        \hspace{2ex}其中:法定公益金 &  &  &  &  & \\
    \hspace{2ex}未分配利润 & 89,685,672.95 & 56,648,630.99 & 33,037,041.96 & 58.32 & 8.96\\		
\bfseries 所有者权益合计 & 246,575,027.61 & 209,772,855.39 & 36,802,172.22 & 17.54 & 9.98\\
  \hspace{2ex}\bfseries \bfseries 负债和所有者权益总计 & 403,747,108.06 & 368,852,226.45 & 34,894,881.61 & 9.46 & 9.46\\
\bottomrule
\end{longtable}\label{asset-debt}

\subsection{资产状况分析}{}
从\tabref{asset-debt}中,中恒通公司在上一个会计年度内发生的资产变动情况如下:
\begin{compactenum}[(1) ]
 \item 本期公司总资产规模增加34 894千元,增加幅度为9.46\%,主要来自应收账款与其他应收款项、固定资产投资的增加,应收账款回收情况将对公司经营产生一定影响。提醒投资者关注公司过去事项的应收账款回收率,以及整体行业的违约风险。
 \item 公司流动资产增加42 582千元,同比增加了 31.12\%,对总资产的影响为11.54\%。由于流动资产具有较强的资金流动性,因此公司流动资产的增加能够保证公司的偿还能力,满足对资金流动性的需要。特别的,公司的货币资金减少17 296千元,降幅为43.85\%,表明中恒容公司在近期有大量的货币资金流出,需要在短期内获得一定比例的货币资金融资注入;应收账款增加了21 411千元,增幅达53.96\%,对总资产的影响为6.62\%;其他应收款项增加30 751千元,增长幅度为462.09\%,对总资产的影响为8.34\%。
 \item 在建工程减少了19 149千元,减少幅度为53.83,对总资产的影响为5.19\%。结合公司主营业务增加而其他营业收入减少,我们推测中恒容公司有意减少其他非核心竞争力的汽车配件产品生产。结合公司存货增长幅度为5.62\%,对总资产的影响为0.48\%,其中在产品减少6 331千元,减少幅度为80.17\%,且其积压库存产品上升5 557千元,涨幅为63.74\%,这可能会影响公司在后续阶段的持续盈利能力,降低资金的周转与利用率。
\end{compactenum}

\subsubsection{货币资金}
\renewcommand*{\arraystretch}{0.8}
\setlength{\tabcolsep}{8pt}
\begin{longtable}{>{\footnotesize}c>{\footnotesize}r>{\footnotesize}r>{\footnotesize}r>{\footnotesize}r}
%\caption[cap in list]{货币资金}\\  % 1
 \multicolumn{4}{c}{\footnotesize \bfseries 货币资金} & {\scriptsize 单位:元}\\
\hline\hline
\rowcolor{mycyan} {\bfseries \footnotesize  项目} & {\bfseries \footnotesize  2013年度}\hspace{2ex} & {\bfseries \footnotesize   2012年度}\hspace{2ex} &  {\bfseries \footnotesize  增减额}\hspace{4ex}      & {\bfseries \footnotesize  增减($\%$)} \\  \endfirsthead          % 2
%\caption{续表} \\  
 \multicolumn{4}{c}{\footnotesize \bfseries 货币资金(续表)} & {\scriptsize 单位:元}\\  % 3
\hline\hline
\rowcolor{mycyan} {\bfseries \footnotesize  项目} & {\bfseries \footnotesize  2013年度}\hspace{2ex} & {\bfseries \footnotesize   2012年度}\hspace{2ex} &  {\bfseries \footnotesize  增减额}\hspace{4ex}      & {\bfseries \footnotesize  增减($\%$)}  \\  \endhead                % 4
\hline
\endfoot
\hline   % 内容开始
现金 & 273,114.88 & 15,219.68 & 257,895.20 & 16.94\\
银行存款 & 21,877,885.23 & 39,432,127.36 & -17,554,242.13 & -44.52\\
\midrule
\bfseries 合计 & 22,151,000.11 & 39,447,347.04 & -17,296,346.93 & -43.85\\
\bottomrule
\end{longtable}

\subsubsection{应收票据}
\renewcommand*{\arraystretch}{0.8}
\setlength{\tabcolsep}{8pt}
\begin{longtable}{>{\footnotesize}c>{\footnotesize}r>{\footnotesize}r>{\footnotesize}r>{\footnotesize}r}
%\caption[cap in list]{应收票据}\\  % 1
 \multicolumn{4}{c}{\footnotesize \bfseries 应收票据} & {\scriptsize 单位:元}\\
\hline\hline
\rowcolor{mycyan} {\bfseries \footnotesize  项目} & {\bfseries \footnotesize  2013年度}\hspace{2ex} & {\bfseries \footnotesize   2012年度}\hspace{2ex} &  {\bfseries \footnotesize  增减额}\hspace{4ex}      & {\bfseries \footnotesize  增减($\%$)} \\  \endfirsthead          % 2
 \multicolumn{4}{c}{\footnotesize \bfseries 应收票据(续表)} & {\scriptsize 单位:元}\\   % 3
\hline\hline
\rowcolor{mycyan} {\bfseries \footnotesize  项目} & {\bfseries \footnotesize  2013年度}\hspace{2ex} & {\bfseries \footnotesize   2012年度}\hspace{2ex} &  {\bfseries \footnotesize  增减额}\hspace{4ex}      & {\bfseries \footnotesize  增减($\%$)}  \\  \endhead                % 4
\hline
\endfoot
\hline   % 内容开始
应收票据 & 1,800,000.00 & 0.00 & 1,800,000.00 & -\\
\midrule
\bfseries 合计 & 1,800,000.00 & 0.00 & 1,800,000.00 & -\\
\bottomrule
\end{longtable}

\subsubsection{应收账款}
截止2014年1月31日,公司预付款项的应收账款实际情况如下:
\renewcommand*{\arraystretch}{0.6}
\setlength{\tabcolsep}{4pt}
\begin{longtable}{>{\scriptsize}l>{\scriptsize}r>{\scriptsize}r>{\scriptsize}r>{\scriptsize}r}
%\caption[cap in list]{应收账款}\\  % 1
 \multicolumn{4}{c}{\footnotesize \bfseries 应收账款} & {\scriptsize 单位:元}\\
\hline\hline
\rowcolor{mycyan} {\bfseries \footnotesize  客户名称} & {\bfseries \footnotesize  与公司关系}\hspace{2ex} & {\bfseries \footnotesize   金额}\hspace{2ex} &  {\bfseries \footnotesize  时间}\hspace{4ex}      & {\bfseries \footnotesize  未结算原因} \\  \endfirsthead          % 2
 \multicolumn{4}{c}{\footnotesize \bfseries 应收账款(续表)} & {\scriptsize 单位:元}\\                     % 3
\hline\hline
\rowcolor{mycyan} {\bfseries \footnotesize  客户名称} & {\bfseries \footnotesize  与公司关系}\hspace{2ex} & {\bfseries \footnotesize   金额}\hspace{2ex} &  {\bfseries \footnotesize  时间}\hspace{4ex}      & {\bfseries \footnotesize  原因} \\  \endhead                % 4
\hline
\endfoot
\hline   % 内容开始
四川金腾汽车配件有限公司 &  & 
4,370,460.00  &  & \\
四川赢信投资管理有限公司 &  &
5,707,697.80 &  & \\
济南昊升商贸有限公司 &  &
4,467,376.00 &  & \\
福建亿晟贸易有限公司 &  &
3,873,506.44 &  & \\
广东梅县昌盛汽车配件有限公司大新城分公司 &  &
4,624,478.32 &  & \\
厦门嘉麒世纪贸易有限公司 &  &
1,808,639.20 &  & \\
贵州鑫康达物资贸易有限公司 &  &
2,828,797.58 &  & \\
东风德纳车桥有限公司厦门分公司 &  &
3,185,420.06 &  & \\
佛山市富合汽车工业配件有限公司 &  &
2,154,358.80 &  & \\
十堰海奥汽车实业有限公司 &  &
1,470,877.94 &  & \\
浙江杭州丰润汽配 &  &
2,530,790.16 &  & \\
佛山市永力泰车轴有限公司 &  &
2,070,640.99 &  & \\
重庆赛东汽配有限公司 &  &
3,519,458.00 &  & \\
福州新华士特汽车零部件有限公司 &  &
2,022,901.90 &  & \\
湘龙锦翔汽配 &  &
2,862,183.71 &  & \\
广州市特耐得车轴有限公司 &  &
1,118,104.32 &  & \\
广州华劲机械制造有限公司 &  &
1,452,875.77 &  & \\
东风德纳车桥有限公司十堰工厂 &  &
1,864,392.00 &  & \\
南宁市正庄望州商贸有限公司 &  &
758,486.39 &  & \\
漳州市力胜工贸有限公司 &  &
1,536,405.52 &  & \\
十堰德致中和有限公司 &  &
4,583,383.61 &  & \\
汕头新和盛 &  &
2,889,730.00 &  & \\
其他客户 &  &
4,737,459.5  &  & \\
\midrule
\bfseries 合计 & - & 66,438,424.01 & - & -\\
\bottomrule
\end{longtable}

\subsubsection{其他应收款}
中恒容公司保留的其他应收款主要为应收客户铺底金或押金。具体如下:
\renewcommand*{\arraystretch}{0.6}
\setlength{\tabcolsep}{4pt}
\begin{longtable}{>{\scriptsize}l>{\scriptsize}r>{\scriptsize}r>{\scriptsize}r>{\scriptsize}r}
%\caption[cap in list]{其他应收款}\\  % 1
 \multicolumn{4}{c}{\footnotesize \bfseries 其他应收款} & {\scriptsize 单位:元}\\
\hline\hline
\rowcolor{mycyan} {\bfseries \footnotesize  客户名称} & {\bfseries \footnotesize  与公司关系}\hspace{2ex} & {\bfseries \footnotesize   金额}\hspace{2ex} &  {\bfseries \footnotesize  时间}\hspace{4ex}      & {\bfseries \footnotesize  未结算原因} \\  \endfirsthead          % 2
 \multicolumn{4}{c}{\footnotesize \bfseries 其他应收款(续表)} & {\scriptsize 单位:元}\\             % 3
\hline\hline
\rowcolor{mycyan} {\bfseries \footnotesize  客户名称} & {\bfseries \footnotesize  与公司关系}\hspace{2ex} & {\bfseries \footnotesize   金额}\hspace{2ex} &  {\bfseries \footnotesize  时间}\hspace{4ex}      & {\bfseries \footnotesize  原因} \\  \endhead                % 4
\hline
\endfoot
\hline   % 内容开始
福建亿晟贸易有限公司 &  &
200,000.00  &  & \\
浙江总经销(林嘉庆) &  &
200,000.00  &  & \\
四川赢信投资管理公司 &  &
200,000.00  &  & \\
漳州力胜工贸有限公司 &  &
354,261.00  &  & \\
龙岩锦翔工贸有限公司 &  &
603,758.25  &  & \\
龙岩福盛工贸有限公司 &  &
401,479.35  &  & \\
佛山市富合汽车工业配件有限公司 &  &
100,000.00  &  & \\
广州华劲机械制造有限公司 &  &
100,000.00  &  & \\
广州特耐得车轴有限公司 &  &
100,000.00  &  & \\
佛山永力泰车轴有限公司 &  &
100,000.00  &  & \\
浙江双臣车桥有限公司 &  &
100,000.00  &  & \\
福建永驰车业有限公司 &  &
100,000.00  &  & \\
东风德纳十堰工厂 &  &
100,000.00  &  & \\
浙江杭州丰润汽配 &  &
200,000.00  &  & \\
福州新易达有限公司 &  &
200,000.00  &  & \\
湘龙锦翔 &  &
200,000.00  &  & \\
山东济南林寿山 &  &
200,000.00  &  & \\
十堰海奥汽车实业有限公司 &  &
200,000.00  &  & \\
贵州鑫康达物资贸易有限公司 &  &
200,000.00  &  & \\
福建东风汽车配件贸易有限公司 &  &
200,000.00  &  & \\
郑州天贝(王记山) &  &
200,000.00  &  & \\
佛山炜德汽车配件有限公司 &  &
100,000.00  &  & \\
长春邱绍臻 &  &
100,000.00  &  & \\
其他 &  &
30,440,200.34 &  & \\
\midrule
\bfseries 合计 & - & 34,899,698.94 & - & -\\
\bottomrule
\end{longtable}

\subsubsection{预付账款}
截止2014年1月31日,公司预付款项的预付账款实际情况如下:
\renewcommand*{\arraystretch}{0.6}
\setlength{\tabcolsep}{4pt}
\begin{longtable}{>{\scriptsize}l>{\scriptsize}r>{\scriptsize}r>{\scriptsize}r>{\scriptsize}r}
%\caption[cap in list]{预付账款}\\  % 1
 \multicolumn{4}{c}{\footnotesize \bfseries 预付账款} & {\scriptsize 单位:元}\\
\hline\hline
\rowcolor{mycyan} {\bfseries \footnotesize  客户名称} & {\bfseries \footnotesize  与公司关系}\hspace{2ex} & {\bfseries \footnotesize   金额}\hspace{2ex} &  {\bfseries \footnotesize  时间}\hspace{4ex}      & {\bfseries \footnotesize  未结算原因} \\  \endfirsthead          % 2
%\caption{续表} \\  
 \multicolumn{4}{c}{\footnotesize \bfseries 预付账款(续表)} & {\scriptsize 单位:元}\\           % 3
\hline\hline
\rowcolor{mycyan} {\bfseries \footnotesize  客户名称} & {\bfseries \footnotesize  与公司关系}\hspace{2ex} & {\bfseries \footnotesize   金额}\hspace{2ex} &  {\bfseries \footnotesize  时间}\hspace{4ex}      & {\bfseries \footnotesize  原因} \\  \endhead                % 4
\hline
\endfoot
\hline   % 内容开始
沈阳巨浪特种机床科技有限公司 &  &
2,053,000.00  &  & \\
杭州为公精密机械有限公司 &  &
1,496,875.58  &  & \\
常州市新墅机床数控设备有限公司 &  &
1,354,798.20  &  & \\
宝鸡忠诚机床股份有限公司 &  &
1,107,425.28  &  & \\
龙岩祥瑞机床设备 &  &
1,728,433.48  &  & \\
北京嘉华天祥科技有限公司 &  &
801,479.35  &  & \\
沈阳峰立制动器有限公司 &  &
4,053,000.00  &  & \\
青岛青铸机械有限公司 &  &
603,758.25  &  & \\
其他 &  &
2,316,993.12 &  & \\
\midrule
\bfseries 合计 &   & 66,438,424.01 & - & -\\
\bottomrule
\end{longtable}


\subsubsection{存货}
\renewcommand*{\arraystretch}{0.8}
\setlength{\tabcolsep}{8pt}
\begin{longtable}{>{\footnotesize}c>{\footnotesize}r>{\footnotesize}r>{\footnotesize}r>{\footnotesize}r}
%\caption[cap in list]{存货}\\  % 1
 \multicolumn{4}{c}{\footnotesize \bfseries 存货} & {\scriptsize 单位:元}\\
\hline\hline
\rowcolor{mycyan} {\bfseries \footnotesize  项目} & {\bfseries \footnotesize  2013年度}\hspace{2ex} & {\bfseries \footnotesize   2012年度}\hspace{2ex} &  {\bfseries \footnotesize  增减额}\hspace{4ex}      & {\bfseries \footnotesize  增减($\%$)} \\  \endfirsthead          % 2
%\caption{续表} \\  
 \multicolumn{4}{c}{\footnotesize \bfseries 存货(续表)} & {\scriptsize 单位:元}\\               % 3
\hline\hline
\rowcolor{mycyan} {\bfseries \footnotesize  项目} & {\bfseries \footnotesize  2013年度}\hspace{2ex} & {\bfseries \footnotesize   2012年度}\hspace{2ex} &  {\bfseries \footnotesize  增减额}\hspace{4ex}      & {\bfseries \footnotesize  增减($\%$)}  \\  \endhead                % 4
\hline
\endfoot
\hline   % 内容开始
原材料 & 12,224,158.64 & 5,521,531.50 & 6,702,627.14 & 121.39\\
包装物 & 4,945,521.51 & 5,152,510.31 & -206,988.80 & -4.02\\
在产品 & 1,565,912.94 & 7,897,893.33 & -6,331,980.39 & -80.17\\
外购商品 & 0.00 & 3,963,785.22 & -3,963,785.22 & -100\\
库存商品 & 14,275,859.55 & 8,718,489.75 & 5,557,369.80 & 63.74\\
\midrule
\bfseries 合计 & 33,011,452.64 & 31,254,210.11 & 1,757,242.53 & 5.62\\
\bottomrule
\end{longtable}

\subsubsection{长期投资}
\renewcommand*{\arraystretch}{0.8}
\setlength{\tabcolsep}{4pt}
\begin{longtable}{>{\footnotesize}c>{\footnotesize}r>{\footnotesize}r>{\footnotesize}c>{\footnotesize}c}
%\caption[cap in list]{长期投资}\\  % 1
 \multicolumn{4}{c}{\footnotesize \bfseries 长期投资} & {\scriptsize 单位:元}\\
\hline\hline
\rowcolor{mycyan} {\bfseries \footnotesize  项目} & {\bfseries \footnotesize  2013年度}\hspace{2ex} & {\bfseries \footnotesize   2012年度}\hspace{2ex} &  {\bfseries \footnotesize  增减额}\hspace{4ex}      & {\bfseries \footnotesize  增减($\%$)} \\  \endfirsthead          % 2
%\caption{续表} \\  
 \multicolumn{4}{c}{\footnotesize \bfseries 长期投资(续表)} & {\scriptsize 单位:元}\\                     % 3
\hline\hline
\rowcolor{mycyan} {\bfseries \footnotesize  项目} & {\bfseries \footnotesize  2013年度}\hspace{2ex} & {\bfseries \footnotesize   2012年度}\hspace{2ex} &  {\bfseries \footnotesize  增减额}      & {\bfseries \footnotesize  增减($\%$)}  \\  \endhead                % 4
\hline
\endfoot
\hline   % 内容开始
中恒容(吉林)机械制造有限公司 & 62,354,733.21 & 62,354,733.21 & 0 & 0\\
\midrule
\bfseries 合计  & 62,354,733.21 & 62,354,733.21 & 0 & 0\\
\bottomrule
\end{longtable}

\subsubsection{固定投资及累计折旧}
\renewcommand*{\arraystretch}{0.8}
\setlength{\tabcolsep}{6pt}
\begin{longtable}{>{\footnotesize}l>{\footnotesize}r>{\footnotesize}r>{\footnotesize}c>{\footnotesize}c}
%\caption[cap in list]{固定投资及累计折旧}\\  % 1
 \multicolumn{4}{c}{\footnotesize \bfseries 固定投资及累计折旧} & {\scriptsize 单位:元}\\
\hline\hline
\rowcolor{mycyan} \hspace{4em}{\bfseries \footnotesize  项目} & {\bfseries \footnotesize  2013年度}\hspace{2ex} & {\bfseries \footnotesize   2012年度}\hspace{2ex} &  {\bfseries \footnotesize  增减额}\hspace{4ex}      & {\bfseries \footnotesize  增减($\%$)} \\  \endfirsthead          % 2
%\caption{续表} \\  
 \multicolumn{4}{c}{\footnotesize \bfseries 固定投资及累计折旧(续表)} & {\scriptsize 单位:元}\\                         % 3
\hline\hline
\rowcolor{mycyan} {\bfseries \footnotesize  项目} & {\bfseries \footnotesize  2013年度}\hspace{2ex} & {\bfseries \footnotesize   2012年度}\hspace{2ex} &  {\bfseries \footnotesize  增减额}      & {\bfseries \footnotesize  增减($\%$)}  \\  \endhead                % 4
\hline
\endfoot
\hline   % 内容开始
一、固定资产原值 & 162,248,865.28 & 136,437,284.86 & 25,811,580.42 & 18.92\\
\hspace{2em}其中:房屋及建筑物 & 31,729,179.32 & 28,400,411.61 & 3,328,767.71 & 11.72\\
\hspace{3em}机械设备 & 118,947,373.08 & 108,036,873.25 & 10,910,499.83 & 10.10\\
\hspace{3em}其他设备 & 11,572,312.88 & 0.00 & 11,572,312.88	& -\\
\hspace{3em}减:累计折旧 & 41,726,115.28 & 28,092,340.84 & 13,633,774.44 & 48.53\\
二、固定资产净值 & 120,522,750.00 & 108,344,944.02 & 12,177,805.98 & 11.24\\
\bottomrule
\end{longtable}
						
\subsubsection{在建工程}
在建工程期末余额为16 427 288.42元,系在建厂房、综合楼、道路等,未办理验收结算手续。
\renewcommand*{\arraystretch}{0.8}
\setlength{\tabcolsep}{8pt}
\begin{longtable}{>{\footnotesize}c>{\footnotesize}r>{\footnotesize}r>{\footnotesize}c>{\footnotesize}c}
%\caption[cap in list]{在建工程}\\  % 1
 \multicolumn{4}{c}{\footnotesize \bfseries 在建工程} & {\scriptsize 单位:元}\\
\hline\hline
\rowcolor{mycyan} {\bfseries \footnotesize  项目} & {\bfseries \footnotesize  2013年度}\hspace{2ex} & {\bfseries \footnotesize   2012年度}\hspace{2ex} &  {\bfseries \footnotesize  增减额}\hspace{4ex}      & {\bfseries \footnotesize  增减($\%$)} \\  \endfirsthead          % 2
%\caption{续表} \\  
 \multicolumn{4}{c}{\footnotesize \bfseries 在建工程(续表)} & {\scriptsize 单位:元}\\                        % 3
\hline\hline
\rowcolor{mycyan} {\bfseries \footnotesize  项目} & {\bfseries \footnotesize  2013年度}\hspace{2ex} & {\bfseries \footnotesize   2012年度}\hspace{2ex} &  {\bfseries \footnotesize  增减额}      & {\bfseries \footnotesize  增减($\%$)}  \\  \endhead                % 4
\hline
\endfoot
\hline   % 内容开始
在建工程 & 16,427,288.42 & 35,577,262.79 & -19,149,974.37 & -53.83 \\
\midrule
\bfseries 合计  & 16,427,288.42 & 35,577,262.79 & -19,149,974.37 & -53.83 \\
\bottomrule
\end{longtable}

\subsubsection{无形资产}
\renewcommand*{\arraystretch}{0.8}
\setlength{\tabcolsep}{8pt}
\begin{longtable}{>{\footnotesize}c>{\footnotesize}r>{\footnotesize}r>{\footnotesize}c>{\footnotesize}c}
%\caption[cap in list]{无形资产}\\  % 1
 \multicolumn{4}{c}{\footnotesize \bfseries 无形资产} & {\scriptsize 单位:元}\\
\hline\hline
\rowcolor{mycyan} {\bfseries \footnotesize  项目} & {\bfseries \footnotesize  2013年度}\hspace{2ex} & {\bfseries \footnotesize   2012年度}\hspace{2ex} &  {\bfseries \footnotesize  增减额}\hspace{4ex}      & {\bfseries \footnotesize  增减($\%$)} \\  \endfirsthead          % 2
%\caption{续表} \\  
 \multicolumn{4}{c}{\footnotesize \bfseries 无形资产(续表)} & {\scriptsize 单位:元}\\                              % 3
\hline\hline
\rowcolor{mycyan} {\bfseries \footnotesize  项目} & {\bfseries \footnotesize  2013年度}\hspace{2ex} & {\bfseries \footnotesize   2012年度}\hspace{2ex} &  {\bfseries \footnotesize  增减额}      & {\bfseries \footnotesize  增减($\%$)}  \\  \endhead                % 4
\hline
\endfoot
\hline   % 内容开始
    无形资产  & 24,677,461.73  & 24,987,020.01  & -309,558.28  & -1.24\\
\midrule
\bfseries 合计  & 24,677,461.73  & 24,987,020.01  & -309,558.28  & -1.24\\
\bottomrule
\end{longtable}

\subsubsection{长期待摊销费用}
\renewcommand*{\arraystretch}{0.8}
\setlength{\tabcolsep}{8pt}
\begin{longtable}{>{\footnotesize}c>{\footnotesize}r>{\footnotesize}r>{\footnotesize}c>{\footnotesize}c}
%\caption[cap in list]{长期待摊销费用}\\  % 1
 \multicolumn{4}{c}{\footnotesize \bfseries 长期待摊销费用} & {\scriptsize 单位:元}\\
\hline\hline
\rowcolor{mycyan} {\bfseries \footnotesize  项目} & {\bfseries \footnotesize  2013年度}\hspace{2ex} & {\bfseries \footnotesize   2012年度}\hspace{2ex} &  {\bfseries \footnotesize  增减额}\hspace{4ex}      & {\bfseries \footnotesize  增减($\%$)} \\  \endfirsthead          % 2
%\caption{续表} \\  
 \multicolumn{4}{c}{\footnotesize \bfseries 长期待摊销费用(续表)} & {\scriptsize 单位:元}\\                    % 3
\hline\hline
\rowcolor{mycyan} {\bfseries \footnotesize  项目} & {\bfseries \footnotesize  2013年度}\hspace{2ex} & {\bfseries \footnotesize   2012年度}\hspace{2ex} &  {\bfseries \footnotesize  增减额}      & {\bfseries \footnotesize  增减($\%$)}  \\  \endhead                % 4
\hline
\endfoot
\hline   % 内容开始
    长期待摊费用 & 328,860.83 & 734,631.97 & -405,771.14 & -55.23\\
\midrule
\bfseries 合计 & 328,860.83 & 734,631.97 & -405,771.14 & -55.23\\
\bottomrule
\end{longtable}


\subsection{债务状况分析}{}
从\tabref{asset-debt}中,中恒通公司在上一个会计年度内发生的权益总额较上年同期增加了34 894千元,增长幅度为9.46\%,说明公司在本年度权益总额有较大幅度的增长。进一步分析可知:
\begin{compactenum}[(1) ]
 \item 本年度公司负债总额减少了1 907千元,减少幅度为1.2\%。其中,流动负债减少8 477千元,减少幅度为5.81\%,使得权益总额减少了2.3\%。这主要是由于应付票据减少37 500千元,减少幅度达到65.79\%,对权益总额的影响为10.17\%,这说明公司在过去的会计年度里为应付票据的支付偿还。
 \item 所有者权益增加了36 802千元,同比增长幅度为17.54\%,对权益总额的影响为9.98\%。这主要是由于公司在本年度内未分配利润的大幅度增长引起的,同时,盈余公积增加3 765千元,增长幅度为39.72\%,对权益总额的影响为1.02\%。
 \item 本年度内公司权益总额的增长主要来自从当年盈利中保留的为分配利润,这表明该公司在本年度内经营还是卓有成效的。
\end{compactenum}











