% !Mode:: "TeX:UTF-8"
%% common definitions both for main and cover
%
%%%%%%%%%%%%%%%%%%%%%%%%%%%%%%%%%%%%%%%%%%%%%%%%%%%%%%%%%%%
% 重定义字号命令
%%%%%%%%%%%%%%%%%%%%%%%%%%%%%%%%%%%%%%%%%%%%%%%%%%%%%%%%%%%

\newcommand{\xiaochu}{\fontsize{30pt}{40pt}\selectfont}    % 小初, 1.5倍行距
\newcommand{\yihao}{\fontsize{26pt}{36pt}\selectfont}    % 一号, 1.4倍行距
\newcommand{\erhao}{\fontsize{22pt}{28pt}\selectfont}    % 二号, 1.25倍行距
\newcommand{\xiaoer}{\fontsize{18pt}{18pt}\selectfont}    % 小二, 单倍行距
\newcommand{\sanhao}{\fontsize{16pt}{24pt}\selectfont}    % 三号, 1.5倍行距
\newcommand{\xiaosan}{\fontsize{15pt}{22pt}\selectfont}    % 小三, 1.5倍行距
\newcommand{\sihao}{\fontsize{14pt}{21pt}\selectfont}    % 四号, 1.5倍行距
\newcommand{\banxiaosi}{\fontsize{13pt}{19.5pt}\selectfont}    % 半小四, 1.5倍行距
\newcommand{\xiaosi}{\fontsize{12pt}{18pt}\selectfont}    % 小四, 1.5倍行距
\newcommand{\dawuhao}{\fontsize{11pt}{11pt}\selectfont}    % 大五号, 单倍行距
\newcommand{\wuhao}{\fontsize{10.5pt}{10.5pt}\selectfont}    % 五号, 单倍行距
\newcommand{\xiaowu}{\fontsize{9pt}{9pt}\selectfont}    % 小五号, 单倍行距

\makeatletter
%% fill parbox to given width
\newlength{\@fillparboxlen}
\newcommand\fillparbox[2]{%
  \settowidth{\@fillparboxlen}{#2}
  \parbox[t]{\ifdim#1>\@fillparboxlen#1\else\@fillparboxlen\fi}{\hbadness 10000%
    \noindent\parfillskip 0pt #2}}
\newcommand\classification[1]{\def\XMUT@value@classification{#1}}
\newboolean{XMUT@boolean@confidential}
\newcommand\confidential[2]{\def\XMUT@value@confidential{#1}%
  \ifthenelse{\boolean{#2}}
    {\setboolean{XMUT@boolean@confidential}{true}}
    {\setboolean{XMUT@boolean@confidential}{false}}}
\newcommand\confidentialdate[3]{%
  \def\XMUT@value@confidentialdate@year{#1}%
  \def\XMUT@value@confidentialdate@month{#2}%
  \def\XMUT@value@confidentialdate@day{#3}}
\newcommand\UDC[1]{\def\XMUT@value@UDC{#1}}
\newcommand\serialnumber[1]{\def\XMUT@value@serialnumber{#1}}
\newcommand\studentsn[1]{\def\XMUT@value@studentsn{#1}}
\newcommand\school[1]{\def\XMUT@value@school{#1}}
\newcommand\degree[1]{%
  \def\XMUT@value@degree{#1}%
  \hypersetup{pdfsubject={厦门大学#1学位论文}}}
\renewcommand\title[2]{%
  \def\XMUT@value@title{#1}
  \def\XMUT@value@englishtitle{#2}
  \hypersetup{pdftitle={#1}}}
\renewcommand\author[1]{\def\XMUT@value@author{#1}
	\hypersetup{pdfauthor={#1}}}
\newcommand\advisor[2]{\def\XMUT@value@advisor{#1}%
    \def\XMUT@value@advisortitle{#2}}
\newcommand\major[1]{\def\XMUT@value@major{#1}}
\newcommand\submitdate[2]{%
  \def\XMUT@value@submitdate@year{#1}%
  \def\XMUT@value@submitdate@month{#2}}
  %\def\XMUT@value@submitdate@day{#3}}
\newcommand\defenddate[2]{%
  \def\XMUT@value@defenddate@year{#1}%
  \def\XMUT@value@defenddate@month{#2}}
  %\def\XMUT@value@defenddate@day{#3}}
\newcommand\grantdate[2]{%
  \def\XMUT@value@grantdate@year{#1}%
  \def\XMUT@value@grantdate@month{#2}}
  %\def\XMUT@value@grantdate@day{#3}}
\newcommand\chairman[1]{\def\XMUT@value@chairman{#1}}
\newcommand\appraiser[1]{\def\XMUT@value@appraiser{#1}}
\newcommand\team[1]{\def\XMUT@value@team{#1}}
\newcommand\fundteam[1]{\def\XMUT@value@fundteam{#1}}
\newcommand\lab[1]{\def\XMUT@value@lab{#1}}
%% add pdfkeywords, because this step must be done before \maketitle,
%% so we could not use the \keywords command in abstract.
\newcommand\pdfkeywords[1]{%
  \hypersetup{pdfkeywords={#1}}}
%
\makeatother
