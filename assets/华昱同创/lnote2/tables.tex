\chapter{表格}

\section{简单表格}

\texttt{tabular} 环境提供了最简单的表格功能。它用 \verb|\hline| 命令表示横线,\verb+|+ 表示竖线;用 \verb|&| 来分列,用 \verb|\\| 来换行;每列可以采用居中、居左、居右等横向对齐方式,分别用 \texttt{l、c、r} 来表示。

\begin{example}[h]
\LoadFBTDemo[numbers=left]{texlet/tab-simple}
\caption{简单表格}
\label{tab:simple_tab}
\end{example}

%\clearpage

在插图一章中我们介绍了一种图形浮动环境 \texttt{figure};表格也有一种类似的浮动环境 \texttt{table},其标题和交叉引用的用法和图形浮动环境类似。我们可以用它给 \autoref{tab:simple_tab} 中的表格穿件马甲,顺便把表格简化为科技文献中常用的三线表。\autoref{tab:simple_tab} 中的三条横线一样粗细,如果嫌它不够美观,可以使用 Simon Fear\indexFear{} \footnote{这名字看起来像个笔名。《圣经·新约》路加福音卷有一个故事:耶稣于某湖边传教时,命渔人西门撒网,得鱼甚多。西门惊诧,五体投地,口称罪人。耶稣道:西门莫怕,打渔这个职业没什么前途,你还是跟我打人罢 (Jesus said unto Simon, Fear not; from henceforth thou shalt catch men.) 。西门遂忝列夫子门墙。另外有个恐怖小说系列\emph{Fear Street}里的主角也叫西门怕怕。} 的 \texttt{booktabs} 宏包\citep{Fear_2005}。三条横线就分别用 \verb|\toprule、\midrule、\bottomrule| 等命令表示。改进后的表格见 \autoref{tab:threesome_tab}。

\begin{example}[h]
\LoadFBTDemo[numbers=left]{texlet/tab-threesome}
\caption{浮动三线表}
\label{tab:threesome_tab}
\end{example}

\texttt{tabular} 环境中的行可以采用居顶、居中、居底等纵向对齐方式,分别用 \texttt{t、c、b} 来表示,缺省的是居中对齐。列之间的分隔符也可以改用其他符号,比如用 \verb+||+ 来画双竖线。

\verb|语法:[纵向对齐]{横向对齐和分隔符}|

\section{宽度控制}

有时我们需要控制某列的宽度,可以将其对齐方式参数从 \texttt{l、c、r} 改为 \verb|p{宽度}|。这时纵向对齐方式是居顶,\texttt{t、c、b}等参数失效。

\begin{example}[h]
\LoadFBTDemo[numbers=left]{texlet/tab-width}
\caption{控制列宽}
\label{tab:width_tab}
\end{example}

使用宽度控制参数之后,表格内容缺省居左对齐。我们可以用列前置命令 \verb|>{}| 配合 
\verb|\centering、\raggedleft| 命令来把横向对齐方式改成居中或居右。列前置命令仅对紧邻其后的一列有效,其语法如下:

\verb|语法:>{命令}列参数|

\LoadCode[numbers=left]{texlet/tab-width-aligned-esc}

\begin{example}[h]
\LoadDemo{texlet/tab-width-aligned}
\caption{控制列宽和横向对齐}
\label{tab:width_aligned_tab}
\end{example}

若要控制整个表格的宽度,可以使用Carlisle\indexCarlisle{}的 \texttt{tabularx} 宏包\citep{Carlisle_1999}的同名环境,其语法如下,其中 \texttt{X} 参数表示某列可以折行。

\verb|语法:{表格宽度}{横向对齐、分隔符、折行}|

%\LoadCode{texlet/tab_tabularx_esc}
\begin{example}[h]
\LoadFBTDemo[numbers=left]{texlet/tab-tabularx}
\caption{控制表格宽度}
\label{tab:tabularx_tab}
\end{example}

如果想把纵向对齐方式改为居中和居底,可以使用Mittelbach\indexMittelbach 和Carlisle的 \texttt{array} 宏包\citep{Mittelbach_2009},它提供了另两个对齐方式参数:\verb|m{宽度}、b{宽度}|。

\section{跨行跨列}

有时表格某单元格需要横跨几列,我们可以使用 \verb|\multicolumn| 命令,同时使用 \texttt{booktabs} 宏包的 \verb|\cmidrule| 命令来画横跨几列的横线。它们的语法如下:

\verb|语法:\multicolumn{横跨列数}{对齐方式}{内容}|\\
\indent\verb|语法:\cmidrule{起始列-结束列}|

\begin{example}[h]
\LoadFBTDemo[numbers=left]{texlet/tab-multicol}
\caption{跨列表格}
\label{tab:multicol_tab}
\end{example}

跨行表格可以使用 \texttt{multirow} 宏包的 \verb|\multirow| 命令,其语法如下,

\verb|语法:\multirow{竖跨行数}{宽度}{内容}|

\begin{example}[h]
\LoadFBTDemo[numbers=left]{texlet/tab-multirow}
\caption{跨行表格}
\label{tab:multirow_tab}
\end{example}

\section{数字表格}

当表格中包含大量数字时,手工调整小数点和数位的对齐很麻烦,这时可以使用Rochester\indexRochester{} \footnote{曾就职于昆士兰大学昆虫系,后转到澳大利亚联邦科学与工业研究组织 (Commonwealth Scientific and Industrial Research Organisation) 。} 的 \texttt{warpcol} 宏包\citep{Rochester_2007}。它为 \texttt{tabular} 环境提供了一个列对齐参数 \texttt{P},其语法如下,其中 \texttt{m} 和 \texttt{n} 分别是小数点前后的位数,数字前的负号可选。

\verb|语法:P{-m.n}|

\begin{example}[!h]
\LoadFBTDemo[numbers=left]{texlet/tab-num}
\caption{数字表格}
\label{tab:num_tab}
\end{example}

\autoref{tab:num_tab} 中使用 \verb|multicolumn| 命令是为了保护表头,防止它们被 \texttt{P} 参数误伤。把跨列命令的列数设为1是设置单元格格式的一种常用方法。

\section{长表格}

有时表格太长要跨页,可以使用Carlisle\indexCarlisle 的 \texttt{longtable} 宏包\citep{Carlisle_2004}。这位同学对表格情有独钟,表格的宏包被他承包了一半。我们需要做以下工作:

\begin{enumerate}
    \item 首先用 \texttt{longtable} 环境取代 \texttt{tabular} 环境;
    \item 然后在表格开始部分定义每页页首出现的通用表头,表头最后一行末尾不用 \verb|\\| 换行,而是加一个 \verb|\endhead| 命令;
    \item 接着定义首页表头 (如果它和通用表头不同的话) ,同样地最后一行用 \verb|\endfirsthead| 命令结尾;
    \item 然后是以 \verb|\endfoot| 命令结尾的通用表尾;
    \item 然后是以 \verb|\endlastfoot| 命令结尾的末页表尾 (如果它和通用表尾不同的话) ;
    \item 最后是表格的具体内容。
\end{enumerate}

\LoadCode[numbers=left]{texlet/tab-long-esc}

\begin{Code}[numbers=left,firstnumber=last]
    & `天生丽质难自弃,一朝选在君王侧。`\\
    & `回眸一笑百媚生,六宫粉黛无颜色。`\\
    & `春寒赐浴华清池,温泉水滑洗凝脂。`\\
    & `侍儿扶起娇无力,始是新承恩泽时。`\\
    & `云鬓花颜金步摇,芙蓉帐暖度春宵。`\\
    & `春宵苦短日高起,从此君王不早朝。`\\
    & `承欢侍宴无闲暇,春从春游夜专夜。`\\
    & `后宫佳丽三千人,三千宠爱在一身。`\\
    & `金屋妆成娇侍夜,玉楼宴罢醉和春。`\\
    & `姊妹弟兄皆列土,可怜光彩生门户。`\\
    & `遂令天下父母心,不重生男重生女。`\\
    & `骊宫高处入青云,仙乐风飘处处闻。`\\
    & `缓歌慢舞凝丝竹,尽日君王看不足。`\\
    & `渔阳鼙鼓动地来,惊破霓裳羽衣曲。`\\
    & `九重城阙烟尘生,千乘万骑西南行。`\\
    & `翠华摇摇行复止,西出都门百余里。`\\
    & `六军不发无奈何,宛转蛾眉马前死。`\\
    & `花钿委地无人收,翠翅金雀玉搔头。`\\
    & `君王掩面救不得,回看血泪相和流。`\\
    & `黄埃散漫风萧索,云栈萦纡登剑阁。`\\
    & `峨嵋山下少人行,旌旗无光日色薄。`\\
    & `蜀江水碧蜀山青,圣主朝朝暮暮情。`\\
    & `行宫见月伤心色,夜雨闻铃断肠声。`\\
\end{longtable}
\end{Code}

\begin{longtable}{ll}
    \multicolumn{2}{r}{接上页} \\
    \toprule
    作者 & 作品 \\
    \midrule
    \endhead
\caption{长表格} \\
    \toprule
    作者 & 作品 \\
    \midrule
    \endfirsthead
    \bottomrule
    \multicolumn{2}{r}{接下页\dots} \\
    \endfoot
    \bottomrule
    \endlastfoot
    白居易 & 汉皇重色思倾国,御宇多年求不得。\\
    & 杨家有女初长成,养在深闺人未识。\\
    & 天生丽质难自弃,一朝选在君王侧。\\
    & 回眸一笑百媚生,六宫粉黛无颜色。\\
    & 春寒赐浴华清池,温泉水滑洗凝脂。\\
    & 侍儿扶起娇无力,始是新承恩泽时。\\
    & 云鬓花颜金步摇,芙蓉帐暖度春宵。\\
    & 春宵苦短日高起,从此君王不早朝。\\
    & 承欢侍宴无闲暇,春从春游夜专夜。\\
    & 后宫佳丽三千人,三千宠爱在一身。\\
    & 金屋妆成娇侍夜,玉楼宴罢醉和春。\\
    & 姊妹弟兄皆列土,可怜光彩生门户。\\
    & 遂令天下父母心,不重生男重生女。\\
    & 骊宫高处入青云,仙乐风飘处处闻。\\
    & 缓歌慢舞凝丝竹,尽日君王看不足。\\
    & 渔阳鼙鼓动地来,惊破霓裳羽衣曲。\\
    & 九重城阙烟尘生,千乘万骑西南行。\\
    & 翠华摇摇行复止,西出都门百余里。\\
    & 六军不发无奈何,宛转蛾眉马前死。\\
    & 花钿委地无人收,翠翅金雀玉搔头。\\
    & 君王掩面救不得,回看血泪相和流。\\
    & 黄埃散漫风萧索,云栈萦纡登剑阁。\\
    & 峨嵋山下少人行,旌旗无光日色薄。\\
    & 蜀江水碧蜀山青,圣主朝朝暮暮情。\\
    & 行宫见月伤心色,夜雨闻铃断肠声。\\
\end{longtable}


%\clearpage
\section{宽表格}

表格太宽时可以使用Fairbairns\indexFairbairns{} \footnote{1970年代剑桥数学学士,电脑硕士,现任剑桥网管。UK FAQ的维护者。} 等人的 \texttt{rotating} 宏包\citep{Fairbairns_2010}。其方法很简单,用 \texttt{sidewaystable} 环境替代 \texttt{table} 环境即可。

\begin{Code}[numbers=left]
\begin{sidewaystable}[htbp]
\caption{`主流英文词典`}
\label{tab:dict}
\centering
\begin{tabularx}{550pt}{Xllcrrr}
    \toprule
    Title & Abbr & Publisher & Year & Pages & Entries & Price \\
    \midrule
    Oxford English Dict, 2nd Ed & OED & Oxford Univ 
        & 1989 & 21,728 & 616,500 & 995 \\
\end{Code}

\LoadCode[numbers=left,firstnumber=last]{texlet/tab-rotate-esc}
\begin{sidewaystable}[htbp]
\caption{主流英语词典}
\label{tab:dict}
\centering
\begin{tabularx}{550pt}{Xllcrrr}
    \toprule
    Title & Abbr & Publisher & Year & Pages & Entries & Price \\
    \midrule
    Oxford English Dict, 2nd Ed & OED & Oxford Univ 
        & 1989 & 21,728 & 616,500 & 995 \\
    \midrule
    Shorter Oxford English Dict, 7th Ed & SOED & Oxford Univ 
        & 2007 & 3,888 & 600,000 & 175 \\
    New Oxford Dict of English, 2nd & NODE & Oxford Univ 
        & 2005 & 2,112 & 355,000 & 68 \\
    Webster's Third New International Dict & W3 & Merriam-Webster
        & 1961 & 2,816 & 476,000 & 129 \\
    American Heritage Dict, 4th Ed & AHD & Houghton Mifflin
        & 2000 & 2,112 & 90,000 & 60 \\
    Random House Webster's Unabridged Dict, 2nd Ed & Random & Random House 
        & 2005 & 2,256 & 315,000 & 69 \\
    \midrule
    Concise Oxford Dict, 11th Ed & COD & Oxford Univ
        & 2006 & 1,728 & 240,000 & \\
    Chambers Dict, 10th Ed & Chambers & Chambers Harrap 
        & 2006 & 1,872 & & 50 \\
    Collins English Dict, 9th Ed & Collins & HarperCollins
        & 2007 & 1,888 & & 67 \\
    Longman Dict of Contemporary English, 4th Ed & Longman & Longman 
        & 2005 & & 207,000 & 71 \\
    Merriam-Webster's Collegiate Dict, 11th Ed & & Merriam-Webster 
        & 2003 & 1,664 & 225,000 & 26 \\
    American Heritage College Dict, 4th Ed & & Houghton Mifflin 
        & 2007 & 1,664 & & 26 \\
    Random House Webster's College Dict & & Random House 
        & 2005 & 1,632 & & 26 \\
    Webster's New World College Dict, 4th Ed & & John Wiley \& Sons
        & 2004 & 1,744 & 160,000 & 26 \\
    \bottomrule
\end{tabularx}
\end{sidewaystable}

\lstset{firstnumber=1}

\section{彩色表格}

若想给表格增加点色彩,可以使用Carlisle的 \texttt{colortbl} 宏包\citep{Carlisle_2001}。它提供的 \verb|\columncolor、\rowcolor、\cellcolor| 命令可以分别设置列、行、单元格的颜色。这三个命令的基本语法相似:

\verb|语法:{颜色}|

\verb|\columncolor| 需要放到列前置命令里,\verb|rowcolor、\cellcolor| 分别放到行、单元格之前。\texttt{colortbl} 宏包可以使用 \texttt{xcolor} 宏包的色彩模型;两者同时,前者不能直接加载,需要通过后者的选项 \texttt{table} 来加载。三个命令同时使用时,它们的优先顺序为:单元格、行、列。

\begin{example}[h]
\LoadFBTDemo[numbers=left]{texlet/tab-color}
\caption{彩色表格}
\label{tab:color_tab}
\end{example}

\autoref{tab:color_tab} 这样的彩色表格过于花里胡哨,和包老师低调中年宅男的定位极不相称。\texttt{xcolor} 宏包的 \verb|rowcolors| 命令 (需要 \texttt{colortbl} 宏包的支持) 可以分别设置奇偶行的颜色,甚合吾意。该命令语法如下:

\verb|语法:{起始行}{奇数行颜色}{偶数行颜色}|

\begin{example}[h]
\LoadFBTDemo[numbers=left]{texlet/tab-color-alt}
\caption{彩色表格}
\label{tab:alt_color_tab}
\end{example}

\autoref{tab:alt_color_tab} 中代码第14行的 \verb|\hiderowcolors| 命令是用来暂停显示前面设置的奇偶行颜色,否则后面的其他表格会继续显示颜色。另一个命令 \verb|\showrowcolors| 可以用来重新激活奇偶行颜色设置。

\bibliographystyle{unsrtnat}
\bibliography{lnotes2}
